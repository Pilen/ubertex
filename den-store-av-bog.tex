\documentclass[10pt,a4paper,danish]{article}
\usepackage[danish]{babel}
\usepackage[utf8]{inputenc}
% \usepackage[margin=3cm]{geometry}
\usepackage[margin=4cm]{geometry}
\usepackage{amsmath}
\usepackage{amssymb}
\usepackage{listings}
\usepackage{fancyhdr}
\usepackage[hidelinks]{hyperref}
\usepackage{booktabs}
\usepackage{graphicx}
\usepackage{xfrac}
\usepackage[dot, autosize, outputdir="auto/"]{dot2texi}
\usepackage{tikz}
\usetikzlibrary{shapes}
\usepackage{ulem}
\usepackage{subcaption}
\usepackage{mdframed}
\usepackage{color}
\usepackage{marginnote}
\usepackage{pdfpages}
\usepackage{shorttoc}
\usepackage{setspace}

\setlength{\marginparwidth}{1.0in}
\setlength{\marginparsep}{0.1in}
\let\oldmarginnote\marginnote
\renewcommand{\marginnote}[1]{\oldmarginnote{\raggedright{}\footnotesize #1}}

\definecolor{boxygray}{gray}{0.98}
\mdfdefinestyle{boxy}{%
  backgroundcolor = boxygray}

\definecolor{notered}{RGB}{255, 180, 180}
\mdfdefinestyle{note}{%
  hidealllines = true,
  backgroundcolor = notered}
\newcommand{\note}[1]{\begin{mdframed}[style=note]\textbf{Note:}
    #1\end{mdframed}}

\newcommand{\bad}[1]{\begin{mdframed}[style=note]#1\end{mdframed}}

\definecolor{verbgray}{gray}{0.92}
\mdfdefinestyle{code}{%
  hidealllines = true,
  backgroundcolor = verbgray}

\let\oldv\verbatim
\let\oldendv\endverbatim

\def\verbatim{\mdframed[style=code]\oldv}
\def\endverbatim{\oldendv\endmdframed}

% \newcommand{\code}[1]{\mdframed[style=code]\texttt{#1}\endmdframed}
% \newcommand{\code}[1]{\texttt{#1}}
\newcommand{\code}[1]{\colorbox{verbgray}{\texttt{#1}}}

\newcommand{\commnt}[1]{}

\pagestyle{fancy}
\fancyhead{}
\fancyfoot{}
\lhead{\nouppercase\leftmark}
\rhead{Den Store AV-Bog}
\rfoot{\thepage}
% \setlength\parskip{1em}
\setlength\parskip{0.8em}
\setlength\parindent{0em}

%% Titel og forfatter
% \title{{\Huge{}\fontfamily{pzc}\selectfont{}Den Store AV-bog}}
\title{\Huge{}Den Store AV-Bog}
\author{Søren Pilgård - DIKUrevyen}

%% Start dokumentet
\begin{document}

%% Vis titel
\maketitle
\newpage
{\linespread{0.0}
  \shorttoc{Indhold (kort)}{1}
}

%% Vis indholdsfortegnelse
\tableofcontents
\newpage

\section{Introduktion}
\begin{center}
  {\Huge \textit{ ``Er der en AV-mand tilstede?!?''}}
\end{center}
Sådan kunne spørgsmålet lyde. Når du har gjort dig bekendt med denne bog er du
forhåbentligt istand til at svare \textit{``Ja!''}.
Denne bog henvender sig til dig der aldrig har lavet AV i en revy før og skal
have en udførlig introduktion, til dig der har lavet AV før men lige skal have
lidt hjælp til hvordan det nu var med det hersens AV-system, og til dig der har
lavet AV i mange år men stadig har behov for at slå op i dokumentationen i ny og
næ.

Dette er en introduktion til DIKUrevyens \textbf{AV-system}.
Systemet er udviklet af Søren Pilgård, og selv om det startede hos DIKUrevyen
har det allerede haft bred udbredelse blandt andet i DIKUrevyen,
SaTyRrevyen, Biorevyen og Matematikrevyen.
Systemet er af den slags der nok aldrig bliver helt færdigt, der er altid lige
noget mere der kan tilføjes, laves om og forbedres, det samme gælder denne bog.
Det bør dog ikke afskrække dig fra at kaste dig ud i det.

\marginnote{At vi kalder det AV-\textit{mand} skal du ikke tage så nøje, der findes super
  seje og kreative AV-folk af alle køn!}
Dette kapitel samt kapitel \ref{sec:god-av-mand} forklarer lidt generelt om hvad
det vil sige at lave revy og hvad AV-manden laver, samt nogle guidelines til
hvordan man gør det godt som AV-mand.


\subsection{Hvordan foregår en revy?}
Dette er måske din første revy, eller måske bare din første revy på Natfak.
Her vil vi for god ordens skyld kort gennemgå hvad en revy i denne sammenhæng
indebærer. De fleste revyer har en nogenlunde ens opbygning dette gælder især
for revyerne i SaTyRfællesskabet som indefatter DIKURevyen, Fysikrevyen,
Matematikrevyen, Biorevyen, MBKrevyen, og så selvfølgeligt SaTyRrevyen selv.
En revy er overordnet delt op i forskellige numre, f.eks. sange eller sketches,
disse numrer er, som regel, men bestemt ikke altid uafhængige af hinanden uden
overlap i hverken karakterer, tema eller indhold. Der sker dog ofte undtagelser
iform af f.eks. følgetoner med fortløbene handling, eller f.eks. sketches der
naturligt glider over i en sang. Numrene er grupperet i akter af en halv time
til tre kvarters varighed med en 15-20 minutters pause.
Der er typisk 2 eller 3 akter efterfulgt af et par ekstranumre.

En klassik revy foregår således at klokken 19.00 åbner dørene til foyeren,
typisk er de allerede åbne, men det er her folk begynder at stille sig i kø uden
foran dørene til auditoriet. Kl. 19.15 åbner dørene ind til auditoriet og
publikum strømmer ind. Når folk har fundet deres pladser sker det ofte at de
forskellige studieretninger begynder at synge af hinanden. Dette er en god
gammel tradition hvor nye svarvers dukker op og stemningen bliver sat. Det er
dårlig stil når revyen går i gang mens folk synger, så man skal sørge for at
time det før eller efter en sang. Når klokken er 19.30 og backstage er klar
signalerer en boss eller instruktør til \TeX{}nikken at nu kan vi starte.
Når der ikke synges sætter lysmanden så revyen i gang ved at dæmpe lyset i
salen.
Et øredøvende sus vil da gå igennem auditoriet som hele det feststemte publikum
fylder med råb, banken og hujen.

Nu er revyen i gang!\\


De fleste revyer starter med en bandintro, en video der præsenterer bandet, når
videoen er færdig sætter bandet sig til rette på bandscenen mens folk brøler
\textit{``Bandet!, Bandet!...''} nogle revyer har så en velkomst sketch hvor man
præsenterer revyen på en sjov måde (fysik starter f.eks. altid med at aflyse
revyen), andre går bare direkte over i en startsang der skal få gang i salen.

\marginnote{Nogle revyer foretrækker at holde 20 minutters pause, men skrive 15}[0cm]
Herefter følger en masse blandede sange og sketches i form af akt 1.
Man slutter typisk en akt med en sang med gang i så folk går feststemte til
pause.
Det er lysmandens opgave at åbne op for lyset, men AV manden starter typisk med
at have et pauseskilt/billede/video til at fortælle at nu er der pause først.
Herefter er der pause i 15-20 minutter mens publikum køber øl og går på
toilettet.
(Som AV-mand kan det også være en god ide at benytte toiletterne nede backstage
og fylde op på vand/drikkevarer nu).
Lidt før pausen er klar begynder bandet at spille ind, de spiller typisk noget
musik der passer til deres tema eller som de har brugt mens de øvede sig.

Så starter næste akt (hver opmærksom på at der altid er et par stykker der
kommer forsent og som konsekvent glemmer at lukke dørene (så hav en løber
klar)). I en 3 akters forestilling kører andet akt stille og roligt (sådan
konstruktionsmæsigt), i en 2 akters svarer det til at gå direkte til 3.
Hver opmærksom på at nogle 2 akters forestillinger kan have en tendens til at
have lidt længere akter. Sidste akt slutter af med et slutnummer, en god sang
der får folk op og køre. 3/4 inde i sangen kommer alle skuespillerne ud og der
siges tak til alle der skal have tak og alle på scenen bukker et par gange.
Så lunter folk ud af scenen og lyset går ned så der bliver helt mørkt. Imens
klapper folk ihærdigt og råber \textit{``Ekstra nummer!, Ekstra nummer!...''}
for det siger traditionen. Det er ofte enten AV eller lys der starter ekstra
numrene man skal altid trække den lidt, så trække den lidt mere, og så liiiige
trække den lidt mer', både fordi sådan er det, samt for at de kan nå at skifte
kostumer/gøre rekvisiter klar. Og så går ekstranumrene i gang.
Disse består som regel af 3-4 numre, nogle gange en video og slutter af med en
slut-slutsang. Denne sidste sang er handler ofte om at nu er revyen slut, men
der er efterfest bagefter og så skal den have fuld gas.
Så er revyen færdig, man kan smide et skilt op om at der er efterfest. Hvis man
har flere forestillinger men kun én efterfest kan man smide en henvisning om at
gå på \textit{Caféen?!} op.


Generelt prøver man ofte at holde det intelligente humor i første akt hvor folk
stadig kan forstå det og ikke ville ``kede'' sig, mens 3 akt er mere plat humor,
dårlige ordspil og billige punchlines. I takt med at publikum bliver mere og
mere fulde passer dette også meget fint til hvad de magter.
Publikum kan være helt fantastiske, men de kan også være larmende, forstyrende
og ret overvældende. Dette gælder både for dem på scenen, men det kan også være
skræmmende når \TeX{}nikken laver en fejl og hele auditoriet brøler
\textit{``TeXnikken sejler''}. Det vigtigste er at holde hovedet koldt i en hver
situation. Få ro på, tænk igennem hvad der går galt, lad være med at opildne
publikum, fokuser, og få tingene tilbage på sporet.
Vi har \textbf{ALLE} været derude hvor det hele gik galt, og bagefter gik det
hele altsammen, bare husk på ovenstående.

Hvis du er i tvivl om et begræb så tag et kig i kapitel \ref{subsec:hvad-er-hvad}.

\subsection{Hvad laver en AV-mand?}

Hvad er det så AV-manden laver til en revy?
Jo, ser du, TeXnikken består af 3 dele, der er lyd, de sørger for at bandets
musik, sangernes sang og skuespillernes grynten\commnt{Too harsh?} kommer ud af højtalerne.
Lydmanden laver så at sige ikke noget lyd selv, han sørger bare for at alle kan
høre det og at det lyder godt. Lysmanden sørger for at man kan se noget, det gør
han med et væld af farvede lamper og spots, med lys og skygge samt en gang
taktisk røg kan han transformere stemningen og skabe en fest på scenen.

\marginnote{Nogle revyer er også begyndt at have en Epelipsiadvarsel,
  i starten af revyen og før blinkende numre}
AV-mandens rolle er mere alsidig, en slags digital rekvisit man kan bruge til
lidt af hvert. Det kunne f.eks. være at agere powerpoint-show til en
``fremvisning'', vise billeder til at illustrere forskellige pointer, vise
videoer, vise pauseskilte, vise hvilke sponsorer der er, afspille digital musik
til dansenumre, afspille lydeffekter som pistolskud og prutter. En af de
vigtigste opgaver er dog at lave overtekster. Det kan være utroligt svært at
lave god lyd i Store UP1, hvis sangerne samtidigt er lidt usikre og der er en
masse larm i auditoriet kan det være utroligt svært at høre hvad der bliver
sunget. Samtidigt er der mange sange hvor man gerne vil give publikum mulighed
for at synge med i omkvædende, derfor har man valgt at indføre overtekster.
Overtekster er ligesom undertekster på en film, de sungne tekster bliver bare
vist på et hvidt lærred oppe bag scenen for at alle kan se dem.
Dette har vist sig at være en stor success.

\marginnote{\textbf{Pas på dig selv!} Det er nemt at brænde sig på et stort
  projekt.}
Hvis man er kreativ og har teknisk kunnen kan man også bevæge sig ud i mere
avancerede ting, animationer af alle mulige slags som passer til hvad der sker
på scenen. Revy-systemet understøtter en masse seje ting, og der kommer hele
tiden flere til så hvis man har mod på lidt programmering er det bare at give
sig i kast med effekterne. Men pas på, lad være med at bide over mere end hvad
du kan sluge. Alle disse avancerede detaljer er lækkerier der kan oppe
oplevelsen, men de kan aldrig stå alene og hvis du ser det som en for stor
opgave skal du ikke være bange for at spørge om hjælp eller sige fra.
\marginnote{Men når det er sagt: Man kan kun skabe noget nyt ved at prøve nye ting af.}[-1cm]



\subsection{Hvad er hvad?}
\label{subsec:hvad-er-hvad}
Her følger en række gængse termer der kan være praktiske at bruge, og andre
småting der er værd at huske.

\begin{description}
  % Revyen generelt
\item[Scene:] Til hver revy sættes en stor hjemmebygget træscene op i bunden af
  Store UP1.
\item[Bandscene:] En lille forhøjet scene til venstre for scenen, hvor bandet
  sidder og spiller.
\item[Bagtæppet:] Et tæppe til at aflukke til den bagerste meter af scenen, så
  man kan stille sig klar uset.
\item[Trekanten:]
\item[Højre:] Højre side af scenen set fra TeXnikken og publikum.
  Den side hvor folk oftest kommer ind fra.
  (Skuespillere kan ikke finde ud af højre og venstre)
\item[Venstre:] Venstre  side af scenen set fra TeXnikken og publikum.
  Den side hvor bandet sidder
  (Skuespillere kan ikke finde ud af højre og venstre)
\item[TeXnikken:] Lyd, Lys og AV, samt diverse hjælpere.
  Kortsagt dem der sidder foran scenen og arbjeder under revyen
\item[Backstage:] Området bag scenen. Bruges også om dem der arbjeder der. De
  andre ``skjulte'' roller som dem der laver rekvisiter og kostumer.
\item[Rekvisitten:] Dem der laver rekvisiterne/stedet de laver dem (backstage)
\item[Rekvisitkælderen:] Revyernes fælles opbevaringslokale, backstage under bandscenen.
  (Pas på skimmelsvamp)
\item[Dyrestalden:] En del af DIKU, gangen der ligger lige når man kommer ud af Store UP1 fra
  scenen.
\item[Omklædningen:] Backstage bag de store metaldøre og indtil trædørene står skuespillernes ting
  under revyen, gør det svært at komme igennem!
\item[Hyggeområdet:] Området bag skuespillernes omklædning, for foden af den sydlige trappe.
  Her findes ofte snacks og drikkevarer under revyen (loot det i pauserne)
\item[Harlem:] Et ``hemmeligt'' lokale i DIKUs kælder som revyerne råder over. Her øver
  bandet indtil de rykker ind i auditoriet. Her står også det meste af den
  texnik som revyerne råder over samt kasser med kabler.
  Sørg for at spørge før du låner størrere ting, husk at ryde op efter dig, husk
  at tape kabler sammen inden de dumpes. Spørg din revyboss om adgang hertil
\item[Indre Harlem:] Inde i Harlem findes indre Harlem der bruges til at redigere videoer samt de
  optagne film fra forestillingen.
  Her står det udstyr vi passer ekstra godt på.
  Kun de særligt betroede får adgang hertil og man skal spise en skovsnegl og
  kysse en datalog for at få lov!
\item[Adgang:] Husk at snakke med din boss om at få aktiveret dit studiekort så du kan komme
  ind på DIKU og i Harlem. I TeXnikken kommer og går man ofte på skæve
  tidspunkter så det er praktisk med adgang.
\item[Instruktør:] Dette er meget forskelligt fra revy til revy, men typisk en
  kreativ/kuntnerisk/skuespilsmæsig ansvarlig for de forskellige numre
\item[Boss:] Ansvarlig for hele revyen (bestemmer)
\item[Universitetsparken 1:] Stedet der i mange år har huset DIKU (i hvertfald
  de studerende). Skal forlades i år ....
\item[Store UP1:] Det store auditorium på Universitetsparken 1, også
  kendt som store knirke. Det er her alle revyerne på natfak afholder deres revyer.
\item[Caféen?!:] En af fredagsbarene på Natfak. Mange efterfester holdes her, og
  ofte sender man publikum til Caféen?! om fredagen efter første
  forestilling/generalprøven.
  % AV
\item[Overtex:] Et stort hvidt område over scenen under loftet hvor AV
  projekterer det meste af sine ting
\item[Højtex:] Et fancy begræb for alle de projektioner der er over Overtex på
  selve loftet
\item[Lavtex:] Et fancy begræb for alle de projektioner der er under Overtex,
  altså ned på selve scenen (antagelivis på et lærrede der opstilles undervejs)
\item[Fisk:] Et fancy ord for en kort film der ligger imellem to numre.
  Bruges ofte som et sjovt indslag nogen kom på samt til at give lidt ekstra tid
  til at nå et svært kostume/sceneskift.
\item[Datamat:] Fysisk implementation af Turings abstrakte maskine, hvad man på
  Engelsk ville kalde en ``computer''.
\item[\LaTeX:] Et ``sprog'' til at skrive og opsætte tekst, tænk HTML men til at
  lave nydelige artikler, formler og pdf'er. Også brugt til at lave overtekster.
\item[\TeX:] Forgængeren som \LaTeX er bygget på, de to termer bruges oftest
  synonymt.
\item[Emacs:] Gammelt tekstredigeringsprogram (Tænk notepad, men mere avanceret
  end Photoshop). Kan udvides helt utroligt og danner basis for DIKUrevyens AV-system.
\item[Script:] Et kort program, en sekvens af instruktioner for hvad der skal
  ske (automatisk) når det bliver kørt.
\item[Lisp] Et gammelt programmeringssprog. Der findes mange varianter men det
  er kernen i Emacs og udgør dermed måden man laver scripts på i AV-systemet
\end{description}


\subsection{Forudsætninger}
\note{TODO: Dårligt afsnit. Find ud af hvor det passer, og omforumler det til
  det her skal du lære}

Hvad skal man kunne for at bruge DIKUrevyens AV-system?

\begin{itemize}
\item \LaTeX
\item Linux
\item Emacs
\item Emacs Lisp
\item Python
\end{itemize}

\textbf{\LaTeX}\\
Bruges til at skrive selve overteksterne. Der bruges Beamer til at lave et langt
slideshow, overtex.sty definerer en række makroer der gør det let at lave en
lang præsentation.
De fleste på natfak burde kunne \LaTeX, hvis ikke er den mængde der bruges til
at lave normale overtekster ret lille og burde kunne mestres ved bare at kigge i
nogle af de gamle filer.

\textbf{Emacs}\\
Emacs bliver brugt som kontrolcenter til det hele.
Det er derfor væsentligt at have en hvis forståelse for hvordan Emacs virker.
Emacs er et voldsomt konfigurerbart program til at arbejde med tekst.
Jeg er selv stor Emacsbruger og har opbygget et helt unikt system.
På sigt er det håbet at lave en standard konfiguration som folk der ikke er
emacsbrugere kan udnytte. En sådan konfiguration ville kunne skjule at det
overhovedet er emacs der kører bag det hele.

\textbf{Emacs Lisp}\\
Emacs Lisp er det primære scripting sprog der bliver brugt.
Det bruges til at automatisere ting i Emacs, derudover kan der kaldes externe
kommandoer og kommunikere med andre maskiner.
Selve revysystemet bruger en stor del Emacs Lisp så hvis noget stopper med at
virke er det godt at kende. Til almindeligt AV brug kan man dog nøjes med en
stærkt begrænset del som denne guide nok skal introdusere.

\textbf{Linux}\\
DIKUrevyens AV-system er bygget og kører på linux styresystemet.
Det betydder at man for at bruge systemet effektivt er nød til at have en hvis
forståelse for at bruge en linuxmaskine igennem en terminal.
Et håb er en gang at have en form for standard opsætning af datamater hvor alt
nødvendigt er installeret, som AVmand skal man så blot sætte udstyrret op og
lave indholdet.

\textbf{Python}\\
Bliver brugt til de forskellige programmer og værktøjer der udgør resten af
AV-systemet.
Det burde kun være nødvendigt at kunne kode python for at udvikle/vedligeholde
systemet.


\newpage
\section{Tanken bag}
% \begin{center}
%   {\Huge \textit{ ``Hvad tænkte han dog på?''}}
% \end{center}
Den grundliggende filosofi bag systemet er at AV skal kunne komplementere eller understøtte en
revy uden at komme i vejen. Der skal dermed kun vises det som publikum skal
se og høre og intet andet.
Det kan være med til at bryde indlevelsen og virke utroligt amatøragtigt lige så
snart publikum ser en cursor, en film der maksimeres eller rammerne på et
vindue. Det er sådanne fejltagelser der hiver folk tilbage til virkeligheden i
auditoriet fremfor den fiktion man skaber.
Man skal ikke tænke over at det er AV, men at det er en mega fed revy, folk bør
således kun lægge mærke til AV når der sker noget ekstraordinært, og
forhåbentligt ikke på grund af noget der ikke skulle ske.

Det simpleste AV-system man kan lave (og som er det de fleste bare ville gøre
uden at tænke videre) er at have en datamat tilkoblet en
projektor. Datamaten kan derfra køre et presentationsprogram, f.eks. powerpoint.

\begin{figure}[h!]
  \centering
  \begin{dot2tex}
    digraph{
      rankdir=TD;
      node [shape=ellipse];
      d [label="Datamat"];
      p [shape=trapezium, label="Projektor"];

      d -> p;
    }
  \end{dot2tex}
  \caption{Simpel AV opsætning.}
\end{figure}

Dette er dog ret primitivt.
Det er utroligt nemt at komme til at lave fejl, så som lige at få skubbet
cursoren hen på et andet desktop eller at filmen spiller på den forkerte skærm.
Ting der er sket gang på gang til mange revyer, og som hurtigt får publikum til
at råbe \textit{``TeXniken sejler!''}.

Derudover har man et problem hvis man skal bruge mere end 1 projektor.

\begin{figure}[h!]
  \centering
  \begin{dot2tex}
    digraph{
      rankdir=TD;
      node [shape=ellipse];
      d [label="Datamat"];
      p1 [shape=trapezium, label="Projektor"];
      p2 [shape=trapezium, label="Projektor"];
      p3 [shape=trapezium, label="Projektor"];

      d -> p1;
      d -> p2;
      d -> p3;
    }
  \end{dot2tex}
  \caption{En datamat, flere projektorer.}
\end{figure}

\begin{figure}[h!]
  \centering
  \begin{dot2tex}
    digraph{
      rankdir=TD;
      node [shape=ellipse];
      d1 [label="Datamat"];
      d2 [label="Datamat"];
      d3 [label="Datamat"];
      p1 [shape=trapezium, label="Projektor"];
      p2 [shape=trapezium, label="Projektor"];
      p3 [shape=trapezium, label="Projektor"];

      d1 -> p1;
      d2 -> p2;
      d3 -> p3;
    }
  \end{dot2tex}
  \caption{Flere datamater, flere projektorer.}
\end{figure}

Man kan enten koble flere projektorer på en datamat, hvilket kræver en datamat
der er i stand til dette, hvilket ikke særligt mange er. Og det kan give en
hovedepine hvis man samtidigt prøver at holde det ``skjult''.
Alternativt kan man have flere datamater koblet til hver sin projektor.
Nu skal man så bare navigere rundt mellem en helt masse maskiner eller have en
AV-mand pr. maskine hvilket heller ikke er særligt praktisk.
Desuden har vi stadig problemet med at man let kommer til at dumme sig på samme
måde som i den første opsætning.

Det vi i virkeligheden ønsker os er en abstraktion mellem det at \textit{styrre} AV og
det at \textit{vise} AV. Hvis tingene kører adskilt og har klart definerede
ansvarsområder kommer man ikke lige så let til at lave fejl.
Vi ønsker derfor at have en central maskine der styrer det hele, denne
kommunikerer med andre maskiner der sørger for at vise ting via projektorer.


\begin{figure}[h!]
  \centering
  \begin{dot2tex}
    digraph{
      rankdir=TD;
      node [shape=ellipse];
      c [label="Kontrol datamat"];
      d1 [shape=box, label="Arbejder Datamat"];
      d2 [shape=box, label="Arbejder Datamat"];
      p1 [shape=trapezium, label="Projektor"];
      p2 [shape=trapezium, label="Projektor"];

      c -> d1;
      c -> d2;
      d1 -> p1;
      d2 -> p2;
    }
  \end{dot2tex}
  \caption{Central styring, flere arbejdere med hver deres projektor.}
\end{figure}

Det er dette princip DIKUrevyens AV-system benytter sig af.

\note{Lidt baggrundshistorie for de nørdede.\\
  Det første hjemmelavede AV-system blev lavet af Troels Henriksen.
  Han lavede et programmeringssprog \textit{Sindre} i Haskell til at lave
  grænseflader.  I det sprog lavede han et \texttt{dmenu} lignende program
  kaldet \textit{sinmenu} som han brugte som interface. Selve overteksterne
  bestod af en pdf som han lokalt oversatte til rå tekst som blev fodret ind i
  et script der brugte hans grænseflade. Dette script kommunikerede med en
  server koblet til en projektor over en sshforbindelse.  \texttt{Xpdf} kan
  køres som en server der kan modtage kommandoer fra en kommandolinje, på denne
  måde kunne man derfor styre overtekster over ssh.

  Desværre skalerede løsningen ikke særligt godt. Da der begyndte at komme mange
  AV-effekter kunne man ikke både køre overtekster og andet AV alene.

  Dette blev (i hvertfald forsøgt) rettet op på da jeg (Søren Pilgård) efterfølgende begyndte at skrive et nyt
  AV-system til DIKUrevyen 2012.
  At skrive et AV-system er dog ikke en nem tjans og det endte med at foregå i en
  løbende process med forbedringer til hver revy.
}
\newpage
\note{Herfra skal der skrives om, det er mere en ``se hvad jeg har lavet'' end et
  ``nødvendigt mindset''
  \hrule
}
DIKUrevyens AV-system består af en central grænseflade der kan kommunikere med flere
forskellige maskiner der kan vise AV materiale.

Selve grænsefladen er udviklet som en række udvidelser til Emacs.
Grunden til dette er at da jeg startede indså jeg at systemet skulle kunne
følgende ting:
\begin{itemize}
\item Kommunikere med en server der viser indholdet.
\item Det skulle kunne vise ting overskueligt i en grafisk grænseflade.
\item Der skulle være mulighed for at indlejre scripts så man kan køre ting
  automatisk på bestemte steder i forestillingen.
\item Det ville være praktisk hvis man kunne rette fejl i texkoden mens man
  viste pdf'en da man ellers risikere at glemme dem.
\end{itemize}

Det gik hurtigt op for mig at det ville være fjollet at udvikle noget fra bunden
da Emacs i forvejen kunne meget af dette. Desuden er jeg Emacsmand og så hurtigt
hvordan en integration ville være nice.
Den grundliggende formel for at vise overtekster er at man åbner et LaTeX
dokument der bruger beamer pakken til at lave overteksterne. Så starter man
\texttt{ubertex}minor modet, dette sørger for at pdfen bliver lagt op på
serveren. Herefter skjules de fleste LaTeXkommandoer og et overlay lægges der
viser hvad der bliver vist. Man kan så trykke ``næste'' hvilket rykker overlayet ned
og synkroniserer serveren til at vise det tilsvarende slide i pdfen.
Man kan også trykke vilkårlige steder i tex filen og rykke direkter hertil i
overteksterne. Desuden kan der indsættes kode der eksekveres når man når til det
pågældende slide.

% TODO: billede

Derudover findes minormodet \texttt{uberscript} der lader en afvikle scripts.
Det kunne f.eks. være en sketch med en række lydeffekter eller et kald til en video.
Et script kunne også være aktoversigten hvorfra man ved at trykke ``næste''
kommer ind i det næste nummer, og når dette er færdigt kommer man så tilbage og
er klar til næste.

Hvis alt går som det skal, skal man som AVmand kun trykke på en knap (næste) for
at afvikle en revy.
Dette må være essensen af et godt AV system, når man kun skal tage sig af
timingen på skuespillerne/sangerne samt disses fejltagelser.

\section{- Guidelines til en AV mand}
% \label{sec:god-av-mand}
% \begin{center}
%   {\Huge \textit{ ``Lyt til mig en gang''}}
% \end{center}


\textit{TODO: Her kommer der til at være en række mere generele råd om hvordan
  man laver god AV}

Det vigtigste man skal huske for at være en god AV-mand, er at have det sjovt!
Hvis du ikke har det sjovt, bliver det ikke en sjov revy.
Så lad være med at stresse for meget, ignorerer når de andre sejler og glemmer
at give dig hvad du skal bruge og hyg dig.
Hvis du først lader dig blive presset af det hele mister du overblikket og
kreativiteten, så hellere sige ``Det går nok'' og tage det som det kommer, det
skal nok blive en revy.

I dette kapitel findes en række råd og guidelines. Hvordan man laver AV er en
smagssag og afhænger meget af hvad man laver og sammenhængen. AV er således en
kunst, og selvom det der står her kan lyde som regler, ved enhver kunstner
hvornår man skal bryde dem. Nogle gange bryder man reglerne for at få tingene
til at gå op i en højere enhed, andre gange for at lave et helt nyt regelsæt.
Selvom du måske er anarkist og vil gå din egne veje som AV-mand vil jeg dog
anbefale dig at læse og forstå hvad der står her.

\note{Husk der er mange måder at lave AV på, det her er nogle af de erfaringer
  der har virket godt}

\note{Sæt altid højere standarder til dig selv end andre}

\subsection{Om at lave gode overtekster}
% 6 ord
% Læser i høj grad på ordenes form.
% Drfr kn mn stdg lse dnne stnng
% Sætningen læses relativt hurtigt, men var den længere ville man skulle læse hele
% teksten.
% Man kan ca overskue

% <billede af en 4 lego brik>
% kan genkendes med det samme

% <billede af en 9'er lego klods>
% man skal tælle





% husk blanke slides

\newpage
\section{- Opsætning af tekniken}
% \begin{center}
%   {\Huge \textit{ ``Det er faktisk slet ikke så svært''}}
% \end{center}

Til en standard opsætning ala DIKUrevyens skal du bruge:
\begin{figure}[h!]
  \centering
  \begin{dot2tex}
    digraph{
      rankdir=TD;
      node [shape=ellipse];
      c [label="Kontrol datamat"];
      h [shape=octagon, label="LAN-hub"];
      d1 [shape=box, label="Brok"];
      d2 [shape=box, label="Intro"];
      p1 [shape=trapezium, label="Projektor"];
      p2 [shape=trapezium, label="Projektor"];

      c -> h;
      h -> d1;
      h -> d2;
      d1 -> p1;
      d2 -> p2;
    }
  \end{dot2tex}
  \caption{Standard opsætning}
\end{figure}

Hvis det ønskes (Når det er færdigt) kan man køre med fjernstyrede projektorklapper.
\begin{figure}[h!]
  \centering
  \begin{dot2tex}
    digraph{
      rankdir=TD;
      node [shape=ellipse];
      c [label="Kontrol datamat"];
      h [shape=octagon, label="LAN-hub"];
      a [shape=diamond, label="arduino"];
      d1 [shape=box, label="Brok"];
      d2 [shape=box, label="Intro"];
      p1 [shape=trapezium, label="Projektor"];
      p2 [shape=trapezium, label="Projektor"];

      c -> h;
      h -> d1;
      h -> d2;
      h -> a;
      a -> p1;
      d1 -> p1;
      d2 -> p2;
    }
  \end{dot2tex}
  \caption{Standard opsætning, med arduino til projektorklap}
\end{figure}

\newpage
Systemet kan udvides, her ses f.eks. en opsætning med højtex (uden
projektorklapper) som brugt til DIKU Jubilæumsrevy.
\begin{figure}[h!]
  \centering
  \begin{dot2tex}
    digraph{
      rankdir=TD;
      node [shape=ellipse];
      c [label="Kontrol datamat"];
      h [shape=octagon, label="LAN-hub"];
      d1 [shape=box, label="Brok"];
      d2 [shape=box, label="Intro"];
      p1 [shape=trapezium, label="Projektor"];
      p2 [shape=trapezium, label="Projektor"];
      h0 [shape=box, label="left"];
      h1 [shape=box, label="mid"];
      h2 [shape=box, label="right"];
      hp0 [shape=trapezium, label="Projektor"];
      hp1 [shape=trapezium, label="Projektor"];
      hp2 [shape=trapezium, label="Projektor"];

      c -> h;
      h -> d1;
      h -> d2;
      d1 -> p1;
      d2 -> p2;

      h -> h0;
      h -> h1;
      h -> h2;
      h0 -> hp0;
      h1 -> hp1;
      h2 -> hp2;
    }
  \end{dot2tex}
  \caption{Højtex}
\end{figure}


\subsection{Kontrol}
Dette er din primære indgang til systemet. Jeg anbefaler at man bruger en
Foldedatamat, gerne ens egen bærbare.
En bærbar har den fordel at skærmen kan indstilles til både at sidde ned og stå
op. Derudover har den tastatur og mus indbygget så det ikke fylder i den ellers
rodede texnik. Og så kan man tage den med sig så man kan arbejde videre andre
steder. F.eks. er det praktisk at kunne plugge den ud så man kan arbejde videre
på sine overtekster til et ellers kedeligt senemøde i hyggehjørnet.


\subsection{Projektorer}
Til DIKUrevyen bruges der typisk i hvertfald 2 projektorer.
Der bruges én der peger op på lærredet over scenen, her vises der overtekster,
almindelige av-ting til sketches samt små film.

Kantinen ejer en stor Epson som bruges til introen/av på scenen

DIKUs projektorer bruges til alt andet.
DIKU har en række forskellige projektorer, hvor det kan være svært at kende
forskel på en del af dem.
%% TODO: udvid med mere konkrete råd om projektorer.

En god tradition, som en AV-mand bør holde i hævd, er at rense projektornes filtre
når man henter dem i begyndelsen af revyugen.
Der er tilsyneladende ikke andre der gør det, så lad det blive en del af
rutinen. Så overopheder de ikke lige så nemt.


Til projektorene er der bygget en række projektorkasser af gamle colakasser.
Disse gør det en del nemmere at indstille projektorene.\\
\textit{I gamle dage, da jeg var ond. Da blev projektorene stablet på bøger til
  de stod sirligt, den ærede AV-mand Troels Henriksen brugte mangt en stund på
  at bande og svovle når disse blev rykket}\\
Nu gør projektorkasserne det en del nemmere da man kan stripse/tape kasserne
fast og det hele bliver langt mere stabilt.
Desuden kan der komme langt mere luft til.\\
\textit{I gamle dage, da jeg var ond. Da blev projektorne så varme at de
  overophedede, så vi måtte til HCØ og hente tøris til køling. Det gav også
  kolde drikkevare (til tider frosne).}

Se afsnittene om Ubertex, Uberscript og Emacs for at finde ud af hvordan
softwaren bruges.

\subsubsection{Projektorklapper}
En ulempe ved projektore er at deres ``sorte'' ikke er mangel på lys, men bare
\textit{mindre} lys. Det betyder at når alt lyset i StUP1 er slukket og en
projektor står og lyser, er der stadig ret meget lys på scenen. Det ser
\textbf{MEGET} dumt ud og gør at man kan se hvad der sker på scenen.
Det er primært et problem for projektoren til overteksterne og kan accepteres
til f.eks. højtex (da loftet ikke er lige så reflekterende)

Det er desværre ikke en løsning at slukke/tænde projektorne da det tager for lang
tid/er besværligt/ skydder farver op når de tændes.

Derfor bruges en `projektorklap'. Der har i mange år været brugt en halv
papkasse på en stang. Det fungerer, men det kan godt være lidt stressene. Man
skal huske at få klappen på når man er færdig med en sang og har travlt med at
huske hvad der nu skal ske. Til tider sker det også at man glemmer at tage klappen
af, man når typisk at panikke lidt når der ikke kommer noget billede frem. Det
kan betyde at man misser de første par overtekster eller starten af en film.


For at løse dette er det planen at der bygges nogle arduinoer der kan kobles på netværket, disse
kontrolerer en klap foran projektoren. På denne måde kan projektoren automatisk
begynde at skyde billedet op på lærredet
Arduino ftw!

\subsection{Netværk}
\subsubsection{Statisk IP}
Hvis dit netværks interface hedder enp0s25
\begin{verbatim}
sudo ip link set enp0s25 up
sudo ip addr add 192.168.0.XXX/24 dev enp0s25
\end{verbatim}

alternativt
\begin{verbatim}
sudo ifconfig eth0 192.168.0.XXX
\end{verbatim}

\subsubsection{Hosts}
Da det kan være svært at huske alle ip-addreserne kan man i stedet navngive
maskinerne.
Dette gøres lokalt i \texttt{/etc/hosts}
et eksempel:

\begin{verbatim}
192.168.0.20    intro
192.168.0.30    brok
192.168.0.40    left
192.168.0.50    mid
192.168.0.60    right
\end{verbatim}

Nu kan man ssh'e ind ved blot at skrive:

\begin{verbatim}
ssh brok
\end{verbatim}

Det virker også med scp ol.

\subsubsection{Forbindelse uden login}
For at logge ind på systemerne over netværket bruges ssh.
Da det bliver jævnt irriterende hele tiden at skulle taste løsener kan man lægge
sin offentlige ssh nøgle ind på de forskellige maskiner
\begin{verbatim}
cat .ssh/authorized_keys | ssh revy@192.168.0.30 "cat >> ~/.ssh/authorized_keys"
\end{verbatim}
det kan være du først skal oprette mappen \texttt{.ssh}.

alternativt kan du bruge
\begin{verbatim}
ssh-copy-id revy@192.168.0.30
\end{verbatim}
Udskift \texttt{pilen} og ip'en med den relevante bruger og ip.
Husk at du skal have en ssh nøgle først
dannes med:

\begin{verbatim}
ssh-keygen
\end{verbatim}


\subsubsection{ssh mount}
I stedet for at scp'e alt muligt crap frem og tibage kan man benytte sig af et
ssh mount, som er et filsystem over ssh.
Jeg anbefaler at man har filerne liggende lokalt og fra hver maskine der skal
kommunikeres med køres \texttt{sshfs} som får filerne frem.

\textbf{Installer sshfs}
F.eks.:
\begin{verbatim}
sudo pacman -S sshfs}
\end{verbatim}

\textbf{fuse}
\texttt{sshfs} benytter fuse
\begin{verbatim}
sudo modprobe fuse
\end{verbatim}
\subsubsection{Gateway}
Hvis du er intereseret i at få internet på det lokale netværk kan man lave en
gateway.
Dette består i at en af datamaterne på det lokale netværk også er forbundet til
et netværk med internet, f.eks. eduroam.

Der er en guide på \url{sigkill.dk/writings/guides/gateway.html}

På gateway datamaten sættes først den statiske ip (allerede gjordt i forrige
trin)
Her efter køres gateway.sh som root (hentes på siden)

På clienterne der vil udnytte gatewayen sørger man først for at der er en
statisk ip.

herefter skrives
\begin{verbatim}
route add default gw 192.168.0.XXX
\end{verbatim}

eller
\begin{verbatim}
ip route add default via 192.168.0.XXX
\end{verbatim}
Hvor ``192.168.0.XXX'' er ip'en til gatewaymaskinen

herefter sættes en dns server, f.eks. google
\begin{verbatim}
echo nameserver 8.8.8.8 > /etc/resolv.conf
\end{verbatim}

Hvis dette ikke virker kan man evt. kigge på
\url{https://wiki.archlinux.org/index.php/Internet_Share} for en ip baseret løsning


\subsection{Arbejdere}
Arbejdere er de maskiner/servere der er sluttet til projektorne. Det er deres
rolle at vise AV-indholdet det kunne være overtekster, film, lydeffekter,
slides, billeder osv.

På nuværende tidspunkt bruges \texttt{mplayer} til at vise film, \texttt{feh}
til at vise billeder og \texttt{xpdf} til at vise pdf'er.
Det smarte ved \texttt{xpdf} er at det kan startes som en server som kan modtage
kommandoer via en terminal.

Ideen er at man fra sin kontroldatamat kan ssh'e ind på arbejderne og afvikle
den kommando man skal bruge. Emacs med ubertex/uberscript tilbyder basalt set et
interface til at gøre dette nemt.

Det er planen at programmet \texttt{Zeigen} udvikles.
Dette skal erstatte \texttt{feh} og \texttt{xpdf}.
I stedet for at skulle ssh'e ind lytter \texttt{Zeigen} til en port og udfører
kommandoer på givne klokkeslet. Så kan man sige ``om 1/2 sekund skal du afspille
denne video'' på 3 forskellige maskiner, hvorved videoen vises simultant via
\texttt{mplayer}.


Et eksempel på en arbejderdatamat er \texttt{brok} der bruges til overtekster og ting der
skal vises på lærredet.
Derudover findes \texttt{left}, \texttt{mid} og \texttt{right} der bruges til at vise \textit{højtex}.


\subsection{Brok}
\textbf{Her er en guide om at få brok til at virke:}

Sæt ham til
Projektor, tastatur, netværk og strøm.

Hvis der skal spilles lyd fra brok bør der være jord på da der ellers kan komme
en brummen.

Brok har på skrivende tidspunkt ubunut 11.04 natty.
Det betyder også at der er gnome, unity og ting og ækle sager på.
Disse har en tendens til at gøre livet surt, da de overskriver xorg
konfigurationer og gør mærkelige ting man ikke helt kan gennemskue.

\textit{Dette er \uline{IKKE} optimalt!
  Support til 11.04 udløb oktober 2012. Der bør installeres et ordentligt
  styresystem. UDEN gui (installeres seperat).}

\textbf{1: Tænd brok}\\
Brok logger automatisk ind med brugeren \textbf{\textit{revy}} med løsnet \textbf{\textit{hamster}}
Herefter starter unity der har en `kør' dialog ting.

\textbf{2: Start en terminal}\\
Skriv \code{terminal} og tryk enter.
Der vil nu komme en terminal op i hjørnet.

\textbf{3: Kom på netværket}\\
\begin{verbatim}
sudo ifconfig eth0 192.168.0.30
\end{verbatim}
Giv ham en passende ip du kan huske.

\textbf{4: Opret SSH forbindelse}
Der kører allerede en ssh daemon.
forbind til \textbf{\textit{revy}} med løsnet \textbf{\textit{hamster}}.
Du kan evt. uploade din offentlige ssh-nøgle så du slipper for at logge ind.
\begin{verbatim}
ssh revy@192.168.0.30
\end{verbatim}

\textbf{5: Stop gdm}\\
Gnome Desktop Manager skal lukkes.
\begin{verbatim}
sudo service gdm stop
\end{verbatim}
Dette lukker hele den grafiske grænseflade, inklusiv X.

\textbf{6: Start screen}\\
\begin{verbatim}
screen
\end{verbatim}
screen køres så vi kan starte X i baggrunden.
Når programmet startes vises der en menutekst, bare tryk enter.
\texttt{screen} kører nu.

\textbf{7: startx}\\
Start X manuelt.
\begin{verbatim}
startx
\end{verbatim}
Dette starter X op hvorefter .xinitrc eksekveres.

Som det er lige nu startes \texttt{xmonad} som windowmanager.
Det betyder at skærmen pr. default er sort når x er startet


\textbf{8: detach fra screen}\\
Tryk \texttt{Ctrl-a d}. Dette lader alt der kører i screen fortsætte upåvirket
mens du kommer tilbage til det forrige miljø.
for at reattache skriv da \code{screen -r}.

\textbf{9: test xmonad}\\
Test om xmonad virker. På broks tastatur trykkes \texttt{Alt-Shift-Enter}.
En terminal burde åbne sig.
Luk igen med \texttt{Alt-Shift-c}.

\textit{Læs evt. afsnitet om xmonad.}

Da terminalen er en default \texttt{xterm} med hvid baggrund er dette et snedigt trick
til at se hele området projektoren kan lyse op.
Jeg bruger dette ofte når jeg tweaker projektoren.



\textbf{10: prøv ting af}
Med en ssh forbindelse åben, lad os prøve om vi kan få noget til at virke.
Start med at vælge det `display' der skal vises grafiske ting på
\begin{verbatim}
export DISPLAY=:0
\end{verbatim}
Dette skal køres for hver gang du ssh'er ind.

start nu xpdf med en af de gamle overtekst filer der ligger et eller andet sted.

Se sektionen om kommandoer for nærmere detaljer

Hvis man gerne vil lave flere ting simultant på brok kan man sagtens åbne flere
lokale terminaler og ssh'e ind parallelt.




\subsubsection{Ting der køres på brok}
For at hacke uden om alt muligt gøjl køres et par scripts gennem
\texttt{.xinitrc}

\textbf{swarp}\\
\texttt{swarp} er et program der flytter musse-markøren (cursoren) til et givent koordinat.
Swarp findes på \url{tools.suckless.org/swarp}.
\texttt{swarp} findes desuden i arch's repository, det kan være den også findes til
debian baserede styresystemer, potientielt i en suckless pakke.

\texttt{swarp} køres på brok med argumenterne 20000 20000, hvilket flytter
markøren ad helvede til.

istedet for \texttt{swarp} kan man bruge \texttt{unclutter}
med kommandoen \code{unclutter -idle 0 -root \&}
Som i fjern cursoren efter 0 sekunders delay efter bevægelse inklusiv når
cursoren er over rod baggrunden (altså ikke kun over vinduer).

\textbf{fixsleep}
\texttt{fixsleep.sh} er et script der forsøger at forhindre X i at slukke skærm outputet.
X bruger et system der hedder DPMS (Display Power Management Signaling) til
automatisk at slukke skærm outputet efter en periode uden tastaturaktivitet.

fixsleep benytter følgende to kommandoer
\begin{verbatim}
xset dpms 0 30000 40000
xset s 30000
\end{verbatim}
den første linje dækker over \texttt{xset dpms [standby [suspend [off]]]}
den anden linje dækker over \texttt{xset s [timeout [cycle]]}

\textbf{keepon}
\begin{verbatim}
xset dpms force on
\end{verbatim}
Denne kommando forcer dpms fra, (det svarer til at at trykke på en tast)

\texttt{keepon.sh} er et script der ligger i baggrunden og kører denne kommando
whert 30 sekund.


fixsleep og keepon forsøger begge at holde dpms stangen ved at lave aktivitet.
Man kan måske istedet bruge
\begin{verbatim}
xset -dpms; xset s off
\end{verbatim}
Til at slå dpms fra.
De to systemer kan dog \uline{ikke} blandes.

Man kan se om dpms er slået til ved at kalde \code{xset -q}
\subsection{Xmonad}
Windos/super eller alt som modifier knap.

\section{- Master}
% Der sørges automatisk for at det korrekte major mode køres på .tex og .el
% filer
\subsection{ubersicht}
\subsection{ubertex}
\subsection{Konvertering fra manuskrift til overtekster}
\section{- Worker}
\note{Generelt om hvordan en worker virker}
\section{- Lisp}
Vectorer er ikke ens.
I Emacs lisp er de konstante. I Worker Lisp er de dynamiske arrays. ak indsætte
foran og bagi i O(1) tid (ammortiseret)

Lister er en slags ``meta'' type over consceller. En konvention!
En liste er en Cons-celle hvor \texttt{car} indeholder et element i listen, og
\texttt{cdr} indeholder en liste (resten).
Alternativt er en liste \texttt{nil} som angiver den tomme liste eller
slutningen på en liste.
Der er således ikke en konkret type i sproget der hedder \textit{list}, men man
bruger det som om der var.

Quote opfører sig anderledes
(Skal jeg lade være med at deep copy'e?

\subsection{Hvordan programmerer man?}
\subsection{De forskellige lisp dele}
\section{\LaTeX}
\section{- Værktøjer}
\subsection{Schneider}
\textit{Tysk: Skrædder}

\textbf{Dependencies:}
\begin{itemize}
\item Python3
\item ffmpeg
\end{itemize}

Schneider er et værktøj til at skære film og billeder op til at kunne blive vist over flere
projektorer.

Schneider kan på skrivende stund kun dele medier op i flere snit ved siden af
hinanden og ikke over under. Det burde dog være let at rette til.

\subsection{Zeitherr}
\textit{Tysk: Timelord}
Er en bastardiseret udgave af en NTP tidsserver til at holde de forskellige
maskiner synkroniseret.

\subsection{Zeigen}


\note{Externe:}
\subsection{xpdf}
Overtekster køres i xpdf der åbnes manuelt med:
\begin{verbatim}
xpdf -remote ubertex -fullscreen -mattecolor black -fg black
    -bg black -papercolor black filnavn
\end{verbatim}
Da dette er meget langt kan man istedet bruge aliaset
\begin{verbatim}
p filnavn
\end{verbatim}


\subsection{mplayer}
Til at vise videoer manuelt bruges mplayer:
\begin{verbatim}
mplayer -nolirc -msglevel all=-1 -msglevel statusline=5
    -vo gl2 -autosync 30 -cache 1048576
    -cache-min 99:100 -xy 500 -geometry 49%:40% filnavn
\end{verbatim}
eller aliaset
\begin{verbatim}
m filnavn
\end{verbatim}

\subsubsection{Positionering af mplayer}
\textit{TODO: Noget om positionering af mplayer her}


\section{- FAQ}
Ubertex tager ikke højde for pauser i comments, latex gør.

\subsection{\LaTeX{}}
\subsubsection{File `overtex.sty' not found.}
\texttt{overtex.sty} er en fil hvori de overtex speciffike kommandoer er
defineret. \texttt{revy-manus-prepare} indsætter automatisk
\code{\textbackslash{}usepackage\{overtex\}} i overtexfilerne. Når man kalder
\texttt{revy-compile-tex} specificeres automatisk hvor denne fil findes. Kaldes
pdflatex manuelt ved den ikke hvor denne fil findes.
Den simpleste løsning er at oversætte med et kald til \texttt{revy-compile-tex}.

Alternativt kan man enten kopiere \texttt{overtex.sty} ind i mappen hvor \texttt{.tex}
filen ligger, sætte shell variablen \texttt{TEXINPUTS} til at indeholde mappen hvori
\texttt{overtex.sty} ligger eller kopiere filen ind i
\texttt{\~{}/texmf/tex/latex/overtex/overtex.sty} hvor den vil være synlig for oversætteren.

\section{- Problemer/løsninger}
\subsection{Der er lag/forsinkelser}
Lag er irriterende, det opleves primært som en forsinkelse fra man har trykket
på knappen til der sker noget på projektoren. Det er et problem når man skal
lave overtekster.

Til DIKUs jubilæumsrevy var der ca. et sekunds forsinkelte fra jeg trykkede til
at overteksterne blev vist. Jeg har ikke definitivt fundet fejlen endnu men der
bliver arbejdet på det.

Her er først nogle metoder til at lokalisere hvor omtrentligt forsinkelsen
opstår.
Det er mere eller mindre umuligt at lave konkrete målinger så det er noget man
må føle sig frem til (Super naturvidenskabelig metode!).\\
\textit{Følgende tager udgangspunkt i Xpdf, men gøres på samme måde med Zeigen}

Prøv først at sætte systemet til at køre alt lokalt. Aka, kør både Emacs og Xpdf
lokalt og der kommunikeres via ssh til \texttt{localhost}. Hvis forsinkelsen
stadig er der, er fejlen enten i Emacs, i Xpdf, eller i den lokale
hardware.

Mine erfaringer med Xpdf er dog at det ikke er her fejlen ligger.
Prøv nu at starte Xpdf på serveren og ssh ind på denne, giv nu manuelt Xpdf
ordre til at skifte slide.

Min erfaring er at det er meget tilfældigt hvad der præcist sker. Jeg oplevede
at delayed forsvandt efter jeg gjorde ovenstående, også det mellem Emacs og
Xpdf, men kun indtil Xpdf blev lukket.

Det virker også til at Xpdf bliver langsommer afhængigt af længden af
overteksterne og ikke nødvendigvis størrelsen af pdfen.

Jeg ved ikke helt hvordan dette skal løses.
Det er en af grundene til at vi vil lave vores egen fremviser (Zeigen).


\section{- Eksempler}
\note{Med screenshots og eksempler, både på brug af program og effekter
  Se det som hvad jeg ville vise til et AV-føl
  Videosekvens på skrift/billede
}

\appendix
\section{En guide til Linux}
\label{linux_guide}
% Sådan bruger du en terminal
\textbf{cd}
\textbf{ls}
\textbf{cp}
\textbf{mv}

\textbf{ip}
\textbf{ssh}
\textbf{nano}

hvad er /dev/sdX og /dev/sdX1,2,3 osv?
hvordan fungerer IP'er?

xmonad
nævn
\begin{verbatim}
xmonad --recompile
xmonad --restart
\end{verbatim}
\newpage
\section{Installation af Arch Linux}
Dette appendix gennemgår installationen af Arch Linux fra bunden.
Arch Linux er en såkaldt ``Rolling release'' distribution af linux, den er ikke
specielt begynder venlig, men gør hvad man beder den om, hverken mere eller
mindre.

Dette afsnit er ikke et du skal bruge normalt som AV-mand, men dokumenterer
processen om at sætte systemet op fra bunden af, normalt vil der forhåbentligt
være nogle allerede fungerende installationer du kan låne.

Guiden er baseret på
\url{https://wiki.archlinux.org/index.php/Installation_guide} og\\
\url{https://wiki.archlinux.org/index.php/Beginners'_guide}, og antager at du
har styr på linux, se evt. Appendix \ref{linux_guide}.
\note{\url{http://www.muktware.io/arch-linux-guide-the-always-up-to-date-arch-linux-tutorial/} \url{http://www.dedoimedo.com/computers/grub-2.html}}
Husk at der er ``auto completion'' på tab-knappen.

\subsection{Live USB}
Først downloades \texttt{.torrent} filen fra
\url{https://www.archlinux.org/download/} og åbnes med dit yndlings torrent
program, f.eks. \texttt{rtorrent}.

Når \texttt{.iso} filen er hentet kan denne brændes til et usbstick.
\url{https://wiki.archlinux.org/index.php/USB_flash_installation_media}

For at lave en live USB fra et allerede eksisterende linux system sættes USB'en
i, uden at mounte den (eller unmount den).

Følgende kommando sletter alt på usbsticket og laver et live USB
\begin{verbatim}
sudo dd bs=4M if=/path/to/archlinux.iso of=/dev/sdX status=progress && sync
\end{verbatim}

Hvor \texttt{/path/to/archlinux.iso} er stien til \texttt{.iso} filen.
\texttt{/dev/sd}\textbf{X} er stedet hvor USB'en findes, f.eks. /dev/sdb, bemærk
at det er uden partitionen, så IKKE \texttt{/dev/sdb1}.

Sæt nu USBen i maskinen der skal installeres, og start op fra USB'en.

\begin{mdframed}[style=boxy]
For at gendanne Live USBsticket til et ``normalt'' USB stick
\begin{verbatim}
dd count=1 bs=512 if/dev/zero of=/dev/sdX && sync
cfdisk /dev/sdX
\end{verbatim}

Hvis filsystemet Ext4 ønskes:
\begin{verbatim}
mkfs.ext4 /dev/sdX1
e2label /dev/sdX1 USB_STICK
\end{verbatim}
Eller for et klassisk Windows Fat32 system
\begin{verbatim}
mkfs.vfat -F32 /dev/sdX1
dosfslabel /dev/sdX1 USB_STICK
\end{verbatim}

For at lave et Fat32 system kræver det at \texttt{dosfstools} er installeret.
cfdisk bruges til at lave en partition på USB'en, som findes bå
\texttt{/dev/sdX1}, \texttt{USB\_STICK} er navnet der ønskes på USB'sticket.
\end{mdframed}

\subsection{Installation af styresystem}

Hvis man ønsker et andet tastatur layout det gøres med f.eks.
\code{loadkeys colemak} hvis man ønsker at slå bip-lyden fra kan man gøre det
med \code{rmmod pcspkr}

\begin{verbatim}
timedatectl status
\end{verbatim}
Vær sikker på at tiden er indstillet korrekt, ovenstående skal sige
at \textbf{Universal time} er korrekt, din lokale tidszone kan indstilles senere.
Hvis tiden ikke passer ændres den i BIOS'en inden boot.
\note{Brug ntp? \code{timedatectl set-ntp true}}


\subsubsection{Opret forbindelse til internettet}
Hvis det er et helt almindeligt trådet netværk virker det muligvis uden
problemmer
Hvis maskinen er på et seperat netværk uden internet, men med en gateway som
brugt til den normalle revyopsætning gøres følgende:
\begin{verbatim}
ip link show
ip link set enpXsX up
ip addr add 192.168.0.XXX/24 dev enpXsX
ip route add default via 192.168.0.YYY
echo nameserver 8.8.8.8 > /etc/resolv.conf
\end{verbatim}
\texttt{XXX} er ip'en til maskinen, \texttt{YYY} er ip'en til gateway'en
\note{Henvis til gateway}

Hvis det er et almindeligt trådløst netværk benyttes \code{wifi-menu} eller
\texttt{wpa\_supplicant} for mere avancerede opsætninger

Test om der er internet:
\begin{verbatim}
ping google.com
\end{verbatim}
\textit{tryk Ctrl-c for at stoppe.}

\newpage
\subsubsection{Opret partitioner}
Hele situationen omkring UEFI vs. BIOS, GPT vs. MBR, kombineret med bootloaderen
(grub) og hardwareunderstøttelse er noget gøjl.
Det er et virvar af forskellige systemer der ikke altid har lyst til at
samarbejde, ikke fortæller hvorfor det ikke virker, og mangler dokumentation om
hvordan man får det til at virke.
I en periode, da det hele var nyt, og denne guide blev skrevet i første omgang,
var dokumentationen nærmest ikke eksisterende og meget af det her er opdaget ved
manuel afprøvning.

De næste par afsnit handler om forskellige måder jeg har prøvet at sætte
systemer op på.
Oprindeligt brugte jeg GPT-BIOS, da det var hvad jeg endelig fik til at virke,
men prøv dig frem.

\subsubsection{Opret partitioner (MBR-BIOS)}
\begin{verbatim}
fdisk /dev/sda
\end{verbatim}
med \code{parted -l /dev/sda} kan man se om det er GPT eller ej (også kaldet msdos)
tryk \texttt{o} for at skifte type til MBR.
Lav en swap partition og root + evt. home partition
sæt resten op som et GPT-BIOS system
\subsubsection{Opret partitioner (GPT-BIOS)}
Følgende beskriver hvordan jeg plejer/plejede at oprette partitioner, se det
alternative afsnit \ref{gpt_uefi} hvis et UEFI system ønskes.
Jeg tivler på at dette er den korrekte måde at gøre det på, men det virker på
ældre maskiner.

\code{lsblk} viser alle partitioner på alle diske, alternativt brug \code{fdisk -l}

\note{Følgende sletter alt indhold på harddisken og alle eksisterende partitioner}

\begin{mdframed}[style=boxy]
  Man kan slette alt indholdet på en disk med
\begin{verbatim}
sgdisk --zap-all /dev/sdX
\end{verbatim}
\end{mdframed}

\begin{verbatim}
cgdisk /dev/sdX
\end{verbatim}

Formententligt \texttt{sd\textbf{a}}

Tryk enter. (til alt ``brok'')

\texttt{Delete} alle partitoner.

\textbf{Lav plads til GPT partition}
\textit{Dette giver plads til en partions tabel til grub}\
\begin{enumerate}
\item \texttt{New}
\item Tryk enter, first sector skal bare  være default.
\item \texttt{1M} enter \textit{ - tidligere 1007KiB}
\item \texttt{ef02} enter
\item Tryk enter, der behøver ikke at være et navn
\end{enumerate}

\textbf{Lav swap}
\textit{Dette er næppe nødvendigt, men jeg laver den af gammel vane. Alternativt
kunne man lave en swap fil, men en partion er simplest}
\begin{enumerate}
\item Tryk ned. (til det store område med free space.)
\item \texttt{New}
\item Tryk enter, default er fint, formententligt 2048.
\item \texttt{3G} mindre swap kan også vælges (eller udelades).
\item \texttt{8200} swap Hex koden.
\item tryk enter.
\end{enumerate}

\textbf{Lav en partition}
Hvis flere seperate partitioner ønskes, f.eks. opdelt \texttt{/} og
\texttt{/home} oprettes de her.
\begin{enumerate}
\item Tryk ned.
\item \texttt{New}
\item enter
\item enter
\item enter
\item enter
\end{enumerate}

Vælg \texttt{Write}, skriv \texttt{yes}, og afslut.

Nu kan der stå at den gamle partitionstabel stadig er i brug. I så fald genstart og udfør alle trinene inden \texttt{cgdisk} igen.

Når du er klar skal vi så lave nogle filsystemer.
For at få overblik over partitionerne:
\begin{verbatim}
lsblk
\end{verbatim}

\begin{verbatim}
mkfs.ext4 /dev/sda3
mkswap /dev/sda2
swapon /dev/sda2
\end{verbatim}
(Hvis flere partioner blev oprettet til home og root, så gentag øverste linje
for disse partioner)

Ignorer \texttt{sda1} indtil videre.

\begin{verbatim}
mount /dev/sda3 /mnt
\end{verbatim}

Hvis du skulle have lavet en seperat partition til home så lav en mappe
\code{mkdir /mnt/home} og mount home partitionen der \code{mount /dev/sdaX
  /mnt/home}.

\newpage
\subsubsection{(GPT-UEFI)}
\label{gpt_uefi}
Dette er en lidt simplere gennemgang af at sætte et GPT-UEFI system op.
Dette er den nye måde at gøre tingene på som understøttes af moderne hardware.

\code{lsblk} viser alle partitioner på alle diske, alternativt brug \code{fdisk -l}

\note{Følgende sletter alt indhold på harddisken og alle eksisterende partitioner}

\begin{mdframed}[style=boxy]
  Man kan slette alt indholdet på en disk med
\begin{verbatim}
sgdisk --zap-all /dev/sdX
\end{verbatim}
\end{mdframed}

\begin{verbatim}
cgdisk /dev/sdX
\end{verbatim}

Formententligt \texttt{sd\textbf{a}}

Tryk enter. (til alt ``brok'')

\texttt{Delete} alle partitoner.

\textbf{Lav plads til GPT partition}
\textit{Dette giver plads til en EFI partitionstabel}\
\begin{enumerate}
\item \texttt{New}
\item Tryk enter, first sector skal bare  være default.
\item \texttt{512M} enter
\item \texttt{ef00} enter
\item Tryk enter, der behøver ikke at være et navn
\end{enumerate}

\textbf{Lav swap}
\textit{Dette er næppe nødvendigt, men jeg laver den af gammel vane. Alternativt
  kunne man lave en swap fil, men en partion er simplest}
\begin{enumerate}
\item Tryk ned. (til det store område med free space.)
\item \texttt{New}
\item Tryk enter, default er fint, formententligt 2048.
\item \texttt{3G} mindre swap kan også vælges (eller udelades).
\item \texttt{8200} swap Hex koden.
\item tryk enter.
\end{enumerate}

\textbf{Lav en partition}
Hvis flere seperate partitioner ønskes, f.eks. opdelt \texttt{/} og
\texttt{/home} oprettes de her.
\begin{enumerate}
\item Tryk ned.
\item \texttt{New}
\item enter
\item enter
\item enter
\item enter
\end{enumerate}

Vælg \texttt{Write}, skriv \texttt{yes}, og afslut.

Nu kan der stå at den gamle partitionstabel stadig er i brug. I så fald genstart og udfør alle trinene inden \texttt{cgdisk} igen.

Når du er klar skal vi så lave nogle filsystemer.
For at få overblik over partitionerne:
\begin{verbatim}
lsblk
\end{verbatim}

\begin{verbatim}
mkfs.fat -F32 /dev/sda1
mkfs.ext4 /dev/sda3
mkswap /dev/sda2
swapon /dev/sda2
\end{verbatim}
(Hvis flere partioner blev oprettet til home og root, så gentag øverste linje
for disse partioner)

\begin{verbatim}
mount /dev/sda3 /mnt
mkdir /mnt/boot
mount /dev/sda1 /mnt/boot
\end{verbatim}

Hvis du skulle have lavet en seperat partition til home så lav en mappe
\code{mkdir /mnt/home} og mount home partitionen der \code{mount /dev/sdaX
  /mnt/home}.

Du kan bruge \code{lsblk} til at få overblik over partitionerne, og hvor
de er mounted.
\newpage
\subsubsection{Installation}
Sørg for at være på nettet da vi nu skal hente pakker ned til styresystemet.
\begin{verbatim}
pacstrap -i /mnt base base-devel
\end{verbatim}
Tryk enter til spørgsmål.

\begin{verbatim}
genfstab -U -p /mnt >> /mnt/etc/fstab
arch-chroot /mnt /bin/bash
nano /etc/locale.gen
\end{verbatim}

Fjern kommenteringen til linjerne \textit{\#da\_DK.UTF-8 UTF-8} og\textit{\#en\_US.UTF-8 UTF-8}
\note{Henvis til brug af nano}

\begin{verbatim}
locale-gen
echo LANG=en_US.UTF-8 > /etc/locale.conf
export LANG=en_US.UTF-8
ln -s /usr/share/zoneinfo/Europe/Copenhagen /etc/localtime
hwclock --systohc --utc
echo XXX > /etc/hostname
\end{verbatim}
hvor XXX er navnet til maskinen

\note{Hvis du hellere vil arbejde med at netværksinterfacene hedder
  \texttt{eth0} osv. istedet for enpXsX:
  \code{touch /etc/udev/rules.d/80-net-setup-link.rules}}

\note{?}
\begin{mdframed}[style=note]\textbf{kopieret fra noter:}
Brug netctl til automatisk at gå på netværk
(Ret i hovrdan man bruger deet i resten af dokumentationen)
\begin{verbatim}
curl "https://raw/githubusercontent.com/Pilen/ubertex/master/linux/revynet" >
 /etc/netctl/revynet"
nano /etc/netctl/revynet
\end{verbatim}
Og så skal det gerne matche følgende:
\begin{mdframed}[style=code]
  \verbatiminput{linux/revynet}
\end{mdframed}
Hvor XXX erstates med den lokale IP, og YYY med gateway'ens.

\begin{verbatim}
netctl enable revynet
\end{verbatim}
Hvis man ændrer i configurationen, f.eks. IP'en:
\begin{verbatim}
netctl reanable revynet
netctl restart revynet
\end{verbatim}
\end{mdframed}

Hvis maskinen skal kunne gå på trådløst netværk efterfølgende;
\begin{verbatim}
pacman -S iw wpa_supplicant dialog
\end{verbatim}

\begin{verbatim}
mkinitcpio -p linux
\end{verbatim}
\begin{verbatim}
passwd
\end{verbatim}
Skriv løsn til root (f.eks. \texttt{hamsterroot}).

\note{Du kan med fordel oprette din bruger her}
\note{Du kan med fordel sætte sudo op her}
\note{Du kan med fordel installere sshd her}

For et GPT-BIOS system:
\begin{verbatim}
pacman -S grub
grub-install --target=i386-pc --recheck --debug /dev/sda
grub-mkconfig -o /boot/grub/grub.cfg
\end{verbatim}
For et GPT-UEFI system:
\begin{verbatim}
mkdir /boot/efi
pacman -S grub efibootmgr
grub-install --target=x86_64-efi --efi-directory=esp --bootloader-id=grub
grub-mkconfig -o /boot/grub/grub.cfg
\end{verbatim}
\note{
Den vil klage og sige: \texttt{efibootmgr: EFI variables are not supported on
  this system}, ignorer dette?}

Vi er nu klar til at genstarte op i det nyinstallerede system:
\begin{verbatim}
exit
umount -R /mnt
shutdown -h now
\end{verbatim}
\note{?}
\begin{mdframed}[style=note]\textbf{Note:}
  Hvis grub-mkconfig fejler:
\begin{verbatim}
nano /etc/default/grub
...
...
#fix broken grub.cfg gen
GRUB_DISABLE_SUBMENU = y
\end{verbatim}
\end{mdframed}

Hiv usbstikket ud.

Tænd datamaten igen.
Hvis du som mig til jubilæumsrevyen installerede systemet på en harddisk i en
anden datamat en dens egen, kan det være at ramdisken fejler, i grub vælges da
fallback løsningen og du kalder \code{mkinitcpio -p linux} for at skabe et nyt
korrekt image.

\subsection{Opsætning}
\subsubsection{Bruger}
\note{Hvis en bruger og sudo er sat op kan man logge ind som dem, ellers så log
  ind som root}

Log ind som root

\note{Kom på nettet igen}

\begin{verbatim}
ip link show
\end{verbatim}
For at se netværks interfacet

\begin{verbatim}
useradd -m -s /bin/bash revy
\end{verbatim}
\begin{verbatim}
passwd revy
\end{verbatim}
Giv revy et løsn.

Sudo er allerede installeret via base-devel, du kan læse mere om sudo i Appendix \ref{linux_guide}.
\begin{verbatim}
visudo
\end{verbatim}
Tryk pil ned til du finder linjen
\begin{verbatim}
root ALL=(ALL) ALL
\end{verbatim}
placer cursoren under denne og tryk \code{i}
tast \texttt{revy ALL=(ALL) ALL} tryk enter, tryk escape
tryk \text{:wq} enter.

revy kan nu sudo'e.

\begin{verbatim}
pacman -S openssh
\end{verbatim}
\note{Det er nu default at \texttt{PermitRootLogin} er sat til \texttt{no}, så
  det er ikke længere nødvendigt at ændre det i \texttt{/etc/ssh/sshd\_config}}
\begin{verbatim}
systemctl enable sshd.service
systemctl start sshd
\end{verbatim}
Nu kan vi ssh'e ind fra vores lokale maskine \code{ssh revy@192.168.0.XXX}.
\textit{Læs evt. afsnittet om ssh-nøgler.}

Nu kan vi vælge at køre kommandoerne via en ssh forbindelse.
Du kan også genstarte og logge ind som almindelig bruger istedet, flere af
kommandoerne kræver da sudo.

\subsubsection{Grafisk system}
Nu skal vi installere et grafisk interface
\begin{verbatim}
pacman -S xorg-server xorg-server-utils xorg-xinit
\end{verbatim}
Den spørger nu hvilken udbyder af \texttt{libgl} der ønskes, default (1
mesa-libgl) er formententligt et fint valg (medmindre andet ønskes). (tryk 1
efterfulgt af enter)

Den spørger så om hilken udbyder af \texttt{xf86-input-driver} der ønskes,
1: \texttt{evdev} eller 2: \texttt{libinput}. Libinput er et nyere wayland projekt der
også virker til xorg. Begge er tilsyneladende fine, jeg valgte evdev da det er
det ``klassiske'' valg.

Der skal formententligt installeres nogle grafik drivere, læs mere her \url{https://wiki.archlinux.org/index.php/Xorg#Driver_installation}
\texttt{fbdev} og \texttt{vesa} er nogle udemærkede open source fallback drivere hvis
ikke der findes et grafikkort (bemærk at det fører til software rendering).
Hvis det er et intel kort, er intel driveren et godt bud, for nvidia er nouveau
driverne et godt open source bud

\note{mesa og mesa-libgl er installeret af xorg-server}
\begin{verbatim}
pacman -S xf86-video-fbdev xf86-video-vesa xf86-video-intel
\end{verbatim}
Derudover er det en god idé med Hardware video acceleration
\begin{verbatim}
pacman -S libav-intel-driver libav-mesa-driver
\end{verbatim}

% Der kan kun køre én instans af pacman af gangen.
% Så sæt ham til at arbejde mens vi i en anden terminal begynder at konfigurere.
Som windowmanager bruger vi xmonad, en simpel ``tiling'' windowmanager med
minimale vinduesdekorationer, kan styres med tastaturet og fordi jeg kender den.
Se Appendix \ref{linux_guide} for mere om brugen.

\begin{verbatim}
pacman -S xmonad xmonad-cotrib
\end{verbatim}

Vi skal oprette en \texttt{.xinitrc}, den kan hentes sådan
\begin{verbatim}
curl "https://raw/githubusercontent.com/Pilen/ubertex/master/linux/.xinitrc" > .xinitrc
chown revy:revy .xinitrc
nano .xinitrc
\end{verbatim}
Indholdet skal da være:
\begin{mdframed}[style=code]
  \verbatiminput{linux/.xinitrc}
\end{mdframed}
Den usynlige cursor hentes med:
\begin{verbatim}
curl "https://raw/githubusercontent.com/Pilen/ubertex/master/linux/.emptycursor.xbm" > .emptycursor.xbm
\end{verbatim}

\begin{verbatim}
mkdir .xmonad
curl "https://raw/githubusercontent.com/Pilen/ubertex/master/linux/xmonad.hs" > .xmonad/xmonad.hs
chown -R revy:revy .xmonad
\end{verbatim}
\begin{mdframed}[style=code]
  \verbatiminput{linux/xmonad.hs}
\end{mdframed}
\note{Skal borderWidth være 0?}
\begin{verbatim}
xmonad --recompile
\end{verbatim}

Åben \texttt{.bashrc} og tilføj i bunden:
\begin{verbatim}
if [[ -z $DISPLAY && $(tty) = /dev/tty1 ]]; then
    exec startx
fi
\end{verbatim}
Du kan evt. også tilføje følgende alias over linjerne:
\begin{verbatim}
alias d='export DISPLAY=:0'
\end{verbatim}

For at systemet automatisk kan logge ind ved opstart kan man gøre følgende
\begin{verbatim}
curl "https://raw/githubusercontent.com/Pilen/ubertex/master/linux/autologin.conf" > /etc/systemd/system/getty@tty1.service.d/autologin.conf
nano /etc/systemd/system/getty@tty1.service.d/autologin.conf
\end{verbatim}
indhold af \texttt{autologin.conf}:
\begin{mdframed}[style=code]
  \verbatiminput{linux/autologin.conf}
\end{mdframed}
\note{I noterne står det som \code{curl
    "https://raw/githubusercontent.com/Pilen/ubertex/master/linux/autologin.conf"
    | sudo tee /etc/systemd/system/getty@tty1.service.d/autologin.conf}
  Outputet fra curl kan skjule forespørgslen efter løsen, så hvis den hænger er
  det nok derfor}
\subsubsection{Pakker}
\begin{verbatim}
pacman -S alsa-utils rxvt-unicode dmenu xdotool feh mplayer ttf-dejavu screen htop rsync mupdf
\end{verbatim}

\section{Stripped Emacs}
\section{Emacs}
\section{mangler}
revy-shell skal escape kommandoer
Youtube
Arduinoer
Stripped Emacs

\section{Zeigen-protokol}

ADVARSEL beskeder til sketches er lokale for den nuværende eksekverende sketch!!!.

Zegien forstår følgende protokol:

\begin{verbatim}
host [otherhost];<time>;command;options
\end{verbatim}

\texttt{Host} er en mellemrums adskilt liste af navne på de zeigen servere/grupper der
skal have beskeden (tage sig af den). Husk at navne er caseinsensitive.

\texttt{<time>} er klokkeslettet hvor kommandoen skal udføres.
Beskeder i datiden udføres hurtigst muligt, så brug 0 hvis den skal udføres med
det samme.

\texttt{command} er den kommando der ønskes udført
\begin{verbatim}
clearqueue
kill
quit
sketch
start
sync
\end{verbatim}

\texttt{options} er argumenter der gives til kommandoen.
formattet for \texttt{options} afgøres af kommandoen

\subsection{kill}
Lukker den nuværende sketch.

Man kan give ét argument med og kommandoen udføres kun hvis sketchens navn
matcher dette.

Sketchens navn er Ikke casesensitiv.

\subsection{Start}
Start lukker den nuværende sketch og starter en ny.

Man kan give ét argument med og kommandoen udføres kun hvis sketchens navn
matcher dette.

\subsection{Sketch}
\texttt{sketch} kommandoen regner med at option består af følgende
\begin{verbatim}
sketchname;message
\end{verbatim}

Hvis \texttt{sketchname} tilsvarer navnet på den nuværende afviklende sketch,
sendes \texttt{message} til sketchen på det før angivne tidspunkt.
Beskeden modtages i sketchens \textit{receive} metode.

\section{PDF/billedviser}
Kan baseres på poppler.
python-poppler-qt4?

mupdf er et mere minimalistisk/optimeret alternativ til xpdf's poppler.
har mudraw der kan rendere til png.

% \section{Elisp kode}
% \includepdf[pages=-]{elisp-source.pdf}
% \section{C kode}
% \includepdf[pages=-]{c-source.pdf}

\end{document}

\documentclass[10pt,a4paper,danish]{article}
\usepackage[danish]{babel}
\usepackage[utf8]{inputenc}
% \usepackage[margin=3cm]{geometry}
\usepackage[margin=4cm]{geometry}
\usepackage{amsmath}
\usepackage{amssymb}
\usepackage{listings}
\usepackage{fancyhdr}
\usepackage[hidelinks]{hyperref}
\usepackage{booktabs}
\usepackage{graphicx}
\usepackage{xfrac}
\usepackage[dot, autosize, outputdir="auto/"]{dot2texi}
\usepackage{tikz}
\usetikzlibrary{shapes}
\usepackage{ulem}
\usepackage{subcaption}
\usepackage{mdframed}
\usepackage{color}
\usepackage{marginnote}
\usepackage{pdfpages}
\usepackage{shorttoc}
\usepackage{setspace}
\usepackage{color}
\usepackage{pdflscape}
\usepackage{verbatim}

\setlength{\marginparwidth}{1.0in}
\setlength{\marginparsep}{0.1in}
\let\oldmarginnote\marginnote
\renewcommand{\marginnote}[1]{\oldmarginnote{\raggedright{}\footnotesize #1}}

\definecolor{boxygray}{gray}{0.98}
\mdfdefinestyle{boxy}{%
  backgroundcolor = boxygray}

\definecolor{notered}{RGB}{255, 180, 180}
\mdfdefinestyle{note}{%
  hidealllines = true,
  backgroundcolor = notered}
\newcommand{\note}[1]{\begin{mdframed}[style=note]\textbf{Note:}
    #1\end{mdframed}}

\newcommand{\bad}[1]{\begin{mdframed}[style=note]#1\end{mdframed}}

\definecolor{verbgray}{gray}{0.92}
\mdfdefinestyle{code}{%
  hidealllines = true,
  backgroundcolor = verbgray}

\let\oldv\verbatim
\let\oldendv\endverbatim

\def\verbatim{\mdframed[style=code]\oldv}
\def\endverbatim{\oldendv\endmdframed}

% \newcommand{\code}[1]{\mdframed[style=code]\texttt{#1}\endmdframed}
% \newcommand{\code}[1]{\texttt{#1}}
\newcommand{\code}[1]{\colorbox{verbgray}{\texttt{#1}}}

\newcommand{\commnt}[1]{}

\newcommand{\documentcommand}[1]{{\colorbox{lime}{\large \texttt{(#1)}}}\\}
\newcommand{\documentargument}[1]{``\texttt{#1}''}
\newcommand{\optional}[1]{\&\textit{optional}}
\newcommand{\rest}[1]{\&\textit{rest}}
\newcommand{\lquote}[1]{\textquotesingle{}#1}
\newcommand{\lstring}[1]{\"{}#1\"{}}
\newcommand{\documentkey}[2]{\code{#1}\quad\quad--- \textit{#2}\\}

\usepackage{relsize}
\def\SaTyR{S\kern-.0967em%
  \lower.2ex\hbox{\smaller[2]A}%
  \kern-.155em%
  {\smaller[1]T}%
  \kern-.1667em%
  \lower.2ex\hbox{\smaller[2]Y}%
  \kern-.055em%
  R\spacefactor1000 }


\pagestyle{fancy}
\fancyhead{}
\fancyfoot{}
\lhead{\nouppercase\leftmark}
\rhead{Den Store AV-Bog}
\rfoot{\thepage}
% \setlength\parskip{1em}
\setlength\parskip{0.8em}
\setlength\parindent{0em}

%% Titel og forfatter
% \title{{\Huge{}\fontfamily{pzc}\selectfont{}Den Store AV-bog}}
\title{\Huge{}Den Store AV-Bog}
\author{Søren Pilgård - DIKUrevyen}

%% Start dokumentet
\begin{document}

%% Vis titel
\maketitle
\newpage
{\linespread{0.0}
  \shorttoc{Indhold (kort)}{1}}
\newpage

%% Vis indholdsfortegnelse
\tableofcontents
\newpage

\note{TODO:
  Skal det omtales som Master, Controller eller kontrolmaskine?
  Skal det kaldes en kommando eller en instruktion?
}

\section{Introduktion}
\begin{center}
  {\Huge \textit{``Er der en AV-mand tilstede?!?''}}
\end{center}
Sådan kunne spørgsmålet lyde. Når du har gjort dig bekendt med denne bog er du
forhåbentligt istand til at svare \textit{``Ja!''}.
Denne bog henvender sig til dig der aldrig har lavet AV i en revy før og skal
have en udførlig introduktion, til dig der har lavet AV før men lige skal have
lidt hjælp til hvordan det nu var med det hersens AV-system, og til dig der har
lavet AV i mange år men stadig har behov for at slå op i dokumentationen i ny og
næ.

Dette er en introduktion til DIKUrevyens \textbf{AV-system}.
Systemet er udviklet af Søren Pilgård, og selv om det startede hos DIKUrevyen
har det allerede haft bred udbredelse blandt andet i DIKUrevyen,
SaTyRrevyen, Biorevyen og Matematikrevyen.
Systemet er af den slags der nok aldrig bliver helt færdigt, der er altid lige
noget mere der kan tilføjes, laves om og forbedres, det samme gælder denne bog.
Det bør dog ikke afskrække dig fra at kaste dig ud i det.

\marginnote{At vi kalder det AV-\textit{mand} skal du ikke tage så nøje, der findes super
  seje og kreative AV-folk af alle køn!}
Dette kapitel samt kapitel \ref{sec:god-av-mand} forklarer lidt generelt om hvad
det vil sige at lave revy og hvad AV-manden laver, samt nogle guidelines til
hvordan man gør det godt som AV-mand.


\subsection{Hvordan foregår en revy?}
Dette er måske din første revy, eller måske bare din første revy på Natfak.
Her vil vi for god ordens skyld kort gennemgå hvad en revy i denne sammenhæng
indebærer. De fleste revyer har en nogenlunde ens opbygning dette gælder især
for revyerne i SaTyRfællesskabet som indefatter DIKURevyen, Fysikrevyen,
Matematikrevyen, Biorevyen, MBKrevyen, og så selvfølgeligt SaTyRrevyen selv.
En revy er overordnet delt op i forskellige numre, f.eks. sange eller sketches,
disse numrer er, som regel, men bestemt ikke altid uafhængige af hinanden uden
overlap i hverken karakterer, tema eller indhold. Der sker dog ofte undtagelser
iform af f.eks. følgetoner med fortløbene handling, eller f.eks. sketches der
naturligt glider over i en sang. Numrene er grupperet i akter af en halv time
til tre kvarters varighed med en 15-20 minutters pause.
Der er typisk 2 eller 3 akter efterfulgt af et par ekstranumre.

En klassik revy foregår således at klokken 19.00 åbner dørene til foyeren,
typisk er de allerede åbne, men det er her folk begynder at stille sig i kø uden
foran dørene til auditoriet. Kl. 19.15 åbner dørene ind til auditoriet og
publikum strømmer ind. Når folk har fundet deres pladser sker det ofte at de
forskellige studieretninger begynder at synge af hinanden. Dette er en god
gammel tradition hvor nye svarvers dukker op og stemningen bliver sat. Det er
dårlig stil når revyen går i gang mens folk synger, så man skal sørge for at
time det før eller efter en sang. Når klokken er 19.30 og backstage er klar
signalerer en boss eller instruktør til \TeX{}nikken at nu kan vi starte.
Når der ikke synges sætter lysmanden så revyen i gang ved at dæmpe lyset i
salen.
Et øredøvende sus vil da gå igennem auditoriet som hele det feststemte publikum
fylder med råb, banken og hujen.

Nu er revyen i gang!\\


De fleste revyer starter med en bandintro, en video der præsenterer bandet, når
videoen er færdig sætter bandet sig til rette på bandscenen mens folk brøler
\textit{``Bandet!, Bandet!...''} nogle revyer har så en velkomst sketch hvor man
præsenterer revyen på en sjov måde (fysik starter f.eks. altid med at aflyse
revyen), andre går bare direkte over i en startsang der skal få gang i salen.

\marginnote{Nogle revyer foretrækker at holde 20 minutters pause, men skrive 15}[0cm]
Herefter følger en masse blandede sange og sketches i form af akt 1.
Man slutter typisk en akt med en sang med gang i så folk går feststemte til
pause.
Det er lysmandens opgave at åbne op for lyset, men AV manden starter typisk med
at have et pauseskilt/billede/video til at fortælle at nu er der pause først.
Herefter er der pause i 15-20 minutter mens publikum køber øl og går på
toilettet.
(Som AV-mand kan det også være en god ide at benytte toiletterne nede backstage
og fylde op på vand/drikkevarer nu).
Lidt før pausen er klar begynder bandet at spille ind, de spiller typisk noget
musik der passer til deres tema eller som de har brugt mens de øvede sig.

Så starter næste akt (hver opmærksom på at der altid er et par stykker der
kommer forsent og som konsekvent glemmer at lukke dørene (så hav en løber
klar)). I en 3 akters forestilling kører andet akt stille og roligt (sådan
konstruktionsmæsigt), i en 2 akters svarer det til at gå direkte til 3.
Hver opmærksom på at nogle 2 akters forestillinger kan have en tendens til at
have lidt længere akter. Sidste akt slutter af med et slutnummer, en god sang
der får folk op og køre. 3/4 inde i sangen kommer alle skuespillerne ud og der
siges tak til alle der skal have tak og alle på scenen bukker et par gange.
Så lunter folk ud af scenen og lyset går ned så der bliver helt mørkt. Imens
klapper folk ihærdigt og råber \textit{``Ekstra nummer!, Ekstra nummer!...''}
for det siger traditionen. Det er ofte enten AV eller lys der starter ekstra
numrene man skal altid trække den lidt, så trække den lidt mere, og så liiiige
trække den lidt mer', både fordi sådan er det, samt for at de kan nå at skifte
kostumer/gøre rekvisiter klar. Og så går ekstranumrene i gang.
Disse består som regel af 3-4 numre, nogle gange en video og slutter af med en
slut-slutsang. Denne sidste sang er handler ofte om at nu er revyen slut, men
der er efterfest bagefter og så skal den have fuld gas.
Så er revyen færdig, man kan smide et skilt op om at der er efterfest. Hvis man
har flere forestillinger men kun én efterfest kan man smide en henvisning om at
gå på \textit{Caféen?!} op.


Generelt prøver man ofte at holde det intelligente humor i første akt hvor folk
stadig kan forstå det og ikke ville ``kede'' sig, mens 3 akt er mere plat humor,
dårlige ordspil og billige punchlines. I takt med at publikum bliver mere og
mere fulde passer dette også meget fint til hvad de magter.
Publikum kan være helt fantastiske, men de kan også være larmende, forstyrende
og ret overvældende. Dette gælder både for dem på scenen, men det kan også være
skræmmende når \TeX{}nikken laver en fejl og hele auditoriet brøler
\textit{``TeXnikken sejler''}. Det vigtigste er at holde hovedet koldt i en hver
situation. Få ro på, tænk igennem hvad der går galt, lad være med at opildne
publikum, fokuser, og få tingene tilbage på sporet.
Vi har \textbf{ALLE} været derude hvor det hele gik galt, og bagefter gik det
hele altsammen, bare husk på ovenstående.

Hvis du er i tvivl om et begræb så tag et kig i kapitel \ref{subsec:hvad-er-hvad}.

\subsection{Hvad laver en AV-mand?}

Hvad er det så AV-manden laver til en revy?
Jo, ser du, TeXnikken består af 3 dele, der er lyd, de sørger for at bandets
musik, sangernes sang og skuespillernes grynten\commnt{Too harsh?} kommer ud af højtalerne.
Lydmanden laver så at sige ikke noget lyd selv, han sørger bare for at alle kan
høre det og at det lyder godt. Lysmanden sørger for at man kan se noget, det gør
han med et væld af farvede lamper og spots, med lys og skygge samt en gang
taktisk røg kan han transformere stemningen og skabe en fest på scenen.

\marginnote{Nogle revyer er også begyndt at have en Epelipsiadvarsel,
  i starten af revyen og før blinkende numre}
AV-mandens rolle er mere alsidig, en slags digital rekvisit man kan bruge til
lidt af hvert. Det kunne f.eks. være at agere powerpoint-show til en
``fremvisning'', vise billeder til at illustrere forskellige pointer, vise
videoer, vise pauseskilte, vise hvilke sponsorer der er, afspille digital musik
til dansenumre, afspille lydeffekter som pistolskud og prutter. En af de
vigtigste opgaver er dog at lave overtekster. Det kan være utroligt svært at
lave god lyd i Store UP1, hvis sangerne samtidigt er lidt usikre og der er en
masse larm i auditoriet kan det være utroligt svært at høre hvad der bliver
sunget. Samtidigt er der mange sange hvor man gerne vil give publikum mulighed
for at synge med i omkvædende, derfor har man valgt at indføre overtekster.
Overtekster er ligesom undertekster på en film, de sungne tekster bliver bare
vist på et hvidt lærred oppe bag scenen for at alle kan se dem.
Dette har vist sig at være en stor success.

\marginnote{\textbf{Pas på dig selv!} Det er nemt at brænde sig på et stort
  projekt.}
Hvis man er kreativ og har teknisk kunnen kan man også bevæge sig ud i mere
avancerede ting, animationer af alle mulige slags som passer til hvad der sker
på scenen. Revy-systemet understøtter en masse seje ting, og der kommer hele
tiden flere til så hvis man har mod på lidt programmering er det bare at give
sig i kast med effekterne. Men pas på, lad være med at bide over mere end hvad
du kan sluge. Alle disse avancerede detaljer er lækkerier der kan oppe
oplevelsen, men de kan aldrig stå alene og hvis du ser det som en for stor
opgave skal du ikke være bange for at spørge om hjælp eller sige fra.
\marginnote{Men når det er sagt: Man kan kun skabe noget nyt ved at prøve nye ting af.}[-1cm]



\subsection{Hvad er hvad?}
\label{subsec:hvad-er-hvad}
Her følger en række gængse termer der kan være praktiske at bruge, og andre
småting der er værd at huske.

\begin{description}
  % Revyen generelt
\item[Scene:] Til hver revy sættes en stor hjemmebygget træscene op i bunden af
  Store UP1.
\item[Bandscene:] En lille forhøjet scene til venstre for scenen, hvor bandet
  sidder og spiller.
\item[Bagtæppet:] Et tæppe til at aflukke til den bagerste meter af scenen, så
  man kan stille sig klar uset.
\item[Højre:] Højre side af scenen set fra TeXnikken og publikum.
  Den side hvor folk oftest kommer ind fra.
  (Skuespillere kan ikke finde ud af højre og venstre)
\item[Venstre:] Venstre  side af scenen set fra TeXnikken og publikum.
  Den side hvor bandet sidder
  (Skuespillere kan ikke finde ud af højre og venstre)
\item[Trekanten:] Området til højre på scenen bag tæpperne. Hvor skuespillerne
  kan stå klar og let komme ind fra siden.
\item[TeXnikken:] Lyd, Lys og AV, samt diverse hjælpere.
  Kortsagt dem der sidder foran scenen og arbjeder under revyen
\item[Backstage:] Området bag scenen. Bruges også om dem der arbjeder der. De
  andre ``skjulte'' roller som dem der laver rekvisiter og kostumer.
\item[Rekvisitten:] Dem der laver rekvisiterne/stedet de laver dem (backstage)
\item[Rekvisitkælderen:] Revyernes fælles opbevaringslokale, backstage under bandscenen.
  (Pas på skimmelsvamp)
\item[Dyrestalden:] En del af DIKU, gangen der ligger lige når man kommer ud af Store UP1 fra
  scenen.
\item[Omklædningen:] Backstage bag de store metaldøre og indtil trædørene står skuespillernes ting
  under revyen, gør det svært at komme igennem!
\item[Hyggeområdet:] Området bag skuespillernes omklædning, for foden af den sydlige trappe.
  Her findes ofte snacks og drikkevarer under revyen (loot det i pauserne)
\item[Harlem:] Et ``hemmeligt'' lokale i DIKUs kælder som revyerne råder over. Her øver
  bandet indtil de rykker ind i auditoriet. Her står også det meste af den
  texnik som revyerne råder over samt kasser med kabler.
  Sørg for at spørge før du låner størrere ting, husk at ryde op efter dig, husk
  at tape kabler sammen inden de dumpes. Spørg din revyboss om adgang hertil
\item[Indre Harlem:] Inde i Harlem findes indre Harlem der bruges til at redigere videoer samt de
  optagne film fra forestillingen.
  Her står det udstyr vi passer ekstra godt på.
  Kun de særligt betroede får adgang hertil og man skal spise en skovsnegl og
  kysse en datalog for at få lov!
\item[Adgang:] Husk at snakke med din boss om at få aktiveret dit studiekort så du kan komme
  ind på DIKU og i Harlem. I TeXnikken kommer og går man ofte på skæve
  tidspunkter så det er praktisk med adgang.
\item[Instruktør:] Dette er meget forskelligt fra revy til revy, men typisk en
  kreativ/kuntnerisk/skuespilsmæsig ansvarlig for de forskellige numre
\item[Boss:] Ansvarlig for hele revyen (bestemmer)
\item[Universitetsparken 1:] Stedet der i mange år har huset DIKU (i hvertfald
  de studerende). Skal forlades i år ....
\item[Store UP1:] Det store auditorium på Universitetsparken 1, også
  kendt som store knirke. Det er her alle revyerne på natfak afholder deres revyer.
\item[Caféen?!:] En af fredagsbarene på Natfak. Mange efterfester holdes her, og
  ofte sender man publikum til Caféen?! om fredagen efter første
  forestilling/generalprøven.
  % AV
\item[Overtex:] Et stort hvidt område over scenen under loftet hvor AV
  projekterer det meste af sine ting
\item[Højtex:] Et fancy begræb for alle de projektioner der er over Overtex på
  selve loftet
\item[Lavtex:] Et fancy begræb for alle de projektioner der er under Overtex,
  altså ned på selve scenen (antagelivis på et lærrede der opstilles undervejs)
\item[Fisk:] Et fancy ord for en kort film der ligger imellem to numre.
  Bruges ofte som et sjovt indslag nogen kom på samt til at give lidt ekstra tid
  til at nå et svært kostume/sceneskift.
\item[Datamat:] Fysisk implementation af Turings abstrakte maskine, hvad man på
  Engelsk ville kalde en ``computer''.
\item[\LaTeX:] Et ``sprog'' til at skrive og opsætte tekst, tænk HTML men til at
  lave nydelige artikler, formler og pdf'er. Også brugt til at lave overtekster.
\item[\TeX:] Forgængeren som \LaTeX er bygget på, de to termer bruges oftest
  synonymt.
\item[Emacs:] Gammelt tekstredigeringsprogram (Tænk notepad, men mere avanceret
  end Photoshop). Kan udvides helt utroligt og danner basis for DIKUrevyens AV-system.
\item[Script:] Et kort program, en sekvens af instruktioner for hvad der skal
  ske (automatisk) når det bliver kørt.
\item[Lisp] Et gammelt programmeringssprog. Der findes mange varianter men det
  er kernen i Emacs og udgør dermed måden man laver scripts på i AV-systemet
\end{description}


\subsection{Forudsætninger (Skal ændres)}
\note{TODO: Dårligt afsnit. Find ud af hvor det passer, og omforumler det til
  det her skal/vil du lære i bogen}

Hvad skal man kunne for at bruge DIKUrevyens AV-system?

\begin{itemize}
\item \LaTeX
\item Linux
\item Emacs
\item Emacs Lisp
\item Python
\end{itemize}

\textbf{\LaTeX}\\
Bruges til at skrive selve overteksterne. Der bruges Beamer til at lave et langt
slideshow, \code{overtex.sty} definerer en række makroer der gør det let at lave en
lang præsentation.
De fleste på natfak burde kunne \LaTeX, hvis ikke er den mængde der bruges til
at lave normale overtekster ret lille og burde kunne mestres ved bare at kigge i
nogle af de gamle filer.

\textbf{Emacs}\\
Emacs bliver brugt som kontrolcenter til det hele.
Det er derfor væsentligt at have en hvis forståelse for hvordan Emacs virker.
Emacs er et voldsomt konfigurerbart program til at arbejde med tekst.
Jeg er selv stor Emacsbruger og har opbygget et helt unikt system.
På sigt er det håbet at lave en standard konfiguration som folk der ikke er
emacsbrugere kan udnytte. En sådan konfiguration ville kunne skjule at det
overhovedet er emacs der kører bag det hele.

\textbf{Emacs Lisp}\\
Emacs Lisp er det primære scripting sprog der bliver brugt.
Det bruges til at automatisere ting i Emacs, derudover kan der kaldes externe
kommandoer og kommunikere med andre maskiner.
Selve revysystemet bruger en stor del Emacs Lisp så hvis noget stopper med at
virke er det godt at kende. Til almindeligt AV brug kan man dog nøjes med en
stærkt begrænset del som denne guide nok skal introdusere.

\textbf{Linux}\\
DIKUrevyens AV-system er bygget og kører på linux styresystemet.
Det betydder at man for at bruge systemet effektivt er nød til at have en hvis
forståelse for at bruge en linuxmaskine igennem en terminal.
Et håb er en gang at have en form for standard opsætning af datamater hvor alt
nødvendigt er installeret, som AVmand skal man så blot sætte udstyrret op og
lave indholdet.

\textbf{Python}\\
Bliver brugt til de forskellige programmer og værktøjer der udgør resten af
AV-systemet.
Det burde kun være nødvendigt at kunne kode python for at udvikle/vedligeholde
systemet.


\newpage
\section{Tanken bag}
% \begin{center}
%   {\Huge \textit{ ``Hvad tænkte han dog på?''}}
% \end{center}
Den grundliggende filosofi bag systemet er at AV skal kunne komplementere eller understøtte en
revy uden at komme i vejen. Der skal dermed kun vises det som publikum skal se
og høre og intet andet.
Det kan være med til at bryde indlevelsen og virke utroligt amatøragtigt lige så
snart publikum ser en cursor, en film der maksimeres eller rammerne på et
vindue. Det er sådanne fejltagelser der hiver folk tilbage til virkeligheden i
auditoriet fremfor den fiktion man skaber.
Man skal ikke tænke over at det er AV, men at det er en mega fed revy, folk bør
således kun lægge mærke til AV når der sker noget ekstraordinært, og
forhåbentligt ikke på grund af noget der ikke skulle ske.

Det simpleste AV-system man kan lave (og som er det de fleste bare ville gøre
uden at tænke videre) er at have en datamat tilkoblet en
projektor. Datamaten kan derfra køre et presentationsprogram, f.eks. powerpoint.

\begin{figure}[h!]
  \centering
  \begin{dot2tex}
    digraph{
      rankdir=TD;
      node [shape=ellipse];
      d [label="Datamat"];
      p [shape=trapezium, label="Projektor"];

      d -> p;
    }
  \end{dot2tex}
  \caption{Simpel AV opsætning.}
\end{figure}

Dette er dog ret primitivt.
Det er utroligt nemt at komme til at lave fejl, så som lige at få skubbet
cursoren hen på et andet desktop eller at filmen spiller på den forkerte skærm.
Ting der er sket gang på gang til mange revyer, og som hurtigt får publikum til
at råbe \textit{``TeXniken sejler!''}.

Derudover har man et problem hvis man skal bruge mere end 1 projektor.

\begin{figure}[h!]
  \centering
  \begin{dot2tex}
    digraph{
      rankdir=TD;
      node [shape=ellipse];
      d [label="Datamat"];
      p1 [shape=trapezium, label="Projektor"];
      p2 [shape=trapezium, label="Projektor"];
      p3 [shape=trapezium, label="Projektor"];

      d -> p1;
      d -> p2;
      d -> p3;
    }
  \end{dot2tex}
  \caption{En datamat, flere projektorer.}
\end{figure}

\begin{figure}[h!]
  \centering
  \begin{dot2tex}
    digraph{
      rankdir=TD;
      node [shape=ellipse];
      d1 [label="Datamat"];
      d2 [label="Datamat"];
      d3 [label="Datamat"];
      p1 [shape=trapezium, label="Projektor"];
      p2 [shape=trapezium, label="Projektor"];
      p3 [shape=trapezium, label="Projektor"];

      d1 -> p1;
      d2 -> p2;
      d3 -> p3;
    }
  \end{dot2tex}
  \caption{Flere datamater, flere projektorer.}
\end{figure}

Man kan enten koble flere projektorer på en datamat, hvilket kræver en datamat
der er i stand til dette, hvilket ikke særligt mange er. Og det kan give en
hovedepine hvis man samtidigt prøver at holde det ``skjult''.
Alternativt kan man have flere datamater koblet til hver sin projektor.
Nu skal man så bare navigere rundt mellem en helt masse maskiner eller have en
AV-mand pr. maskine hvilket heller ikke er særligt praktisk.
Desuden har vi stadig problemet med at man let kommer til at dumme sig på samme
måde som i den første opsætning.

Det vi i virkeligheden ønsker os er en abstraktion mellem det at \textit{styrre} AV og
det at \textit{vise} AV. Hvis tingene kører adskilt og har klart definerede
ansvarsområder kommer man ikke lige så let til at lave fejl.
Vi ønsker derfor at have en central maskine der styrer det hele, denne
kommunikerer med andre maskiner der sørger for at vise ting via projektorer.


\begin{figure}[h!]
  \centering
  \begin{dot2tex}
    digraph{
      rankdir=TD;
      node [shape=ellipse];
      c [label="Kontrol datamat"];
      d1 [shape=box, label="Arbejder Datamat"];
      d2 [shape=box, label="Arbejder Datamat"];
      p1 [shape=trapezium, label="Projektor"];
      p2 [shape=trapezium, label="Projektor"];

      c -> d1;
      c -> d2;
      d1 -> p1;
      d2 -> p2;
    }
  \end{dot2tex}
  \caption{Central styring, flere arbejdere med hver deres projektor.}
\end{figure}

Det er dette princip DIKUrevyens AV-system benytter sig af.

\note{Lidt baggrundshistorie for de nørdede.\\
  Det første hjemmelavede AV-system blev lavet af Troels Henriksen.
  Han lavede et programmeringssprog \textit{Sindre} i Haskell til at lave
  grafiske grænseflader. I det sprog lavede han et \texttt{dmenu} lignende program
  kaldet \textit{sinmenu} som han brugte som interface til systemet. Selve overteksterne
  bestod af en pdf som han lokalt oversatte tilbage til rå tekst som blev fodret ind i
  et script der brugte hans grænseflade. Dette script kommunikerede med en
  server koblet til en projektor over en \texttt{ssh}-forbindelse. \texttt{Xpdf} kan
  køres som en server der kan modtage kommandoer fra en kommandolinje, på denne
  måde kunne man derfor styre overtekster over ssh.

  Desværre skalerede løsningen ikke særligt godt. Da der begyndte at komme mange
  AV-effekter kunne man ikke både køre overtekster og andet AV alene.

  Dette blev (i hvertfald forsøgt) rettet op på da jeg (Søren Pilgård) efterfølgende begyndte at skrive et nyt
  AV-system til DIKUrevyen 2012.
  At skrive et AV-system er dog ikke en nem tjans og det endte med at foregå i en
  løbende process med forbedringer til hver revy.

  Den føte forbedrede udgave udskiftede interfacet man som AV-mand arbejdede i.
  Tanken var at istedet for at arbejde med pdf'en og hive teksten ud af dette,
  kunne man istedet bruge et værktøj der arbejde direkte på \LaTeX{} koden og så
  fjernstyrede pdf-læseren ud fra dette. Emacs blev valgt som base for systemet.
  Det er et system jeg allerede var bekendt med og som jeg i forvejen benyttede
  til at arbejde med \LaTeX{}. Med Emacs kunne man derfor benytte samme program
  til at skrive overteksterne og til at vise/styre dem.
  Den store force ved Emacs er at det er ekstremt programmerbart, man kunne
  derfor nemt bygge av-systemet ovenpå. Ligeledes var det muligt for Emacs at
  kigge i \LaTeX{} koden mens den blev fremvist for at finde indlejret kode som
  ikke blev vist, men som Emacs kunne evaluere for at gøre andre ting samtidigt.
  Dette kunne f.eks. bruges til at afspille lyde under en sang eller vise en
  video.

  Hurtigt viste det sig at være et stærkt værktøj at kunne bruge Emacs til at
  styre hele revyen. Der viste sig dog en ulempe i at bruge forskellige
  programmer til visningen af hver ting. Man kunne f.eks. ikke vise to
  forskellige billeder ovenpå hinanden.
  Ligeledes var der problemer med at en del programmer tog tid at starte op, og
  at pdf-læseren var lang tid om at reagere på at Emacs bad den om at skifte
  slide. Et halvt til et helt sekunds forsinkelse lyder måske ikke af meget, men
  er besværligt når man skal time en linjes overtekst til en hurtig sang.

  Derfor opstod idéen om at bygge et dedikeret program til at vise alt hvad
  Emacs (masteren) kommanderede den til. En første udgave blev udviklet der tog
  udgangspunkt i java-frameworket \url{https://processing.org/} som så kunne
  programmeres til at vise forskellige ting. Til sin sketch kunne man så lave et
  lille javaprogram som man kunne fjernstyre fra Emacs.
  Så blev der programmeret en pdf-viser som Emacs fjernstyrede til at vise
  overtekster. Ligeledes blev der programmeret en billede viser og en lyd
  afspiller osv. Det viste sig dog ikke at være den bedste løsning. I praksis er
  det meget få af de numre der har behov for AV der har specielle ønsker, som
  oftest er det bare en kombination af at vise billeder, lyde, videoer og
  pdf'er. At alting kørte som forskellige programmerede dele gjorde så at man
  ikke nemt kunne opbygge AV'en ud fra allerede definerede dele men skulle
  opbygge alting fra bunden lige så snart man ville bare en lille smule.

  Derfor endte det med at en helt anden backend til arbejderne blev
  programmeret. Idéen var at man let skulle kunne kombinere allerede
  programmerede dele til sit behov. Det blev besluttet at denne backend skulle
  programmeres i elisp, eller i det mindste et sprog der mindede meget om dette.
  På den måde var der således ikke den store forskel på at scripte ting der
  kørte i Emacs på masteren , og ting der kørte i backenden på arbejderne.

  Det er dette system der fungerer som DIKUrevyens AV-system nu.
}
\note{Skal følgende merges med ovenstående?}
DIKUrevyens AV-system består af en central grænseflade der kan kommunikere med flere
forskellige maskiner der kan vise AV materiale.

Selve grænsefladen er udviklet som en række udvidelser til Emacs.
Grunden til dette er at da jeg startede indså jeg at systemet skulle kunne
følgende ting:
\begin{itemize}
\item Kommunikere med en server der viser indholdet.
\item Det skulle kunne vise ting overskueligt i en grafisk grænseflade.
\item Der skulle være mulighed for at indlejre scripts så man kan køre ting
  automatisk på bestemte steder i forestillingen.
\item Det ville være praktisk hvis man kunne rette fejl i texkoden mens man
  viste pdf'en da man ellers risikere at glemme dem.
\end{itemize}

Det gik hurtigt op for mig at det ville være fjollet at udvikle noget fra bunden
da Emacs i forvejen kunne meget af dette. Desuden er jeg Emacsmand og så hurtigt
hvordan en integration ville være nice.
Den grundliggende formel for at vise overtekster er at man åbner et LaTeX
dokument der bruger beamer pakken til at lave overteksterne. Så starter man
\texttt{ubertex}minor modet, dette sørger for at pdfen bliver lagt op på
serveren. Herefter skjules de fleste LaTeXkommandoer og et overlay lægges der
viser hvad der bliver vist. Man kan så trykke ``næste'' hvilket rykker overlayet ned
og synkroniserer serveren til at vise det tilsvarende slide i pdfen.
Man kan også trykke vilkårlige steder i tex filen og rykke direkter hertil i
overteksterne. Desuden kan der indsættes kode der eksekveres når man når til det
pågældende slide.

% TODO: billede

Derudover findes minormodet \texttt{uberscript} der lader en afvikle scripts.
Det kunne f.eks. være en sketch med en række lydeffekter eller et kald til en video.
Et script kunne også være aktoversigten hvorfra man ved at trykke ``næste''
kommer ind i det næste nummer, og når dette er færdigt kommer man så tilbage og
er klar til næste.

Hvis alt går som det skal, skal man som AVmand kun trykke på en knap (næste) for
at afvikle en revy.
Dette må være essensen af et godt AV system, når man kun skal tage sig af
timingen på skuespillerne/sangerne samt disses fejltagelser.

\newpage
\section{- Guidelines til en AV mand}
\label{sec:god-av-mand}
% \begin{center}
%   {\Huge \textit{ ``Lyt til mig en gang''}}
% \end{center}


\textit{TODO: Her kommer der til at være en række mere generele råd om hvordan
  man laver god AV}

Det vigtigste man skal huske for at være en god AV-mand, er at have det sjovt!
Hvis du ikke har det sjovt, bliver det ikke en sjov revy.
Så lad være med at stresse for meget, ignorerer når de andre sejler og glemmer
at give dig hvad du skal bruge og hyg dig.
Hvis du først lader dig blive presset af det hele mister du overblikket og
kreativiteten, så hellere sige ``Det går nok'' og tage det som det kommer, det
skal nok blive en revy.

I dette kapitel findes en række råd og guidelines. Hvordan man laver AV er en
smagssag og afhænger meget af hvad man laver og sammenhængen. AV er således en
kunst, og selvom det der står her kan lyde som regler, ved enhver kunstner
hvornår man skal bryde dem. Nogle gange bryder man reglerne for at få tingene
til at gå op i en højere enhed, andre gange for at lave et helt nyt regelsæt.
Selvom du måske er anarkist og vil gå din egne veje som AV-mand vil jeg dog
anbefale dig at læse og forstå hvad der står her.

\note{Husk der er mange måder at lave AV på, det her er nogle af de erfaringer
  der har virket godt}

\note{Sæt altid højere standarder til dig selv end andre}

\subsection{Om at lave gode overtekster}
% 6 ord
% Læser i høj grad på ordenes form.
% Drfr kn mn stdg lse dnne stnng
% Sætningen læses relativt hurtigt, men var den længere ville man skulle læse hele
% teksten.
% Man kan ca overskue

% <billede af en 4 lego brik>
% kan genkendes med det samme

% <billede af en 9'er lego klods>
% man skal tælle


I afsnit \ref{subsec:konvertering-manus-overtekst} kan du læse mere om hvordan
man via systemet let kan konvertere fra et .tex manuskript til overtekster.


\subsection{Om at vise overtekster}
\note{Noget om timingen}

% husk blanke slides

\newpage
\section{- Opsætning af tekniken}
% \begin{center}
%   {\Huge \textit{ ``Det er faktisk slet ikke så svært''}}
% \end{center}
\note{Dette er en meget gammel gennemgang}

Til en standard opsætning ala DIKUrevyens skal du bruge:
\begin{figure}[h!]
  \centering
  \begin{dot2tex}
    digraph{
      rankdir=TD;
      node [shape=ellipse];
      c [label="Kontrol datamat"];
      h [shape=octagon, label="LAN-hub"];
      d1 [shape=box, label="Brok"];
      d2 [shape=box, label="Intro"];
      p1 [shape=trapezium, label="Projektor"];
      p2 [shape=trapezium, label="Projektor"];

      c -> h;
      h -> d1;
      h -> d2;
      d1 -> p1;
      d2 -> p2;
    }
  \end{dot2tex}
  \caption{Standard opsætning}
\end{figure}

Hvis det ønskes (Når det er færdigt) kan man køre med fjernstyrede projektorklapper.
\begin{figure}[h!]
  \centering
  \begin{dot2tex}
    digraph{
      rankdir=TD;
      node [shape=ellipse];
      c [label="Kontrol datamat"];
      h [shape=octagon, label="LAN-hub"];
      a [shape=diamond, label="arduino"];
      d1 [shape=box, label="Brok"];
      d2 [shape=box, label="Intro"];
      p1 [shape=trapezium, label="Projektor"];
      p2 [shape=trapezium, label="Projektor"];

      c -> h;
      h -> d1;
      h -> d2;
      h -> a;
      a -> p1;
      d1 -> p1;
      d2 -> p2;
    }
  \end{dot2tex}
  \caption{Standard opsætning, med arduino til projektorklap}
\end{figure}

\newpage
Systemet kan udvides, her ses f.eks. en opsætning med højtex (uden
projektorklapper) som brugt til DIKU Jubilæumsrevy.
\begin{figure}[h!]
  \centering
  \begin{dot2tex}
    digraph{
      rankdir=TD;
      node [shape=ellipse];
      c [label="Kontrol datamat"];
      h [shape=octagon, label="LAN-hub"];
      d1 [shape=box, label="Brok"];
      d2 [shape=box, label="Intro"];
      p1 [shape=trapezium, label="Projektor"];
      p2 [shape=trapezium, label="Projektor"];
      h0 [shape=box, label="left"];
      h1 [shape=box, label="mid"];
      h2 [shape=box, label="right"];
      hp0 [shape=trapezium, label="Projektor"];
      hp1 [shape=trapezium, label="Projektor"];
      hp2 [shape=trapezium, label="Projektor"];

      c -> h;
      h -> d1;
      h -> d2;
      d1 -> p1;
      d2 -> p2;

      h -> h0;
      h -> h1;
      h -> h2;
      h0 -> hp0;
      h1 -> hp1;
      h2 -> hp2;
    }
  \end{dot2tex}
  \caption{Højtex}
\end{figure}


\subsection{Kontrol}
Dette er din primære indgang til systemet. Jeg anbefaler at man bruger en
Foldedatamat, gerne ens egen bærbare.
En bærbar har den fordel at skærmen kan indstilles til både at sidde ned og stå
op. Derudover har den tastatur og mus indbygget så det ikke fylder i den ellers
rodede texnik. Og så kan man tage den med sig så man kan arbejde videre andre
steder. F.eks. er det praktisk at kunne plugge den ud så man kan arbejde videre
på sine overtekster til et ellers kedeligt senemøde i hyggehjørnet.


\subsection{Projektorer}
Til DIKUrevyen bruges der typisk i hvertfald 2 projektorer.
Der bruges én der peger op på lærredet over scenen, her vises der overtekster,
almindelige av-ting til sketches samt små film.

Kantinen ejer en stor Epson som bruges til introen/av på scenen

DIKUs projektorer bruges til alt andet.
DIKU har en række forskellige projektorer, hvor det kan være svært at kende
forskel på en del af dem.
%% TODO: udvid med mere konkrete råd om projektorer.

En god tradition, som en AV-mand bør holde i hævd, er at rense projektornes filtre
når man henter dem i begyndelsen af revyugen.
Der er tilsyneladende ikke andre der gør det, så lad det blive en del af
rutinen. Så overopheder de ikke lige så nemt.
Husk at sætte filteret rigtigt i igen, der er ofte mange tapper og dibedutter
der skal passes ind i filteret for at det sidder korrekt.


Til projektorene er der bygget en række projektorkasser af gamle colakasser.
Disse gør det en del nemmere at indstille projektorene.\\
\textit{I gamle dage, da jeg var ond. Da blev projektorene stablet på bøger til
  de stod sirligt, den ærede AV-mand Troels Henriksen brugte mangt en stund på
  at bande og svovle når disse blev rykket}\\
Nu gør projektorkasserne det en del nemmere da man kan stripse/tape kasserne
fast og det hele bliver langt mere stabilt.
Desuden kan der komme langt mere luft til.\\
\textit{I gamle dage, da jeg var ond. Da blev projektorne så varme at de
  overophedede, så vi måtte til HCØ og hente tøris til køling. Det gav også
  kolde drikkevare (til tider frosne).}

Se afsnittene om Ubertex, Uberscript og Emacs for at finde ud af hvordan
softwaren bruges.

Der er et forlænger VGA kabel.
En ide er at sætte det i krok og føre det over til projektoren, så kan man nemt
skifte hvilken maskine der er koblet til projektoren.
\note{TODO: Indsæt diagram fra bagsiden af side 10, Den Store AV-bog printet 27.
  september 2016}

\subsubsection{Projektorklapper}
En ulempe ved projektore er at deres ``sorte'' ikke er mangel på lys, men bare
\textit{mindre} lys. Det betyder at når alt lyset i StUP1 er slukket og en
projektor står og lyser, er der stadig ret meget lys på scenen. Det ser
\textbf{MEGET} dumt ud og gør at man kan se hvad der sker på scenen.
Det er primært et problem for projektoren til overteksterne og kan accepteres
til f.eks. højtex (da loftet ikke er lige så reflekterende)

Det er desværre ikke en løsning at slukke/tænde projektorne da det tager for lang
tid/er besværligt/ skydder farver op når de tændes.

Derfor bruges en `projektorklap'. Der har i mange år været brugt en halv
papkasse på en stang. Det fungerer, men det kan godt være lidt stressene. Man
skal huske at få klappen på når man er færdig med en sang og har travlt med at
huske hvad der nu skal ske. Til tider sker det også at man glemmer at tage klappen
af, man når typisk at panikke lidt når der ikke kommer noget billede frem. Det
kan betyde at man misser de første par overtekster eller starten af en film.


For at løse dette er det planen at der bygges nogle arduinoer der kan kobles på netværket, disse
kontrolerer en klap foran projektoren. På denne måde kan projektoren automatisk
begynde at skyde billedet op på lærredet
Arduino ftw!

\subsection{Netværk}
\note{Numerer datamaterne i [1; 254] Hvor xxx er tallet til maskinen
  Vis f.eks. et eksempel}
\subsubsection{Statisk IP}
\marginnote{\note{How to know?}}
Hvis dit netværks interface hedder enp0s25
\begin{verbatim}
sudo ip link set enp0s25 up
sudo ip addr add 192.168.0.XXX/24 dev enp0s25
\end{verbatim}

alternativt
\begin{verbatim}
sudo ifconfig eth0 192.168.0.XXX
\end{verbatim}

\subsubsection{Hosts}
Da det kan være svært at huske alle ip-addreserne kan man i stedet navngive
maskinerne.
Dette gøres lokalt i \texttt{/etc/hosts}
et eksempel:

\note{Indfør noter, skelne mellem eksempel, og default værdier.
  Nævn at arbejdere har en standard statisk IP som de bruger automatisk.}
\marginnote{controller: 192.168.0.10\\
  overtex istedet for intro?}
\begin{verbatim}
192.168.0.20    intro
192.168.0.30    brok
192.168.0.40    left
192.168.0.50    mid
192.168.0.60    right
\end{verbatim}
\marginnote{back: 192.168.0.31\\Back(bach): Back mirror of Brok}
Nu kan man ssh'e ind ved blot at skrive:

\begin{verbatim}
ssh brok
\end{verbatim}

Det virker også med scp ol.

\subsubsection{Forbindelse uden login}
For at logge ind på systemerne over netværket bruges ssh.
Da det bliver jævnt irriterende hele tiden at skulle taste løsener kan man lægge
sin offentlige ssh nøgle ind på de forskellige maskiner
\begin{verbatim}
cat .ssh/authorized_keys | ssh revy@192.168.0.30 "cat >> ~/.ssh/authorized_keys"
\end{verbatim}
det kan være du først skal oprette mappen \texttt{.ssh}.

alternativt kan du bruge
\begin{verbatim}
ssh-copy-id revy@192.168.0.30
\end{verbatim}
Udskift \texttt{revy} og ip'en med den relevante bruger og ip.
Husk at du skal have en ssh nøgle først
dannes med:

\begin{verbatim}
ssh-keygen
\end{verbatim}


\subsubsection{ssh mount}
I stedet for at scp'e alt muligt crap frem og tibage kan man benytte sig af et
ssh mount, som er et filsystem over ssh.
Jeg anbefaler at man har filerne liggende lokalt og fra hver maskine der skal
kommunikeres med køres \texttt{sshfs} som får filerne frem.

\textbf{Installer sshfs}
F.eks.:
\begin{verbatim}
sudo pacman -S sshfs
\end{verbatim}

\textbf{fuse}
\texttt{sshfs} benytter fuse
\begin{verbatim}
sudo modprobe fuse
\end{verbatim}
Hvis dette ikke virker kan det være fordi kernen er blevvet opdateret, men ikke
genstartet. I så fald, kan du prøve at genstarte maskinen.

\subsubsection{Internet gateway}
Hvis du er intereseret i at få internet på det lokale netværk kan man lave en
gateway.
Dette består i at en af datamaterne på det lokale netværk også er forbundet til
et netværk med internet, f.eks. eduroam.

Der er en guide på \url{sigkill.dk/writings/guides/gateway.html}

På gateway datamaten sættes først den statiske ip (allerede gjordt i forrige
trin)
Her efter køres gateway.sh som root (hentes på siden)

På clienterne der vil udnytte gatewayen sørger man først for at der er en
statisk ip.

herefter skrives
\begin{verbatim}
ip route add default via 192.168.0.XXX
\end{verbatim}

eller alternativt
\begin{verbatim}
route add default gw 192.168.0.XXX
\end{verbatim}
Hvor ``192.168.0.XXX'' er ip'en til gatewaymaskinen

herefter sættes en dns server, f.eks. google
\begin{verbatim}
echo nameserver 8.8.8.8 > /etc/resolv.conf
\end{verbatim}

Hvis dette ikke virker kan man evt. kigge på
\url{https://wiki.archlinux.org/index.php/Internet_Share} for en ip baseret løsning
\note{Baser hele afsnittet på dette (ip)

  uddyb her}

\note{internet0 = wlp3s0\\net0 =enp0s25}
\subsection{Arbejdere}
Arbejdere er de maskiner/servere der er sluttet til projektorne. Det er deres
rolle at vise AV-indholdet det kunne være overtekster, film, lydeffekter,
slides, billeder osv.

På nuværende tidspunkt bruges \texttt{mplayer} til at vise film, \texttt{feh}
til at vise billeder og \texttt{xpdf} til at vise pdf'er.
Det smarte ved \texttt{xpdf} er at det kan startes som en server som kan modtage
kommandoer via en terminal.

Ideen er at man fra sin kontroldatamat kan ssh'e ind på arbejderne og afvikle
den kommando man skal bruge. Emacs med ubertex/uberscript tilbyder basalt set et
interface til at gøre dette nemt.

Det er planen at programmet \texttt{Zeigen} udvikles.
Dette skal erstatte \texttt{feh} og \texttt{xpdf}.
I stedet for at skulle ssh'e ind lytter \texttt{Zeigen} til en port og udfører
kommandoer på givne klokkeslet. Så kan man sige ``om 1/2 sekund skal du afspille
denne video'' på 3 forskellige maskiner, hvorved videoen vises simultant via
\texttt{mplayer}.


Et eksempel på en arbejderdatamat er \texttt{brok} der bruges til overtekster og ting der
skal vises på lærredet.
Derudover findes \texttt{left}, \texttt{mid} og \texttt{right} der bruges til at vise \textit{højtex}.


\subsection{Hvordan virker en arbejder}
Og hvordan bruger man dem.
``plug and play''
auto netværknetctl + revynet
default statiske ip'er
Forhåbentligt skal man som almindelig AV-mand rode så lidt som muligt, kun sætte
maskinerne fysisk op og tænde, så ordnes resten via masteren iform af Emacs.
Indimellem kan det dog være nødvendigt med maintanence (Det ordner Søren).


\subsection{Brok (Deprecated)}
\textbf{Her er en guide om at få brok til at virke:}

Sæt ham til
Projektor, tastatur, netværk og strøm.

Hvis der skal spilles lyd fra brok bør der være jord på da der ellers kan komme
en brummen.

Brok har på skrivende tidspunkt ubunut 11.04 natty.
Det betyder også at der er gnome, unity og ting og ækle sager på.
Disse har en tendens til at gøre livet surt, da de overskriver xorg
konfigurationer og gør mærkelige ting man ikke helt kan gennemskue.

\textit{Dette er \uline{IKKE} optimalt!
  Support til 11.04 udløb oktober 2012. Der bør installeres et ordentligt
  styresystem. UDEN gui (installeres seperat).}

\textbf{1: Tænd brok}\\
Brok logger automatisk ind med brugeren \textbf{\textit{revy}} med løsnet \textbf{\textit{hamster}}
Herefter starter unity der har en `kør' dialog ting.

\textbf{2: Start en terminal}\\
Skriv \code{terminal} og tryk enter.
Der vil nu komme en terminal op i hjørnet.

\textbf{3: Kom på netværket}\\
\begin{verbatim}
sudo ifconfig eth0 192.168.0.30
\end{verbatim}
Giv ham en passende ip du kan huske.

\textbf{4: Opret SSH forbindelse}
Der kører allerede en ssh daemon.
forbind til \textbf{\textit{revy}} med løsnet \textbf{\textit{hamster}}.
Du kan evt. uploade din offentlige ssh-nøgle så du slipper for at logge ind.
\begin{verbatim}
ssh revy@192.168.0.30
\end{verbatim}

\textbf{5: Stop gdm}\\
Gnome Desktop Manager skal lukkes.
\begin{verbatim}
sudo service gdm stop
\end{verbatim}
Dette lukker hele den grafiske grænseflade, inklusiv X.

\textbf{6: Start screen}\\
\begin{verbatim}
screen
\end{verbatim}
screen køres så vi kan starte X i baggrunden.
Når programmet startes vises der en menutekst, bare tryk enter.
\texttt{screen} kører nu.

\textbf{7: startx}\\
Start X manuelt.
\begin{verbatim}
startx
\end{verbatim}
Dette starter X op hvorefter .xinitrc eksekveres.

Som det er lige nu startes \texttt{xmonad} som windowmanager.
Det betyder at skærmen pr. default er sort når x er startet


\textbf{8: detach fra screen}\\
Tryk \texttt{Ctrl-a d}. Dette lader alt der kører i screen fortsætte upåvirket
mens du kommer tilbage til det forrige miljø.
for at reattache skriv da \code{screen -r}.

\textbf{9: test xmonad}\\
Test om xmonad virker. På broks tastatur trykkes \texttt{Alt-Shift-Enter}.
En terminal burde åbne sig.
Luk igen med \texttt{Alt-Shift-c}.

\textit{Læs evt. afsnitet om xmonad.}

Da terminalen er en default \texttt{xterm} med hvid baggrund er dette et snedigt trick
til at se hele området projektoren kan lyse op.
Jeg bruger dette ofte når jeg tweaker projektoren.



\textbf{10: prøv ting af}
Med en ssh forbindelse åben, lad os prøve om vi kan få noget til at virke.
Start med at vælge det `display' der skal vises grafiske ting på
\begin{verbatim}
export DISPLAY=:0
\end{verbatim}
Dette skal køres for hver gang du ssh'er ind.

start nu xpdf med en af de gamle overtekst filer der ligger et eller andet sted.

Se sektionen om kommandoer for nærmere detaljer

Hvis man gerne vil lave flere ting simultant på brok kan man sagtens åbne flere
lokale terminaler og ssh'e ind parallelt.




\subsubsection{Ting der køres på brok}
For at hacke uden om alt muligt gøjl køres et par scripts gennem
\texttt{.xinitrc}

\textbf{swarp}\\
\texttt{swarp} er et program der flytter musse-markøren (cursoren) til et givent koordinat.
Swarp findes på \url{tools.suckless.org/swarp}.
\texttt{swarp} findes desuden i arch's repository, det kan være den også findes til
debian baserede styresystemer, potientielt i en suckless pakke.

\texttt{swarp} køres på brok med argumenterne 20000 20000, hvilket flytter
markøren ad helvede til.

istedet for \texttt{swarp} kan man bruge \texttt{unclutter}
med kommandoen \code{unclutter -idle 0 -root \&}
Som i fjern cursoren efter 0 sekunders delay efter bevægelse inklusiv når
cursoren er over rod baggrunden (altså ikke kun over vinduer).

\textbf{fixsleep}
\texttt{fixsleep.sh} er et script der forsøger at forhindre X i at slukke skærm outputet.
X bruger et system der hedder DPMS (Display Power Management Signaling) til
automatisk at slukke skærm outputet efter en periode uden tastaturaktivitet.

fixsleep benytter følgende to kommandoer
\begin{verbatim}
xset dpms 0 30000 40000
xset s 30000
\end{verbatim}
den første linje dækker over \texttt{xset dpms [standby [suspend [off]]]}
den anden linje dækker over \texttt{xset s [timeout [cycle]]}

\textbf{keepon}
\begin{verbatim}
xset dpms force on
\end{verbatim}
Denne kommando forcer dpms fra, (det svarer til at at trykke på en tast)

\texttt{keepon.sh} er et script der ligger i baggrunden og kører denne kommando
whert 30 sekund.


fixsleep og keepon forsøger begge at holde dpms stangen ved at lave aktivitet.
Man kan måske istedet bruge
\begin{verbatim}
xset -dpms; xset s off
\end{verbatim}
Til at slå dpms fra.
De to systemer kan dog \uline{ikke} blandes.

Man kan se om dpms er slået til ved at kalde \code{xset -q}
\subsection{Xmonad}
Windos/super eller alt som modifier knap.

\newpage
\section{- Master}
% Der sørges automatisk for at det korrekte major mode køres på .tex og .el
% filer
Se også \ref{subsec:master-commands}

\subsection{ubersicht}
\label{subsec:ubersicht}
\subsection{ubertex}
\label{subsec:ubertex}
\subsection{Konvertering fra manuskript til overtekster}
\label{subsec:konvertering-manus-overtekst}

\newpage
\section*{\%Emacs (Deprecated, merges med Master)}
\note{Denne section blev fundet i en gammel udprintet udgave af
  den-store-av-bog, men kunne ikke findes i git historikken.
Den er noget udtateret i forhold til hvad vi gerne vil og antager lidt at folk
har styr på ting.}

Her er en kort introduktion til den del af revy-systemet der foregår i Emacs.
Emacser ikke bare en teksteditor men et helt miljø til at arbejde med tekst og
evaluere lisp kode (specifikt en dialekt kaldet Emacs lisp eller bare elisp)

Som AV-mand er en stor del af ens arbejde i bund og grund at arbejde med den
tekst der skal vises og afvikle den kode der viser det, derfor er Emacs et så
centralt element i systemet. I appendix XXX\marginnote{\note{Nej der er ej}} er en
introduktion til at bruge en forsimplet udgave af Emacs. Hvis man synes at Emacs
er lidt besværligt at bruge kan man istedet kigge på ``Simple Emacs'' som er en
konfiguration af Emacs designet til revyen og som minder om et mere traditionelt
program (det kan dog stadig anbefales at sætte sig ind i hvordan Emacs virker).

Revyens AV-system er designet som en række filer indeholdende forskelig
funktionalitet. Som udgangspunkt er det nok at tilføje \code{ubertex/emacs}
mappen til sin \code{load-path} og sw indlæse ``revy''

\begin{verbatim}
(add-to-list 'load-path "path/to/ubertex/emacs")
(require 'revy)
\end{verbatim}

Som udgangspunkt defineres blot to funktioner \code{revy-create} og
\code{recy-load} Først når disse bruges indlæses resten af funktionaliteten. På
denne måde slipper man for at have mere end højst nødvendigt indlæst hvis man
benytter Emacs til andre formål.

Da Emacs ikke har ``namespaces'' eller en anden form for modulsystem er det med
vilje designet med så \textit{få} hjælpe funktioner som muligt. Tanken er at
enhver funktion virker selvstændigt så man som bruger af systemet kan se på
enhver funktion kaledt \code{revy-} og bruge den i sine egne scripts.
Dette betyder dog at enkelte funktioner kan blive lidt lange og indviklede, det
er en pris der må betales for at det bliver så simpelt at programmere op i mod
som muligt.\marginnote{Implementerings noter?}

\subsection{revy.el}
Indeholder grundfunktionaliteten. De to funktioner \code{revy-create} og
\code{revy-load} henholdsvist opretter og indlæser en revy. Herefter indlæses
revyen samt funktionerne til at arbejde med denne.

Det er ikke nok bare at åbne en \code{.revy} fil manuelt.

\subsection{uberrevy.el}
Binder revy modesne sammen. Overordnet er en række variabler defineret her. Når
en revy indlæses, loades uberrevy som sørger for at loade alt andet.

\subsection{ubercom.el}
Indeholder funktionalitet til at kommunikere med leiter (som står for
kommunikationen med arbejderne).

Derudover er en række hjælpefunktioner defineret til at gøre kommunikationen med
arbejderne lettere. Disse kan både bruges interaktivt eller i koden.

\subsection{ubersicht.el}
Ubersicht er et system til at holde styr på scripts. I bund og grund er det et
system bygget til at evaluere et emcas script skridt for skridt. Dette
sammenholdt med en lang række revyspeciffike funktioner giver en stor
fleksibilitet.

Hvert skridt består af en \textit{instruktion} (``instruction'' i koden)

Man kan bruge ubersicht til at lave en fil med aktoversigten, ved hvert nummer
benytter man så ubersichts ``næste'' funktionalitet til at bevæge sig videre til
næste nummer, dette kan så være en kommado der åbner f.eks. en sang eller en
sketch. Man kan ligeledes benytte ubersicht til at have en sekvens af
instruktioner der afvikles under en sketch. Det kunne være at der undervejs i
sketchen skal afspilles nogle lyde, billedr eller film, disse vil være skrevet
ind i filen som en række instruktioner. Når man i sketchen når til at det er tid
til at afvikle en instruktion trykker man blot ``næste. Til sidst vil man typisk
have en ``afslut'' instruktion der returnerer tilbage til det ubersict scrpit
der åbnede det nuværende, det kunne f.eks. være aktoversigten.

Systemet er lavet fleksibelt for at gøre det bredt anvendeligt men en typisk
brug består i én aktfordeling liggende i en fil kaldet \code{revyens-navn.revy}
eller noget i den stil. Indholdet kunne se således ud:

\begin{verbatim}
(revy-start)


Akt 1
(revy-open "sketches/velkommen.sketch")
(revy-open "sange/en_sjov_sang.tex")
(revy-open "sketches/sketchen_med_ordspil.sketch")
(revy-open "video/dum_video.sketch")
(revy-open "sange/nu_er_der_pause_sangen.tex")

Akt2
(revy-open "sange/revyen_er_sjov.tex")
(revy-open "sketches/dårlig-sketch.sketch")
(revy-open "sange/sidste_sang.tex")

Ekstranumre
(revy-open "video/en_anden_video.sketch")
(revy-open "sange/sjovt_ekstranummer.tex")
\end{verbatim}

Den første instruktion man evaluerer er at starte revyen, herefter kører man
igennem de forskellige numre i revyen.

En instruktion er et kald til en elisp funktion\marginnote{eller makro} og
består \textbf{altid} af en startparentes på en ny linje. Dermed bliver
teksterne ``Akt 1'', ``Akt 2'' og ``Ekstranumre'' ignoreret. Der er dermed rig
mulighed for at sætte noter ind. Man kan skrive elisp kommentarer der begynder
med et \code{;}, man kan skrive elisp strenge \code{"Noget
  tekst"} eller blot skrive klartekst (Pas dog på, hvis filen
evalueres som normal elisp kode vil klartekst indskrevet direkte i filen blive
evalueret som kode, hvilket kan give en fejl).\marginnote{Som i *-config.el}

En markør, makerer hvilken instruktion der bliver udført på nuværende tidspunkt.
Markøren farver hele instruktionen så det er tydeligt. Husk at markøren (kaldet
``cursor'' i koden) ikke er det samme som tekstmarkøren hvor tekst bliver indsat
når der tastes (denne kaldes i Emacs for ``point'').

Overordnet er der 3 funktioner relateret til at bruge ubersicht

\code{revy-ubersicht-mode}\\
Er et Emacs major mode, baseret på det indbyggede
emacs-lisp-mode. Dette bruges primært for sine genvejstaster samt at holde styr
på den nuværende sketch. Når man kalder funktionen startes major modet.

\code{revy-ubersicht-next}\\
Rykker markøren frem og udfører den næste instruktion. (Rykker også point).

\code{revy-ubersicht-enter}\\
Eksekverer instruktionen som point er i ellerden efterfølgende.

Derudover har vi adgang til en lang rzkke funktioner der kan bruges som
instruktioner. I virkeligheden kan enhver elisp funktion bruges. En række
funktioner til brug for at styre og afvikle revyen er defineret og findes i
\code{uberinstructions.el}

\subsubsection{Instruktioner der er værd at kende}

\code{revy-start}\\
Starter en revy.

\code{revy-open}\\
Åbner og begynder eksekveringen af et nyt script eller overtekster.

\code{revy-nop}\\
Gør ingenting og ignorerer alle argumenter. Imodsætning til en kommentar
eksekveres denne instruktion stadig, den udfører bare ingenting. Dette kan
bruges hvis f.eks. er en sketch uden AV-indhold, man kan på denne måde stadig
følge med i hvor man er i revyen ved blot at trykke næste.

(alternativt kan man bruge \code{ignore} som gør næsten det samme.)

\code{revy-end-sketch}\\
Afslutter den nuværende sketch og returnerer til den der åbnede den. Hvis man
fra sin aktoversigt.revy har kaldt \code{revy-open} på en sketch, vil et kald
til \code{revy-end-sketch} i sketchen afslutte denne og hoppe tilbage til
aktoversigten så man kan fortsætte til næste nummer.

\code{revy-shell}\\
Er defineret i \code{ubercom.el}, og kan afvikle en shell kommando på enten den
lokale maskine eller på en arbejder.

\textit{TODO: denne funktion skal refaktoreres}

\subsection{ubertex.el}
Ubertex er på mange måder tilsvarende uberscript, men hvor man i et uberscript
benytter sig af elisp funktioner til at udføre en sekvens af instruktioner er
ubertex beregnet til at vise slidesne til en pdf skrevet i \LaTeX{} med beamer
klassen.

Pointen er at man skriver en latex fil med beamerkalssen og overtex pakken. Man
kan der definere hvert slide, samt pauserne id disse. Når nummeret skal vises
kører en pdf fremviser på arbejderne som styres via ubertex i Emacs. Systemet
minder om uberscript, en markørvires hvilken del af latexen der vises som slide
i pdf'en, og man kan så bladre ned igennem.

\begin{verbatim}
\documentclass[14pt]{beamer}
\usepackage[danish]{babel}
\usepackage[utf8]{inputenc}
\usepackage{overtex}

%% Melody: Den der disney sang, https://www.youtube.com/watch?v=dQw4w9WgXcQ

\begin{document}
\obeylines

\begin{overtex}
  % forspil
\end{overtex}

%% S1
\begin{overtex}
  Det her er første linje\pause
  den er lang så vi deler den
\end{overtex}

\begin{overtex}
  % blank
\end{overtex}

\begin{overtex}
  Mere sang\pause{} med en pause\pause
  det er vældigt smart
\end{overtex}
\begin{overtex}
  Flere linjer\pause
  i denne sang
\end{overtex}

\begin{overtex}
  Tralala\pause
  Tralala\pause
  Tralala\pause{} lalala
\end{overtex}

\begin{overtex}
  % slut
\end{overtex}
\begin{overtex}
  \elisp{(revy-end-sketch)}
\end{overtex}
\end{document}
\end{verbatim}

Når man afvikler en sang, bliver latexken skjult af vejen så man kan fokusere på
indholdet. Husk at naår man laver en ændring i latexen livestreames den ikke til
pdf-fremviseren, først skal latexkoden oversættes til en pdf, herefter skal
denne overføres til arbejderne der så skal have at vide at nu skal den nye pdf
vises istedet.

\subsection{manus.el}
Indeholder brugbar funktionalitet til at konvertere et \texttt{.tex} manuskript
til brugbare overtekster.

\code{revy-manus-mode}
Starter \code{revy-manus-mode}. Primært er formålet at stille en række
genvejstaster til rådighed.

\subsubsection{Konvertering fra manuskript til overtekster}
\begin{verbatim}
(add-to-list 'load-path "~/code/ubertex/emacs")
(require 'revy)
\end{verbatim}

\newpage
\section{- Worker}
\note{Generelt om hvordan en worker virker}
Se også \ref{subsec:worker-commands}

\newpage
\section{- Lisp}
Lisp er en famillie af programmeringssprog, en af de ældste udgaver er Emacs
Lisp, eller bare elisp som bruges til at udvide og opbygge programmet Emacs.
Elisp er et ældre sprog, fra 1985, derfor kan mange ting godt virke lidt
specielle og ``sære'', men overordnet fungerer det godt til formålet.
Hvis du kender til lisp i forvejen vil det hjælpe en hel del, men elisp minder
som sprog mere om common lisp end om scheme, racket eller clojure.

Til av-systemet bruges elisp til at programmere masteren. En variant wlisp kører
på workerne. wlisp er designet til at minde meget om elisp, og i praksis behøver
du typisk ikke at tænke dem som to forskellige sprog. Der er dog nogle små
forskelle hist og her, disse vil blive nævnt i dokumentationen herunder.

Ok, men hvad er et programmeringsprog? Et programmeringssprog er en måde at
sammensætte en sekvens af instruktioner som en ``computer'' kan forstå. Du kan
se det som en måde at lave en opskrift, men istedet for at forklare hvordan du
skal bage en æblekage, forklarer vi hvordan en arbejder skal tegne et billede
hvor vi ønsker. Der findes mange forskellige programmeringssprog der kan se
meget forskellige ud, men som alle bygger på nogle fælles underliggende træk.


Udover denne guide til lisp har Emacs en kæmpe mængde dokumentation af alle
mulige systemer til rådighed. I Emacs kan du trykke \code{C-h i} for at få en
liste over alle disse. Der er en guide til at bruge Emacs, men derudover er der
en ``Emacs Lisp Intro'' til at forklare elisp for begyndere, desuden er der en
stor section om elisp der dokumenterer alle de elisp dele Emacs kommer med som
standard.

I Emacs menu, som enten vises i toppen af vinduet som de fleste andre
programmer, alternativt hvis denne menubar er slået fra (som mange gør) kan man
få den frem midlertidigt ved at holde \code{Ctrl} nede og så højreklikke.
I menuen ``Help'' kan du i undermenuen ``More manuals'' også finde elisp
introduktionen samt elisp referencerne.
\subsection{Delene i Lisp / De forskellige lisp dele}
\subsubsection{Værdier}
I elisp og wlisp fremover bare nævnt som lisp har vi et koncept om ``værdier''.
En værdi er noget konkret som f.eks et heltal, et komma tal, en tekst streng, en
liste eller et symbol, derudover findes kode der kan arbejde med disse, det
kunne f.eks. være funktionen \code{+} der tager et par tal og returnere et andet
tal. Alt kode returner en værdi, nogle gange bliver værdi ignoreret, f.eks.
fordi den er ligegyldig. Nogen kange kan kode gøre mere end bare returnere en
værdi, så siger vi at det har en ``sideeffekt'' fordi det har en effekt udover
bare at returnere en værdi. En sideeffekt kunne være at vise et billede,
afspille en lyd, åbne en fil eller noget helt andet. En af de ting der gør lisp
lidt specielt er at det er ret nemt at lave kode der arbejder med andet kode,
fordi vi faktisk kan se kode som en værdi som ``evalueres'' for at gøre noget.
Vi kan styre denne evaluering og hvornår den sker hvilket er et væsentligt
koncept i lisp, og som bruges af av-systemet.

Der findes flere slags værdier, disse ``slags'' kaldes en type.
I lisp har heltal typen \code{integer} og kunne f.eks. være
13, -17, 0, 1074234. Bemærk at for effektivitet er heltal begrænset til en
endelig mængde det betyder at vi ikke kan repræsentere tal størrere end
2147483647, eller mindre end -2147483647. Denne begrænsning er typisk ikke et
problem i de programmer vi laver her.

Et kommatal af typen \code{float} er en slags reelt tal. 9.17, 1.0, -14.30421,
0.1 er alle \texttt{float}s. Vær opmærksom på at der er en begrænset præcision,
tallet vil derfor altid være afrundet til en værdi der kan repræsenteres internt
i computeren. Du kan derfor ikke gemme \(\pi\) med alle decimaler, men må nøjes
med en mindre størrelse. Denne afrunding gør at \texttt{float}s tiltider kan
føre til resultater der ikke helt matcher hvad du forventede, det er derfor
typisk en dårlig idé at bruge \texttt{float} hvis du vil have et eksakt resultat
at sammenligne med. Men til meget almindeligt brug er de en fin type.

En \code{string} er en tekststreng. En sekvens af bogstaver, tal og tegn der
bruges som en tekst. Det kunne f.eks. være et filnavn, en tekst der ønskes
skrevet på skærmen, eller noget andet. En streng kunne se således ud
\code{\"{}Hello, World!\"} bemærk de to dobbeltplinger/quotes som bruges til at
angive at indholdet imellem er teksttrengen. Hvis man gerne vil have en sådan
til at forekomme inde i strengen kan man ``escape'' den med et ``backslash''.
\code{\"{}abc\textbackslash{}\"{}def\"}

Et \code{symbol} er et ord og minder måske lidt om en \code{string}, men et
symbol er kun et enkelt ord, og indeholder altså ingen mellemrum. Man laver
typisk et symbol med \code{\lquote} operatoren
\code{\lquote{fisk}}. Symboler er en af grundstenene i lisp, deres
formål er typisk i at når de evalueres returnerer de ikke sig selv, men en anden
værdi. En variabel er således et symbol der når det bliver evalueret resulterer
i en anden værdi.


En liste skrives som \code{(1 2 3)} en start parantes efterfulgt af indholdet
separeret af mellemrum. Lister bliver intern omsat til hægtede lister i form af
såkaldte ``cons'' celler.
En conscelle er en værdi i computerens hukommelse der indeholder to værdier. Et
hovede og en Hale (tænk på en haletudse). Consceller kan bruges til alle mulige
formål, men deres klassiske brug er til lister.
I en liste er konventionen at det første element i listen findes i hovedet på
den første conscelle, halen peger på den anden conscelle. Det andet element
findes i hovedet på den anden conscelle og den tredje conscelle bliver peget på i halen
på den anden. Således er hele listen opbygget af consceller der peger på værdier
og på resten af listen. I den sidste celle er der jo ikke flere celler tilbage,
derfor indeholder halen symbolet \code{nil} for at angive at der ikke er flere
elementer tilbage. En alternativ måde at skrive \code{nil} på er som \code{()}
altßw de tomme parantesser, eller den tomme liste. \code{nil} og \code{()}
bliver behandlet som akkurat det samme, det er bare to måder at skrive det samme.

\note{TODO: lav et diagram}

Lister er ligesom variabler lidt specielle, de evaluerer ikke til sig selv men
til et funktionskald. Koden \code{(+ 1 2)} bliver derfor evalueret som et kald
til funktionen \code{+} med argumenterne \code{1} og \code{2}. Hvis man gerne
vil have en liste kan man istedet bruge \code{list} funktionen\ref{command:list}
eller man kan bruge \lquote{} operatoren som siger at den efterfølgende
operand ikke skal evalueres \code{\lquote{(1 2 3)}}


Når vi i lisp taler om et udtryk taler vi om en sådan værdi, eller mere præcist
noget som når det bliver evalueret resulterer i en værdi.
Således er en liste der indeholder flere lister bare ét udtryk (men opbygget af
flere andre udtryk). Ligeledes er et funktionskald der som argument tager
resultatet af flere andre funktionskald, også ét udtryk med flere udtryk som
argument.

\subsubsection{Sandhed}
De fleste programmeringssprog har et koncept om sandheden af en værdi. Mange
sprog har en ``boolean'' type til at konkretisere værdierne sandt og falsk,
sådan er det dog \textit{ikke} i lisp. I lisp har vi det lidt specielle symbol
\code{nil} som repræsenterer den falske værdi. I lisp er \code{nil} falsk som
den eneste værdi, alle andre værdier betragtes som at være sande.
Husk på at \code{nil} også er den tomme liste, det vil sige at lister med
indhold er sande, mens den tomme liste er falsk.

Om noget er sandt eller ej bruges i betingelser, som \texttt{if}, \texttt{when},
\texttt{unless} og \texttt{while} der alle gør forskellige ting alt efter om et
af deres argumenter evaluere til en sand eller falsk værdi.


\subsubsection{Variabler}
En variabel minder måske lidt om dem du kender fra matematik, men ikke helt.
Variabler bruges til at huske ting, enten midlertidigt mens vi arbejder med
noget, eller for at huske noget på længere sigt.
En variabel er et navn for en værdi, så når man bruger variablen bruger man
faktisk den værdi der er bundet til den. En variabel har et navn, et symbol, som
er bundet til en konkret værdi på et givent tidspunkt i programeksekveringen.
Når lisp evaluerer et symbol slår den op i en tabel over alle variabler og
finder ud af hvad variablen er bundet til, så returneres denne værdi.
Lad os sige at variablen \code{alder} er bundet til værdien \code{42}, så kan vi
skrive \code{alder} i koden og dette vil returnere værdien \code{42} når det
bliver evalueret.
Hvis du vil vide hvor gammel man bliver om et år kan man så f.eks. skrive
\code{(+ alder 1)} når lisp evaluere dette vil det først evaluere symbolet
\code{alder} til værdien \code{42}, så evalueredet heltalet \code{1}, dette
evaluere til sig selv, altså \code{1}, så evalueres funktionen \code{+} med
argumenterne 42 og 1, dette kald returnere så resultatet 43 som den endelige
værdi.

Så hvis vi skriver \code{x} evalueres dette af lisp til den værdi \code{x} er
bundet til. Hvis vi bare gerne vil have symbolet \code{x}, skal vi istedet
skrive \code{\lquote{x}}

For at tildele en værdi til en variabel kan man bruge \code{setq} som i
\code{(setq alder 44)} dette sætter variablen med navnet \code{alder} til
værdien 44. \texttt{setq} står for ``set quoted'' dette er en makro der ikke
evaluerer navnet men tager det uevalueret for så at sætte den tildelte værdi i
tabellen over alle variabler. I dette tilfælde evalueres \code{alder} altså ikke
til den tidligere værdi, og fordi \code{setq} er en makro behøver vi ikke at
skrive \code{\lquote{alder}}. Der findes også en anden funktion \code{set} som
ikke er en makro, og den \textit{skal} kaldes med et symbol som i \code{(set
  \lquote{alder} 45)} denne funktion bruges dog ganske sjældent i normal kode.

\subsubsection{Funktioner}
En funktion er noget kode der kan eksekveres (udføres) når funktionen bliver
kaldt. En funktion minder på en måde om funktioner som du måske kan huske fra
matematik \(f(x) = 3x^2 + 7x -4\), her er det beskrevet hvordan funktionen \(f\)
afhænger af parametret \(x\) og hvordan resultatet bliver udregnet, vi kan nu
udregne at \(f(8)\) giver værdien 244. Hvis vi gerne ville formulere denne
udregning som en funktion i lisp kunne vi skrive:
\begin{verbatim}
(defun f (x) (+ (* 3 x x) (* 7 x) -4))
\end{verbatim}
Og vi kan som før udregne at \code{(f 8)} giver resultatet 244. Bemærk hvordan
funktionsnavnet står som det første inde i parantesen.
En funktion er således en måde at beskrive hvordan noget udregnes, muligvis
afhængigt af nogle parametre. En funktion er dog lidt mere generel, den kan
nemlig også have ``sideeffekter''. I virkeligheden ville det måske være mere
korrekt at kalde dem procedurer end funktioner, det tager vi dog ikke så tungt
her og holder os til den mere gængse terminologi. Det betyder dog at en funktion
kan mere end blot at beskrive hvordan man udregner et resultat, de kan faktisk
bruges til at beskrive vilkårlige sekvenser af hvordan ting skal gøres.
Det kunne f.eks. være at tegne et landskab ud fra nogle billeder af træer,
en himmel, et hus og en mark.

Når vi kalder en funktion i lisp kan vi give nogle argumenter med, f.eks. gav
vi argumentet 8 til funktionen \code{f}. Disse evalueres og gemmes så i
parametret som en variabel, herefter udføres kroppen af funktionen med
parametrene. I ovenstående eksempel er \code{x} derfor en variabel som er bundet
til 8 når vi kalder funktionen med 8 som argument.

Så kan lisp gå i gang med at evaluere kroppen. I dette tilfælde består kroppen
af et kald til funktionen \code{+}, der gives her 3 argumenter til denne
funktion, først evalueres første argument, det er et kald til funktionen
\code{*}. Vi giver 3 argumenter til \code{*}, først tallet 3 som evaluere til
sig selv, herefter \code{x}, \code{x} er en variabel og evaluere til det den er
bundet til, i det her tilfælde 8, herefter evalueres \code{x} igen, dette udfra
at \(x \times x = x^2\). Nu udregner lisp kroppen af multiplikationen der giver
resultatet 192, dette er altså det første argument til \code{+}, nu udregnes
andet argument \code{(* 7 x)} som ligeledes er et kald til \code{*}, 7 evaluere
til 7, \code{x} evaluerer til 8, og produktet er 56, som det andet argument til
\code{+}, det tredje argument er tallet \(-4\) som evaluere til sig selv. Nu har
lisp altså udregnet de tre argumenter til \code{+} funktionen som altså svarer
til \code{(+ 192 56 -4)} dette kan lisp så udregne og det giver 244. Nu kan lisp
så se at der ikke er mere at lave i funktionen \code{f}, derfor returneres den
seneste værdi som resultatet af hele funktionen, \code{f} returnere således 244
tilbage til der hvor den blev kaldt.
\marginnote{\code{+} er en lidt speciel funktion, den kan tage vilkårligt mange
  argumenter som den summerer, ligeledes kan \code{*} bruges til at udregne
  produktet af alle sine argumenter}

Det kan ses som meget arbejde, men det hele gøres automatisk af lisp når vi
eksekverer koden, et program er således opbygget af en masse kald til
forskellige funktioner der typisk kalder andre funktioner indtil alle værdierne
er udregnet. Det kan virke lidt uoverskueligt i starten, men denne måde at bygge
funktioner op på er ret central i lisp (og de fleste andre programmeringssprog),
det tilader os at opbygge et lag af abstraktioner, vi kan nu snakke om at
udregne \(f\) frem for at snakke om multiplikation og addition, eller det at
tegne et landskab frem for at tegne enkelte dele.


\note{\textbf{Hvad er forskellen på et argument og et parameter?}
  parametret er navnet på variablen/variablerne som funktionen tager dem, det er
  altså de navne værdierne bliver bundet til. Når vi kalder en funktion giver vi
  den nogle værdier, disse kaldes argumenter. Argumenter bruges altså når vi
  kalder funktioner, parametre bruges i funktionens definition.

  I praksis er det dog ikke specielt vigtigt hvad du kalder dem, mange bruger de
  to udtryk uden at tænke over hvad det præcis er, og ingen vil tænke synderligt
  over det om du siger argument eller parameter.
}

Måden vi erklærer en funktion er med \code{defun}, dette er en makro som
erklærer en funktion, hvilke parametre den har og hvad den gør når den bliver
kaldt. Det første argument til \code{defun} er funktionens navn, dette skal vi
bruge hver gang vi fremover vil kalde funktionen. Det andet argument er en liste
af parametrene funktionen har, som udgangspunkt skal en funktion kaldes med et
argument til hvert parameter funktionen har, man skal derfor give præcis det
samme antal argumenter som funktionen har parametre.
Efter listen af parametre kommer der potentielt en såkaldt ``docstring'', dette
er en kort dokumentation af hvad funktionen gør. Docstringen er valgfri, til sin
korte funktion der skal tegne et landskab til revyen behøver man den næppe, men langt de fleste
funktioner i Emacs har f.eks. en sådan dokumentation.

Herefter kommer selve kroppen. Dette er en vilkårlig lang sekvens af udtryk, når
funktionen evalueres, evalueres disse udtryk et ad gangen. resultatet af
evalueringen af det sidste udtryk bliver returneret som resultatet af hele
funktionskaldet.

Udover de obligatoriske ``normale'' parametre kan man også have valgfrie
argumenter. Til dette bruger man det lidt magiske ord \code{\&optional} i sin
sekvens af parametre, efter dette er de efterfølgende parametre valgfrie.
Det betyder at når man kalder koden skal man angive alle de obligatoriske
parametre, og man kan, men behøver ikke at angive de valgfrie argumenter.
det kunne f.eks. se således ud:
\begin{verbatim}
(defun foo (a b &optional c d)
  (list a b c d))
\end{verbatim}
Funktionen med det intetsigende navn \textit{foo} skal kaldes med to argumenter,
men man kan kalde den med ialt 2, 3 eller 4 argumenter. De argumenter man kalder
funktionen med bliver gemt i parametrene i rækkefølge. De parametre der ikke
bliver givet som argument bliver tildelt værdien \code{nil}, altså den
intetsigende falske værdi. Dette er så måden at se på om der ikke blev givet et
argument.

Vi kan f.eks. kalde \code{(foo 1 2 3 4)} og \code{list} kaldet i \code{foo} vil
lave lave en liste ud af argumenterne og resultatet vil da være \code{(1 2 3
  4)}. Hvis vi kalder \code{(foo 1 2 3)} er resultatet \code{(1 2 3 nil)}, hvis
vi kalder \code{(foo 1 2)} er resultatet \code{(1 2 nil nil)}.

Når vi ikke giver et argument med er parametret således sat til \code{nil}, vi kan bruge dette
til at lave en anderledes opførsel for en funktion:
\begin{verbatim}
(defun bar (&optional a)
  (if a
      "Jeg fik en værdi"
    "Jeg fik ikke en værdi")
\end{verbatim}
Her har vi en betingelse, (mere om dem i \ref{subsubsec:betingelser}) hvis
\code{a} \textit{ikke} er \code{nil} så returneres strengen \code{\lstring{Jeg
    fik en værdi}} ellers, hvis \code{a} er \code{nil}, returneres
\code{\lstring{Jeg fik ikke en værdi}}.
Så hvis vi kalder \code{(bar)} returnerer den \code{\lstring{Jeg fik ikke en værdi}}
og hvis vi kalder \code{(bar 10)} returnerer den  \code{\lstring{Jeg fik en
    værdi}},

Der er dog en ting vi skal være opmærksomme på, vi kan ikke skelne om et
argument ikke blev givet, eller om \code{nil} værdien blev givet som argument.
Hvis vi kalder \code{(bar nil)} vil resultatet derfor være \code{\lstring{Jeg
    fik ikke en værdi}}
Dette kan være lidt overraskende da man jo faktisk gav en værdi med. Man kan
vælge at se det som at hvis man kalder en funktion med \code{nil} som argument
til et valgfrit parameter, så siger man at man explicit bare vil have at den
skal gøre det som den gør som standard uden et argument. Det kan være brugbart
for en funktion der tager flere forskellige valgfrie parametre og har forskellig
opførsel alt efter værdien af disse. Så kan man f.eks. give \code{nil} som værdi
for at få den til at behandle nogle argumenter som om vi ikke gav dem, samtidigt
med at vi giver nogle af de senere explicit. F.eks. kunne vi kalde \code{(foo 1
  2 nil 4)} som ville returnere \code{(1 2 nil 4)}, nu er dette et ret simpelt
tilfælde, men det er praktisk for nogle af de mere komplicerede funktioner i
Emacs.


En anden type valgfrie parametre er ``resten'' parametre, disse kan bruges til
at sige at en funktion kan tage vilkårligt mange argumenter. De bliver så givet
som et rest parameter i en liste indeholdende dem alle.
Man bruger også her det lidt magiske ord \code{\&rest} foran et enkelt parameter
navn som alle de resterende argumenter bliver gemt i.

Vi kunne dermed oprette vores egen udgave af \code{list} funktionen vi så
tidligere:
\begin{verbatim}
(defun my-list (&rest l)
  l)
\end{verbatim}
Denne funktion har ét rest parameter kaldet \code{l}, som alle argumenterne til
\code{my-list} bliver lagt i. Funktionen gør ikke andet en at returnere denne
liste. Vi kan så kalde \code{(my-list 1 2 3)} og vi får \code{(1 2 3)} som
resultat, eller vi kan kalde \code{(my-list 1 2 "a" 3 "c" 4)} og vi får \code{(1
  2 "a" 3 "c" 4)} tilbage som resultat. Hvis vi kalder \code{(my-list)} får vi
den tomme liste tilbage \code{nil} (som er det samme som \code{()}).

Vi kan også kombinere dette med obligatoriske argumenter
\begin{verbatim}
(defun my-list-atleast-2 (a b &rest l)
  (cons a (cons b l)))
\end{verbatim}
Denne funktion sætter de to første argumenter sammen i et par consceller der
peger på hinanden og videre på resten af argumenterne i \code{l}, således
konstrueres der én lang liste. Resultatet er altså lig det fra \code{my-list},
forskellen er at denne funktion fejler hvis ikke den bliver kaldt med 2 eller
flere argumenter.

Vi kan også have valgfrie parametre:
\begin{verbatim}
(defun my-optional (a b &optional c &rest l)
  (cons a (cons b (cons c l))))
\end{verbatim}
Denne lidt mærkelige funktion \textit{skal} have mindst 2 argumenter, ellers
fejler den, man \textit{kan} vælge at give den et tredje, ellers indsættes et
\code{nil} i listen, og man kan give den vilkårligt flere der så indsættes i
enden. \code{(my-optional 1 2 3 4 5)} giver så \code{(1 2 3 4 5)}
\code{(my-optional 1 2)} giver \code{(1 2 nil)} og hvis vi kalder
\code{my-optional} med færrere end 2 argumenter resulterer det i en fejl.

Det giver \textbf{ikke} mening at have flere parametre efter \code{\&rest}
nøgleordet, men man kan navngive det som man vil.

\note{Vær opmærksom på at først skal du angive \textbf{ALLE} obligatoriske
  parametre, så skal du angive \textbf{ALLE} de valgfrie, og herefter kan du
  angive ``resten'', du kan altså ikke blande rundt i rækkefølgen}


En sidste ting omkring funktioner, er at man skal være opmærksom på at de først
kan bruges \textit{efter} de er defineret. Og at det altid er den seneste
definerede udgave der kaldes. Hvis man omdefinerer en funktion behøver man altså
ikke at gøre noget ved de funktioner der allerede er defineret og som kalder
funktionen, de vil automatisk kalde den nyest definerede udgave.

\subsubsection{Makroer}
En makro minder på mange måder om en funktion, en central forskel er dog
\subsubsection{Fejl}
elisp undtagelser, wlisp error
ikke definerede variabler giver fejl.

\subsubsection{Lokale variabler}
Der findes forskellige slags variabler. Dette har noget at gøre med hvor og hvor
længe de er synlige/eksisterer.
En normal variabel hvis vi bare skriver \code{(setq foo 10)} vil være en global
variabel. Denne er synlig i resten af programmet af alt kode. Vi kan således
skrive \code{foo} for at få den tildelte værdi. Hvis vi nu efterfølgende skriver
\code{(setq foo 11)} vil vi opdatere værdien for foo til 11, vi kan således ikke
længere se den gamle værdi 10, men istedet den nye værdi 11.

Til tider kan det være praktisk med en eller flere midlertidige variabler, til
brug mens man udregner noget, men som ikke har noget formål at gemme.
Så kan vi bruge \code{let} eller \code{let*}. Disse introducere nogle
\textit{nye} variabler der kun eksisterer inde i kroppen af let udtrykket.
Hvis vi bruger \verb+;=>+ som en syntaks i vores kode eksempler til at betyde
``evaluere til'' så gælder der da at
\begin{verbatim}
(setq foo 10)
foo ;=> 10
(let ((foo 100))
  foo ;=> 100
  (setq foo 101)
  foo ;=> 101
)
foo ;=> 10
\end{verbatim}
Så \code{foo} sættes til at være værdien 10. I let udtrykket defineres så den
nye variabel der også hedder foo, denne har værdien 100. Den originale variabel
findes stadig, men når vi nu skriver \code{foo} henviser vi altså til den nye
udgave og ikke den gamle. Dette gælder også hvis vi ændrer i værdien, dette
gøres for den nuværende variabel, og ikke den gamle.
Efter let udtrykket henviser \code{foo} altså igen (eller stadig) på den gamle variabel 10.

Denne måde at lave en midlertidig binding er utrolig praktisk, på den måde
undgår vi at overskrive anden data som vi faktisk gerne ville beholde
Det er især praktisk i forbindelse med funktioner. I en funktion vil vi gerne
undgå at et kald har en unødig indvirkning på anden kode. Ved at bruge et let
udtryk ungår vi dette.
Her er et par eksempler, disse giver nok mere mening når du om lidt har læst afsnittet
om funktioner.

\begin{verbatim}
(defun foo ()
   (setq a 1)
   (bar)
   a)
(defun bar ()
  (setq a 2)
  a)
(foo) ;=> 2
\end{verbatim}
Her henviser \code{a} til den samme globale variabel. Derfor returneres værdien 2.

\begin{verbatim}
(defun foo ()
   (let ((a 1))
     (bar)
    a))
(defun bar ()
  (setq a 2)
  a)
(foo) ;=> 1
\end{verbatim}
Foo erklærer en ny variabel \code{a}, men på grund af ``dynamisk scope'' er det
denne variabel som \code{bar} ser og opdatere. Det endelige resultat er derfor
det samme, men efter kaldet til \code{foo} vil den globale værdi af \code{a}, hvis den
findes være den samme som tidligere.

\begin{verbatim}
(defun foo ()
   (let ((a 1))
     (bar)
    a))
(defun bar ()
  (let ((a 2))
    a))
(foo) ;=> 1
\end{verbatim}
Dette er den korrekte måde at undgå at variabler i andre funktionskald ændres.
\code{foo} erklærer en ny variabel \code{a}, det samme gør \code{bar}, i
\code{bar} vil alle ændringerne af \code{a} inde i let udtrykket derfor ikke
være synlige i \code{foo}.


En anden ting der er omkring variabler er at parametrene til en funktion
effektivt er midlertidige variabler defineret og synlige i funktionskroppen og
bundet til værdien af det tilsvarende argument.
\begin{verbatim}
(defun foo (a)
  (setq a 1)
  (bar a)
  a)
(defun bar (a)
  (setq a 2)
  a)
(foo 0) ;=> 1
\end{verbatim}

Noget om component lokale variabler
Noget om ikke definerede variabler
\subsubsection{...}
\note{
Vectorer er ikke ens.
I Emacs lisp er de konstante. I Worker Lisp er de dynamiske arrays. ak indsætte
foran og bagi i O(1) tid (ammortiseret)

Lister er en slags ``meta'' type over consceller. En konvention!
En liste er en Cons-celle hvor \texttt{car} indeholder et element i listen, og
\texttt{cdr} indeholder en liste (resten).
Alternativt er en liste \texttt{nil} som angiver den tomme liste eller
slutningen på en liste.
Der er således ikke en konkret type i sproget der hedder \textit{list}, men man
bruger det som om der var.

Quote opfører sig anderledes
(Skal jeg lade være med at deep copy'e?)
}
\subsection{Hvordan programmerer man?}
\subsubsection{Betingelser}
\label{subsubsec:betingelser}
\subsubsection{Løkker}
\subsubsection{Hvordan arbejder man med lister}
\subsubsection{Om at finde/rette fejl}

\subsection{Lisp på master}
\label{subsec:master-commands}
Dette er en liste af kommandoer der kan bruges på masteren. En del af disse kan
også bruges interaktivt

\subsubsection{Lokale kommandoer}
Kommandoer der gør ting lokalt på masteren.


\documentcommand{revy-start}
Starter en revy, den første kommando i en .revy fil.
Forbinder til alle arbejderne og starter programmet.
Hvis den fejler er det typisk fordi en server ikke har forbindelse til
netværket (måske er den ikke tændt), fordi masteren ikke har forbindelse til
netværket (har du sat stikket i / fået en statisk IP?) eller fordi programmet er
installeret i en anden mappe end den angivne.
\note{Der er lige nu en fejl i programmet der gør at denne kommando nogle gange
  skal køres to gange for at virke}


\documentcommand{revy-quit}
Lukker programmet på alle arbejderne hårdt og brutalt.
Den inverse funktion af revy-start.
Bør kun bruges fordi tingene virkeligt er gået i hårdknude, f.eks. hvis en
arbejder stopper med at svare når man prøver at eksekvere kode på den.

\documentcommand{revy-open filename \optional{} worker}
Åbner en fil i revysystemet og eksekverer den i dens revysammenhæng.
Se de specifikke revy modes\ref{subsec:ubersicht}, \ref{subsec:ubertex} for en
bedre indføring i de enkelte dele.
Når en fil åbnes på denne måde, vises den i en buffer i det korrekte mode og
eksekveringen starter fra toppen.
Rækkefølgen af filer der åbnes, og deres placering huskes, så et kald til
\code{revy-end-sketch} fra den åbnede fil returnerer til her.

\documentargument{filename} er en tekststreng med navnet på filen. Umiddelbart bruges en
relativ sti \code{(revy-open "sange/lalala.tex")}, og denne sti tager så
udgangspunkt i mappen hvor revyen er gemt (på din lokale maskine).
Alternativt kan man bruge en absolut sti hvis man f.eks. pludselig fik lyst til
at åbne en fil fra en gammel revy.

\documentargument{worker} er navnet på en arbejder, denne vil blive brugt som ``default
worker'' istedet for den nuværende. Så kan man f.eks. åbne en tex/pdf på en
anden arbejder end den nuværende

\code{(revy-open "sange/lalala.tex"\ \lquote{mid})}

\documentargument{revy-nop \rest ...}
Gør ingenting, og ignorerer alle argumenter.
Kan f.eks. bruges til at udkommentere en kommando.
Kan også bruges i .revy filen istedetfor \code{revy-open} for at signalere at
dette nummer ikke indeholder noget AV.
Imodsætning til en kommentar eksekveres denne instruktion stadig, den udfører bare ingenting. Dette kan
bruges hvis f.eks. er en sketch uden AV-indhold, man kan på denne måde stadig
følge med i hvor man er i revyen ved blot at trykke næste.


\begin{verbatim}
(revy-open "sketches/den-sjove.sketch")
(revy-nop "sketches/den-kedelige.sketch")
(revy-open "sange/sjov-sang.tex")
\end{verbatim}
Den kedelige sketch markeres stadig når man trykker videre fra den sjove sketch,
men der sker ikke noget, så ved næste tryk bevæger man sig videre til den sjove
sang.

\documentcommand{revy-end-sketch}
Lukker et nummer.
Kan bruges i alle slags filer, både .tex og .sketch.
Returnerer til den fil der har åbnet den nuværende og fortsætter fra hvor
cursoren stod sidst.
Bruges typisk til at komme tilbage til .revy oversigten fra en sang eller
sketch.

Funktionen kalder \code{revy-abort-all}, og lukker dermed allt kørende kode på arbejderne
automatisk.

\documentcommand{revy-return}
Virker lidt lige som \code{revy-end-sketch}, returnerer til det sted der åbnede
det nuværende nummer. Men lukker ikke kørende kode. Den kører altså ikke
\code{revy-abort-all}.

\documentcommand{revy-restart}
Genstarter det nuværende nummer. Svarer til at lukke og åbne filen.

\subsubsection{Generelle kommandoer}
\documentcommand{revy-abort-all}
Den store røde stop knap! Stopper alt der kører på alle arbejderne. Alle lyde,
billeder, pdf'er og kode der kører afbrydes.
Er praktisk at have i baghånden til når tingene går galt, og kan bruges
interaktivt til dette. Men kan også bruges inde i en sketch til at stoppe alt
der køres samtidigt.
Vær opmærksom på at lyde stoppes ret abrupt, hvilket ikke altid er hvad der
ønskes.
Bør ikke bruges som panik knap hvis f.eks. en sanger synger forkert, i så fald
stoppes hele pdf'en med at blive vist. Så er man nødt til at genåbne filen,
finde tilbage igen, og fortsætte. Til dette formål bør man istedet bruge
``blank'' funktionerne.

\documentcommand{revy-abort}
Lige som \code{revy-abort-all}, men stopper kun de ting der kører på den
nuværende arbejder.
Så den kan f.eks. bruges i en sketch til at holde højtex kørende mens man
stopper et billede og en lyd på overtex.

\documentcommand{revy-blank)}
\label{command:revy-blank}
Stopper visningen af grafik på den nuværende arbejder indtil næste input.
Alt kode og lyd fortsætter med at køre, men skærmen ``cleares'' inden den vises.
I praksis svarer det til at der tegnes en uendelig stor sort firkant oven på
alting.
Blankningen forbliver indtil der modtages en form for kode eller signal.
I praksis indtil du gøre noget der skulle gøre noget på arbejderen.

Bruges typisk interaktivt til at stoppe visningen af sangtekster når sangeren
synger forkert. Så blanker AVmanden, finder frem til hvor sangeren er nået til,
og fortsætter så. Når AVmanden fortsætter holder blankningen op og teksten vises
som normalt.

\documentcommand{revy-blank-all}
Som \code{revy-blank}, men blanker alle arbejdere.
Vær opmærksom på at arbejderne venter enkeltvist på at få et signal om ikke
længere at være blanke.

Bruges ikke så ofte da formålet med at blanke typisk kun er at fjerne det der er
forkert, sangerens tekst, og ikke alt hvad der også er på de andre arbejdere.

\documentcommand{revy-unblank-all}
Sender signal til alle arbejdere om at de ikke længere skal være blanke.
Bruges typisk efter et \code{revy-blank-all}

\documentcommand{revy-calibrate}
Viser et kalibrerings billede.
Er en simpel måde at sikre sig at alle arbejdere virker, er tændt og deres
projektorer står indstillet korrekt.

\subsubsection{Audio Visuelle kommandoer}
\documentcommand{revy-image file \optional{} position}
Sender et program til den nuværende arbejder der viser et billede.
Den indbyggede \code{image} funktion på arbejderne tegner et billede på en
position en enkelt gang i et enkelt frame.
For at få vist et billede i længere tid kan man bruge denne funktion der sender
en stump kode der definerer en billede viser der kontinuerligt viser billedet.
Den bliver ved med at tegne billedet i hvert frame indtil den afbrydes.
\code{revy-image} sætter bare det nuværende kode til at være denne billede
viser.

Se \ref{command:update} for hvordan det præcist virker med at afvikle kode i
hvert frame.

\begin{verbatim}
(revy-image "billeder/hund.png") ;; Viser et billede af en hund.
(revy-image "billeder/kat.png")  ;; Viser et billede af en kat istedet.
\end{verbatim}

``Position'' argumentet er lidt magisk og kan en hel masse
\note{TODO: Forklar noget om dette}

\documentcommand{revy-image-preload \rest{} files}
Bruges ikke til noget.


\documentcommand{revy-pdf-open file}
Åbner en pdf fil og viser slide 0 (det første).
Til sangtekster vil man typisk bruge ubertex\ref{subsec:ubertex} som også
sagtens kan bruges til at lave andre slideshows en bare overtekster.
Til billeder og diasshow er det ofte at foretrække at bruge enkelte billeder da
de nemmere kan justeres enkeltvist med ``position'' argumentet for hvert enkelt
slide.
Men ind imellem får man som AV-mand en pdf med billeder der skal vises.
Til dette kan man bruge \code{revy-pdf-open} som definerer en pdf-viser meget
lig den førnævnte billedeviser.

Vær opmærksom på at det kan tage lidt tid at åbne en stor pdf.

\note{Vær opmærksom på at revysystemet 0-indeksere slides i pdf'er.
  en pdf på 3 sider vil da indeholde slide 0, 1 og 2.}

\documentcommand{revy-pdf-goto-slide slide}
Skifter hvilket slide den \textit{nuværende} pdf-viser viser. ``slide'' er et
0-indekseret tal der angiver hvilket slide der skal vises.

\note{Hvad sker der hvis tallet er størrere end antallet af slides?}

\documentcommand{revy-pdf-next}
Skifter slidet den \textit{nuværende} pdf-viser viser til det næste

Bør generelt ikke bruges i en sketch da man ikke længere kan gå til en vilkårlig
position og fortsætte.
Hvis en skuespiller f.eks. hopper fra slide 5 til slide 8 i sit diasshow, kan
man ikke hoppe med hvis man kun er sat op til at bruge \code{revy-pdf-next}.
Hvis man derimod bruger \code{revy-pdf-goto-slide} kan man rykke direkte ned til
slide 8 og vise dette.

\note{Hvad sker der hvis det nuværende slide er det sidste?}

\documentcommand{revy-sound file \optional{} volume}
Afspiller en lyd.
Lyden bliver afspillet som en sepperat komponent, så den har ikke indvirkning på
anden afviklet kode/lyde.
Man kan således bruge revy-sound flere gange til at afspille flere lyde på samme
tid.
Man kan angive en ønsket volume, et heltal mellem 0 (min) og 128 (maks).
Hvis ikke lydstyrken angives spilles den på maks.

\note{Vær opmærksom på at den lydmixer der bruges lige nu er ret dårlig og
  forringer lydkvaliteten en hel del. Til forestillingen er dette dog ikke det
  store problem på grund af andet lyd/larm i salen.

  Hvis man virkelig ønsker sig at en bedre lydkvalitet, f.eks. ved musikken til
  et dansenummer kan man istedet for \code{revy-sound} bruge \code{revy-mplayer}
  der udover videoer også kan afspille musik. Så mister man dog en del
  fleksibilitet til at stoppe/fade lyde.}

\documentcommand{revy-stop-sounds}
Stopper alle lyde der afspilles på den nuværende arbejder. Virker lidt som \code{revy-abort},
men kun på lyde.

\documentcommand{revy-fade-sounds \optional{} duration}
Virker lidt som \code{revy-stop-sounds}, men istedet for at stoppe lydene brat,
så fades de langsomt ud.
Man kan angive en varighed i sekunder, som standard tager det 4 sekunder at fade
ud.

Er et godt alternativ til \code{revy-stop-sounds} i mange situationer da det
ofte lyder bedre.

Hvis man kalder den interaktivt spørger den efter en varighed.

\documentcommand{revy-mplayer file \optional{} x y w h}
Afspiller en video. Kører det eksterne program ``mplayer'' på den nuværende
arbejder som afspiller videoen.
Kan også afspille lydfiler.

\note{Hvordan indstiller man mplayer?}

\documentcommand{revy-kill-mplayer}
Lukker alle instanser af ``mplayer'' på den nuværende arbejder.
Ofte kan man bruge revy-abort istedet.

\documentcommand{revy-text \optional{} text}
Viser en tekst midt på skærmen af den nuværende arbejder.

Kan kaldes interaktivt, hvor den spørge om en tekst der så vises. Man kan bruge
piletasterne til at finde tidligere tekster.
\note{Snak om Markup sproget}


\subsubsection{Interaktive kommandoer}
\documentcommand{revy-create}
Opretter en ny revy. Fører dig igennem en længere ``wizard'' (opsætnings guide).
\note{Forklaring om hvordan denne virker}

\documentcommand{revy-load}
Loader en allerede oprettet revy.
Spørger dig om du vil indlæse en af de revyer den kender til, typisk den du
arbejder på.
Ellers kan du vælge ``other revy...'' hvorefter du kan angive den præcise
placering for en revy.

\documentcommand{revy-mode-enter}
\documentcommand{revy-mode-next}
\documentcommand{revy-mode-point-forward}
\documentcommand{revy-mode-point-backward}


\subsubsection{Andre}
\documentcommand{revy-on-worker worker function \rest{} args}
\documentcommand{revy-create-workers}
\documentcommand{revy-send-lisp}
\documentcommand{revy-send-command}
\documentcommand{revy-shell}
\documentcommand{revy-shell-sync}
\documentcommand{revy-shell-local}
\documentcommand{revy-shell-local-sync}
\documentcommand{revy-elisp}
\documentcommand{revy-upload-files}
\documentcommand{revy-upload-files-sync}
\documentcommand{revy-build}
\documentcommand{revy-compile-tex}

\subsection{Lisp på worker}
\label{subsec:worker-commands}

\documentcommand{update}
\label{command:update}
Du skal forstå dette her hvis du vil lave noget som helst avanceret...
Så er det ærgeligt at dokumentationen er mangelfuld, beklager...


\documentcommand{progn}
\documentcommand{quote}
\documentcommand{eval}
\documentcommand{list}
\label{command:list}
\documentcommand{cons}
\documentcommand{car}
\documentcommand{hd}
\documentcommand{head}
\documentcommand{cdr}
\documentcommand{tl}
\documentcommand{tail}
\documentcommand{if}
\documentcommand{when}
\documentcommand{unless}
\documentcommand{while}
\documentcommand{and}
\documentcommand{or}
\documentcommand{print}
\documentcommand{set}
\documentcommand{setq}
\documentcommand{let}
\documentcommand{let*}
\documentcommand{eq}
\documentcommand{equal}
\documentcommand{not}
\documentcommand{defun}
\documentcommand{lambda}
\documentcommand{+}
\documentcommand{-}
\documentcommand{*}
\documentcommand{>}
\documentcommand{sin}
\documentcommand{cos}
\documentcommand{randint}
\documentcommand{color}
\documentcommand{clear-color}
\documentcommand{clear}
\documentcommand{fill}
\documentcommand{image}
\documentcommand{pdf}
\documentcommand{text}
\documentcommand{calibrate}
\documentcommand{sound}
\documentcommand{sound-stop}
\documentcommand{sound-stop-all}
\documentcommand{sound-fade-all}
\documentcommand{defcomp}
\documentcommand{defcomponent}
\documentcommand{create}
\documentcommand{destroy}
\documentcommand{current-layer}
\documentcommand{deflocal}
\documentcommand{update}
\documentcommand{render}
\documentcommand{send}
\documentcommand{broadcast}
\documentcommand{receive}


magiske
\documentcommand{component\_update\_all}
\documentcommand{message\_dispatch}
\documentcommand{resource\_cache\_size}
\documentcommand{sounds\_playing}
\documentcommand{allocate\_useless}
\documentcommand{render\_test}
\documentcommand{pdf\_test}
\documentcommand{sdl\_internals}
\documentcommand{resource\_usage}
\documentcommand{exit\_program}
\documentcommand{set\_window\_position}


\newpage
\section{- Værktøjer}
\subsection{Schneider}
\textit{Tysk: Skrædder}

\textbf{Dependencies:}
\begin{itemize}
\item Python3
\item ffmpeg
\end{itemize}

Schneider er et værktøj til at skære film og billeder op til at kunne blive vist over flere
projektorer.

Schneider kan på skrivende stund kun dele medier op i flere snit ved siden af
hinanden og ikke over under. Det burde dog være let at rette til.

\subsection{Zeitherr}
\textit{Tysk: Timelord}
Er en bastardiseret udgave af en NTP tidsserver til at holde de forskellige
maskiner synkroniseret.

\note{Externe:}
\subsection{xpdf}
Overtekster køres i xpdf der åbnes manuelt med:
\begin{verbatim}
xpdf -remote ubertex -fullscreen -mattecolor black -fg black
    -bg black -papercolor black filnavn
\end{verbatim}
Da dette er meget langt kan man istedet bruge aliaset
\begin{verbatim}
p filnavn
\end{verbatim}


\subsection{mplayer}
Til at vise videoer manuelt bruges mplayer:
\begin{verbatim}
mplayer -nolirc -msglevel all=-1 -msglevel statusline=5
    -vo gl2 -autosync 30 -cache 1048576
    -cache-min 99:100 -xy 500 -geometry 49%:40% filnavn
\end{verbatim}
eller aliaset
\begin{verbatim}
m filnavn
\end{verbatim}

\subsubsection{Positionering af mplayer}
\textit{TODO: Noget om positionering af mplayer her}

\note{Konfigurer mplayer fra Emacs, væk på sigt}

\subsection{Gimp}
\subsection{Inkscape?}
\subsection{Audacity}

\newpage
\section{- FAQ}
Ubertex tager ikke højde for pauser i comments, latex gør.

\subsection{\LaTeX{}}
\subsubsection{Hvordan indsætter jeg \"{} (quotes)}
Hvis Emacs indsætter \`{}\`{} hver gang du faktisk trykker \"{} så er det fordi
Emacs prøver at være smart og indsætte \LaTeX{}s pæne quotes automatisk. Dette
er selvfølgelig irriterende hvis du ønsker at indsætte \"{} til brug i koden,
f.eks. i indsat elisp. Hvis du blot trykker \"{} en ekstra gang laves det dog om
til det korrekte tegn.

\subsubsection{File `overtex.sty' not found.}
\texttt{overtex.sty} er en fil hvori de overtex speciffike kommandoer er
defineret. \texttt{revy-manus-prepare} indsætter automatisk
\code{\textbackslash{}usepackage\{overtex\}} i overtexfilerne. Når man kalder
\texttt{revy-compile-tex} specificeres automatisk hvor denne fil findes. Kaldes
pdflatex manuelt ved den ikke hvor denne fil findes.
Den simpleste løsning er at oversætte med et kald til \texttt{revy-compile-tex}.

Alternativt kan man enten kopiere \texttt{overtex.sty} ind i mappen hvor \texttt{.tex}
filen ligger, sætte shell variablen \texttt{TEXINPUTS} til at indeholde mappen hvori
\texttt{overtex.sty} ligger eller kopiere filen ind i
\texttt{\~{}/texmf/tex/latex/overtex/overtex.sty} hvor den vil være synlig for oversætteren.

\subsection{- Problemer/løsninger}
\subsubsection{Der er lag/forsinkelser}
Lag er irriterende, det opleves primært som en forsinkelse fra man har trykket
på knappen til der sker noget på projektoren. Det er et problem når man skal
lave overtekster.

Til DIKUs jubilæumsrevy var der ca. et sekunds forsinkelte fra jeg trykkede til
at overteksterne blev vist. Jeg har ikke definitivt fundet fejlen endnu men der
bliver arbejdet på det.

Her er først nogle metoder til at lokalisere hvor omtrentligt forsinkelsen
opstår.
Det er mere eller mindre umuligt at lave konkrete målinger så det er noget man
må føle sig frem til (Super naturvidenskabelig metode!).\\
\textit{Følgende tager udgangspunkt i Xpdf, men gøres på samme måde med Zeigen}

Prøv først at sætte systemet til at køre alt lokalt. Aka, kør både Emacs og Xpdf
lokalt og der kommunikeres via ssh til \texttt{localhost}. Hvis forsinkelsen
stadig er der, er fejlen enten i Emacs, i Xpdf, eller i den lokale
hardware.

Mine erfaringer med Xpdf er dog at det ikke er her fejlen ligger.
Prøv nu at starte Xpdf på serveren og ssh ind på denne, giv nu manuelt Xpdf
ordre til at skifte slide.

Min erfaring er at det er meget tilfældigt hvad der præcist sker. Jeg oplevede
at delayed forsvandt efter jeg gjorde ovenstående, også det mellem Emacs og
Xpdf, men kun indtil Xpdf blev lukket.

Det virker også til at Xpdf bliver langsommer afhængigt af længden af
overteksterne og ikke nødvendigvis størrelsen af pdfen.

Jeg ved ikke helt hvordan dette skal løses.
Det er en af grundene til at vi vil lave vores egen fremviser (Zeigen).


\newpage
\section{- Eksempler}
\note{Med screenshots og eksempler, både på brug af program og effekter
  Se det som hvad jeg ville vise til et AV-føl
  Videosekvens på skrift/billede
}

\newpage
\appendix
\section{\LaTeX}
\LaTeX er et markup sprog der bruges til at designe dokumenter. Man bruger HTML
som du måske kender til at lave hjemmesider, \LaTeX{} derimod bruges til at lave
dokumenter. Det er originalt lavet til at skrive artikler og bøger, især inden
for Datalogi, matematik og fysik, da det har rig mulighed for at opsætte formler
pænt. Mange på NatFak bruger \LaTeX{} til at skrive artikler, aflevering og alt
muligt andet. Den store AV bog som du læser lige nu er f.eks. også skrevet i
\LaTeX{}. Ligeledes bruger en del revyer \LaTeX{} til deres manuskript.
Udover at lave artikler kan man også lave præsentationer ved hjælp af beamer
pakken.
Det er sådan overteksterne kan laves.
Der er en AV-specifik pakke kaldet overtex.sty der kan bruges og som indeholder
makroer Emacs kan forstå og bruge til at fjernstyre backenden.

Hvis du aldrig har arbejdet med \LaTeX{} før kan det godt virke lidt kryptisk og
kompliceret. Du er måske vant til at arbejde med word og power-point som er
såkaldte `What You See Is What You Get'' redskaber (wysiwyg). Det betyder at det
er det sammme program du bruger om du redigerer eller bare kigger i et word
dokument. Du ændrer direkte i det som du ser det.
\LaTeX{} derimod er et programmeringssprog skrevet i en helt simpel tekst fil,
med endelsen .tex det betyder at du ville bruge et program som notepad til at
arbejde med koden, og ikke et program som word. Emacs er et godt bud på et
værktøj til at arbejde med tex filer. At det er et programmeringssprog betyder
at man ikke bare markere et stykke tekst og trykker \textbf{fed} men at man
skriver noget kode der beskriver at det her skal læses som \verb+\textbf{fed}+
det kan tage lidt tid at vende sig til men det er ikke en uoverkomelig opgave.
For så at få en visbar fil, f.eks. en pdf, skal man ``oversætte'' .tex filen.
Dette kan man gøre med \code{pdflatex}. Emacs og især av-systemet har værktøjer
til at gøre dette. Det sker f.eks. når man kalder \code{revy-build} der
oversætter alle .tex filer og overfører dem til arbejderne.

Det kan ske at en .tex fil indeholder fejl, det kan f.eks. være at man har
skrevet en af \LaTeX{}s kommandoer forkert, glemt et special tegn, brugt noget
der ikke findes eller brugt det forkert. Hvis der er en fejl stopper
oversætteren med en fejlbesked. \LaTeX{} har notorisk dårlige fejlbeskeder, og
de er ikke lette at tyde eller at finde automatisk via software.
Hvis du bruger \code{revy-build} vil der blive lavet en buffer kaldet
``*revy-compile-filnavn*'' hvor filnavnet er navnet på texfilen der oversættes.
Du kan kigge i denne buffer for at finde ud af hvad der gik galt.
Det kan også være en idé at kalde \code{revy-manus-clean} som kan finde en del
almindelige fejl.

Her ses et eksempel på en .tex fil:
\begin{verbatim}
\documentclass[14pt]{beamer}
\usepackage[danish]{babel}
\usepackage[utf8]{inputenc}
\usepackage{overtex}

%% Melody: Den der disney sang, https://www.youtube.com/watch?v=dQw4w9WgXcQ

\begin{document}
\obeylines

\begin{overtex}
  % forspil
\end{overtex}

%% S1
\begin{overtex}
  Det her er første linje\pause
  den er lang så vi deler den
\end{overtex}

\begin{overtex}
  % blank
\end{overtex}

\begin{overtex}
  Mere sang\pause{} med en pause\pause
  det er vældigt smart
\end{overtex}
\begin{overtex}
  Flere linjer\pause
  i denne sang
\end{overtex}

\begin{overtex}
  Tralala\pause
  Tralala\pause
  Tralala\pause{} lalala
\end{overtex}

\begin{overtex}
  % slut
\end{overtex}
\begin{overtex}
  \elisp{(revy-end-sketch)}
\end{overtex}
\end{document}
\end{verbatim}

Preamplen, alt teksten inden \verb+\begin{document}+ er fortæller \LaTeX{}
  hvordan indholdet skal forstås.

\% bruges til at lave en kommentar, alt tekst herfra indtil linjens slutning
ignoreres af \LaTeX{}.

\verb+\obeylines+ betyder at \LaTeX{} overholder linjeskiftne. Så et linjeskift
i .tex filen også betyder at der er et linjeskift i pdf'en. Dette er det mest
naturlige i overtekster.

\subsection{AV-specifik}
\begin{verbatim}
\begin{overtex}

\end{overtex}
\end{verbatim}
Definerer et slide, alle linjerne imellem bliver vist sammen.

\verb+\pause+ kan bruges til at dele et slide op i to sider i den endelige pdf.
På den måde vises linjerne inden ``pausen'' først og så de næste linjer
bagefter. Det er almindeligt at have \verb+\pause+ som slutning på de første
linjer i et slide. Men man kan faktisk også have pauser inde i linjerne der så
bliver splittet op. Så anbefales det dog at skrive \verb+\pause{}+ uden de
krøllede paranteser bliver mellemrummet frem til det efterfølgende ord
spist/ignoreret.



\subsection{Generelt brugbare makroer}
Man kan lave forskellige skriftstørrelser.
Følgende kommandoer ændrer størrelsen inde i et scope,

\begin{table}[h!]
  \centering
  \begin{tabular}[h!]{ll}
    \textbackslash{}tiny & {\tiny noget tekst}\\
    \textbackslash{}scriptsize & {\scriptsize noget tekst}\\
    \textbackslash{}footnotesize & {\footnotesize noget tekst}\\
    \textbackslash{}small & {\small noget tekst}\\
    \textbackslash{}normalsize & {\normalsize noget tekst}\\
    \textbackslash{}large & {\large noget tekst}\\
    \textbackslash{}Large & {\Large noget tekst}\\
    \textbackslash{}LARGE & {\LARGE noget tekst}\\
    \textbackslash{}huge & {\huge noget tekst}\\
    \textbackslash{}Huge & {\Huge noget tekst}\\
  \end{tabular}
\end{table}

Et eksempel kunne være følgende taget fra satyr-revy 2016:
\begin{verbatim}
\begin{overtex}
  {\Huge FUCK\pause{} MIT\pause{} LIV!}
\end{overtex}
\end{verbatim}
Som i over tre sider skriver teksten med stort.

\newpage
\section{Sådan bruges ``Simple Emacs''}
\label{sec:simple-emacs}

Emacs kan være svært at bruge.
Buffere, genvejstaster, input, hov noget gik galt.

\note{Copy-pastet fra lisp, nævn Emacs tutorial og Emacs manual.

I Emacs menu, som enten vises i toppen af vinduet som de fleste andre
programmer, alternativt hvis denne menubar er slået fra (som mange gør) kan man
få den frem midlertidigt ved at holde \code{Ctrl} nede og så højreklikke.
I menuen ``Help'' kan du i undermenuen ``More manuals'' også finde elisp
introduktionen samt elisp referencerne.
}

\begin{landscape}
  \newgeometry{top=12cm}
  \thispagestyle{empty}
  {\scriptsize
    \verbatiminput{emacs/help.el}
  }
  \restoregeometry
\end{landscape}

Her er en gennemgang af mange af de genvejstaster der findes i Emacs, specifikt
med den forsimplende konfiguration kaldet Simple Emacs, som findes i
\code{revy-simple}

For at slå den permanent til skal man gå ind i sin \code{.emacs} fil i sin
hjemmemappe og tilføje følgende linjer:

\begin{verbatim}
(add-to-list 'load-path "~/ubertex/emacs")
(require 'revy-simple)
\end{verbatim}

Hvor stien for \code{load}-path peger på \code{emacs} mappen i den mappe hvor \code{ubertex} er installeret.


\documentkey{<f9>}{menu}
Åbner menuen

\documentkey{C-o}{Open file} åben en fil i en ny buffer.

\documentkey{C-s}{Save file} gemmer filen.

\documentkey{C-S-s}{Save all buffers} Gemmer alle ikke gemte buffere.

\documentkey{C-e}{End of line} flytter point til slutningen af linjen.

\documentkey{C-a}{Beginning of line} flytter point til starten af linjen.

\documentkey{C-k}{Kill rest of line} sletter alt tekst fra point til slutningen af
linjen.

\documentkey{C-S-k}{Kill whole line} sletter hele linjen.

\documentkey{H-<backspace>}{Join line} sletter linjeskiftet før denne linje, så
den forige og denne bliver til én lang linje.

\documentkey{C-f}{Find} Find en tekst i bufferen. Når man har trykket tasten,
bliver man spurgt om en tekst, Emacs begynder med det samme at søge ned igennem
bufferen. Søgningen starter fra der hvor point er lige nu. Hvis man vil søge
videre efter næste forkomst kan man trykke \code{C-f} igen. Hvis man når til
slutningen af bufferen vil \code{C-f} starte forfra fra toppen.
Når man er færdig med at søge skal man trykke \code{<enter>} som stopper
søgningen. Man kan trykke \code{C-S-f} for at \textit{reverse} søgeretningen og
søge baglens gennem bufferen. Man kan således bruge \code{C-f} og \code{C-S-f}
til at søge forlæns og baglæns.

\documentkey{C-H-f}{Find reverse} Finder noget baglæns, se \code{C-f}

\documentkey{C-r}{Replace} Spørger først om en tekst der skal erstattes, spørger
så efter en tekst der skal erstatte den. Nu bevæger Emacs sig fra hvor point er
ned over alle matches og spørger om de skal erstattes, hvis man trykker
\code{<space>} eller \code{y} bliver teksten erstattet, hvis man trykker
\code{n} går man videre til næste match uden at erstatte det nuværende. Hvis man
trykker \code{!} erstattes alle følgende matches uden at spørge.


\documentkey{C-z}{Undo} Fortryder den seneste ændring, hver opmærksom på at
Emacs måde at håndtere undos på er lidt anderledes. Hvis man i et ``normalt''
program gør handling A, B, og så C, og trykker undo og nu gør handling D, vil
historikken huske A, B og D, undo og redo kan således kun komme frem og tilbage
mellem disse. Emacs er lidt anderledes, her findes ``redo'' ikke, tilgengæld er
undo en handling der kan undo'es i sig selv. Så hvis man udfører handling A, B og så C,
så trykker man undo og så D, så vil undo-historikken se således ud: A, B, C,
undo C, D. Man kan så bruge undo til at komme tilbage til C. Tilgengæld vil
disse ændringer også være synlige, således kan Emacs huske alt, men det kan godt
være lidt uoverskueligt.

\documentkey{C-t}{Start shell (terminal)} Starter en eshell i Emacs.

\documentkey{C-b}{Switch buffer} Kommer op med en menu der lader dig skifte
buffer i det nuværende vindue.

\documentkey{<menu>}{buffer menu} Menu knappen er den der sidder i højre side
mellem \code{<AltGr>} og den højre \code{<Ctrl>} knap. Den ligner typisk en
lille menu med en cursor (og findes ikke på Mac). Når man trykker på den åbner
den en buffer med en liste af alle åbne buffere i Emacs.
Man kan nu bevæge point ned til at pege på en linje og trykke enter for at åbne
denne buffer.

\documentkey{<f5>}{Close window} Lukker det nuværende vindue. Vær opmærksom på
at hvad du normalt ville kalde et vindue, kalder Emacs et \textit{frame}, mens
Emacs bruger termet ``window'' om dens interne måde at vise flere buffere.
Man kan ikke lukke det sidste/eneste vindue på denne måde.

\documentkey{<f6>}{Split window horizontal} Splitter det nuværende vindue så der
lægges to ved siden af hinanden.

\documentkey{<f7>}{Setup revy windows} Opsætter vinduerne i Emacs så det er
smart i forhold til revyen. Emacs bliver opdelt til ialt to vinduer, det ene er
aktoversigten det andet er hvad man ellers arbejder med.

\documentkey{<f8>}{other window} Skifter fokus til det næste vindue.

\documentkey{C-æ}{other window} Skifter fokus til det næste vindue.

\documentkey{C-ø}{Balance windows} ``Balancerer'' winduerene, så størrelserne
bliver ensartede.


\documentkey{<f1>}{revy manus insert comment} Indætter en kommentar i et ubertex
dokumment, spørger om hvad du gerne vil have indsat i kommentaren, når du
trykker \code{<enter>} indættes en \verb+\comment{}+ med indholdet.
Det kan være en smart måde hvis man står og afvikler en sang til en prøve, og
man lægger mærke til en fejl, hvis man forsøger at rette den med det samme
kommer man muligvis bagud, og hvis man venter glemmer man måske hvor præcis det
var. Hvis man skynder sig at trykke \code{<f1>} og kaster en hurtig kommentar
sammen vil denne blive indsat hvor point er nu, så kan man rette tingene til
senere. Hvis man glemmer det så har man dog ikke ødelagt noget ved et eller
andet forhastet og forfejlet forsøg.


\note{I Emacs bliver Alt knappen af historiske årsager kaldt Meta}

\marginnote{Disse fire genvejstaster gør at du kan holde \code{<Alt>} nede og så
  bruge WASD til at flytte point rundt.}
\documentkey{M-w}{Previous line} Bevæger point til forrige linje.

\documentkey{M-s}{Next line} Bevæger point til næste linje.

\documentkey{M-a}{left char} Bevæger point et tegn til venstre.

\documentkey{M-d}{Right char} Bevæger point et tegn til venstre.



\documentkey{<home>}{revy next} Avancerer revy cursoren fra hvor den er nu
videre til den efterfølgende instruktion eller slide. Hvor pointeren er har
ingen indvirkning. Centrerer derudover cursoren midt på skærmen.

\documentkey{<end>}{revy enter} Eksekverere instruktionen eller det slide der er
under point. Hvis point er udenfor en instruktion eller slide, vil det første
efterfølgende blive eksekveret.

\documentkey{<delete>}{revy blank} ``Blanker'' skærmen. Se \ref{command:revy-blank}.



\newpage
\section{En guide til Linux}
\label{linux_guide}
% Sådan bruger du en terminal
\textbf{cd}
\textbf{ls}
\textbf{cp}
\textbf{mv}

\textbf{ip}
\textbf{ssh}
\textbf{nano}

hvad er /dev/sdX og /dev/sdX1,2,3 osv?
hvordan fungerer IP'er?

xmonad
nævn
\begin{verbatim}
xmonad --recompile
xmonad --restart
\end{verbatim}

\newpage
\section{Installation af Arch Linux}
Dette appendix gennemgår installationen af Arch Linux fra bunden.
Arch Linux er en såkaldt ``Rolling release'' distribution af linux, den er ikke
specielt begynder venlig, men gør hvad man beder den om, hverken mere eller
mindre.

Dette afsnit er ikke et du skal bruge normalt som AV-mand, men dokumenterer
processen om at sætte systemet op fra bunden af, normalt vil der forhåbentligt
være nogle allerede fungerende installationer du kan låne.

Guiden er baseret på
\url{https://wiki.archlinux.org/index.php/Installation_guide} og\\
\url{https://wiki.archlinux.org/index.php/Beginners'_guide}, og antager at du
har styr på linux, se evt. Appendix \ref{linux_guide}.
\note{\url{http://www.muktware.io/arch-linux-guide-the-always-up-to-date-arch-linux-tutorial/} \url{http://www.dedoimedo.com/computers/grub-2.html}}
Husk at der er ``auto completion'' på tab-knappen.

\subsection{Live USB}
Først downloades \texttt{.torrent} filen fra
\url{https://www.archlinux.org/download/} og åbnes med dit yndlings torrent
program, f.eks. \texttt{rtorrent}.

Når \texttt{.iso} filen er hentet kan denne brændes til et usbstick.
\url{https://wiki.archlinux.org/index.php/USB_flash_installation_media}

For at lave en live USB fra et allerede eksisterende linux system sættes USB'en
i, uden at mounte den (eller unmount den).

Følgende kommando sletter alt på usbsticket og laver et live USB
\begin{verbatim}
sudo dd bs=4M if=/path/to/archlinux.iso of=/dev/sdX status=progress && sync
\end{verbatim}

Hvor \texttt{/path/to/archlinux.iso} er stien til \texttt{.iso} filen.
\texttt{/dev/sd}\textbf{X} er stedet hvor USB'en findes, f.eks. /dev/sdb, bemærk
at det er uden partitionen, så IKKE \texttt{/dev/sdb1}.

Sæt nu USBen i maskinen der skal installeres, og start op fra USB'en.

\begin{mdframed}[style=boxy]
For at gendanne Live USBsticket til et ``normalt'' USB stick
\begin{verbatim}
dd count=1 bs=512 if/dev/zero of=/dev/sdX && sync
cfdisk /dev/sdX
\end{verbatim}

Hvis filsystemet Ext4 ønskes:
\begin{verbatim}
mkfs.ext4 /dev/sdX1
e2label /dev/sdX1 USB_STICK
\end{verbatim}
Eller for et klassisk Windows Fat32 system
\begin{verbatim}
mkfs.vfat -F32 /dev/sdX1
dosfslabel /dev/sdX1 USB_STICK
\end{verbatim}

For at lave et Fat32 system kræver det at \texttt{dosfstools} er installeret.
cfdisk bruges til at lave en partition på USB'en, som findes bå
\texttt{/dev/sdX1}, \texttt{USB\_STICK} er navnet der ønskes på USB'sticket.
\end{mdframed}

\subsection{Installation af styresystem}

Hvis man ønsker et andet tastatur layout det gøres med f.eks.
\code{loadkeys colemak} hvis man ønsker at slå bip-lyden fra kan man gøre det
med \code{rmmod pcspkr}

\begin{verbatim}
timedatectl status
\end{verbatim}
Vær sikker på at tiden er indstillet korrekt, ovenstående skal sige
at \textbf{Universal time} er korrekt, din lokale tidszone kan indstilles senere.
Hvis tiden ikke passer ændres den i BIOS'en inden boot.
\note{Brug ntp? \code{timedatectl set-ntp true}}


\subsubsection{Opret forbindelse til internettet}
Hvis det er et helt almindeligt trådet netværk virker det muligvis uden
problemmer
Hvis maskinen er på et seperat netværk uden internet, men med en gateway som
brugt til den normalle revyopsætning gøres følgende:
\begin{verbatim}
ip link show
ip link set enpXsX up
ip addr add 192.168.0.XXX/24 dev enpXsX
ip route add default via 192.168.0.YYY
echo nameserver 8.8.8.8 > /etc/resolv.conf
\end{verbatim}
\texttt{XXX} er ip'en til maskinen, \texttt{YYY} er ip'en til gateway'en
\note{Henvis til gateway}

Hvis det er et almindeligt trådløst netværk benyttes \code{wifi-menu} eller
\texttt{wpa\_supplicant} for mere avancerede opsætninger

Test om der er internet:
\begin{verbatim}
ping google.com
\end{verbatim}
\textit{tryk Ctrl-c for at stoppe.}

\newpage
\subsubsection{Opret partitioner}
Hele situationen omkring UEFI vs. BIOS, GPT vs. MBR, kombineret med bootloaderen
(grub) og hardwareunderstøttelse er noget gøjl.
Det er et virvar af forskellige systemer der ikke altid har lyst til at
samarbejde, ikke fortæller hvorfor det ikke virker, og mangler dokumentation om
hvordan man får det til at virke.
I en periode, da det hele var nyt, og denne guide blev skrevet i første omgang,
var dokumentationen nærmest ikke eksisterende og meget af det her er opdaget ved
manuel afprøvning.

De næste par afsnit handler om forskellige måder jeg har prøvet at sætte
systemer op på.
Oprindeligt brugte jeg GPT-BIOS, da det var hvad jeg endelig fik til at virke,
men prøv dig frem.

\subsubsection{Opret partitioner (MBR-BIOS)}
\begin{verbatim}
fdisk /dev/sda
\end{verbatim}
med \code{parted -l /dev/sda} kan man se om det er GPT eller ej (også kaldet msdos)
tryk \texttt{o} for at skifte type til MBR.
Lav en swap partition og root + evt. home partition
sæt resten op som et GPT-BIOS system
\subsubsection{Opret partitioner (GPT-BIOS)}
Følgende beskriver hvordan jeg plejer/plejede at oprette partitioner, se det
alternative afsnit \ref{gpt_uefi} hvis et UEFI system ønskes.
Jeg tivler på at dette er den korrekte måde at gøre det på, men det virker på
ældre maskiner.

\code{lsblk} viser alle partitioner på alle diske, alternativt brug \code{fdisk -l}

\note{Følgende sletter alt indhold på harddisken og alle eksisterende partitioner}

\begin{mdframed}[style=boxy]
  Man kan slette alt indholdet på en disk med
\begin{verbatim}
sgdisk --zap-all /dev/sdX
\end{verbatim}
\end{mdframed}

\begin{verbatim}
cgdisk /dev/sdX
\end{verbatim}

Formententligt \texttt{sd\textbf{a}}

Tryk enter. (til alt ``brok'')

\texttt{Delete} alle partitoner.

\textbf{Lav plads til GPT partition}
\textit{Dette giver plads til en partions tabel til grub}\
\begin{enumerate}
\item \texttt{New}
\item Tryk enter, first sector skal bare  være default.
\item \texttt{1M} enter \textit{ - tidligere 1007KiB}
\item \texttt{ef02} enter
\item Tryk enter, der behøver ikke at være et navn
\end{enumerate}

\textbf{Lav swap}
\textit{Dette er næppe nødvendigt, men jeg laver den af gammel vane. Alternativt
kunne man lave en swap fil, men en partion er simplest}
\begin{enumerate}
\item Tryk ned. (til det store område med free space.)
\item \texttt{New}
\item Tryk enter, default er fint, formententligt 2048.
\item \texttt{3G} mindre swap kan også vælges (eller udelades).
\item \texttt{8200} swap Hex koden.
\item tryk enter.
\end{enumerate}

\textbf{Lav en partition}
Hvis flere seperate partitioner ønskes, f.eks. opdelt \texttt{/} og
\texttt{/home} oprettes de her.
\begin{enumerate}
\item Tryk ned.
\item \texttt{New}
\item enter
\item enter
\item enter
\item enter
\end{enumerate}

Vælg \texttt{Write}, skriv \texttt{yes}, og afslut.

Nu kan der stå at den gamle partitionstabel stadig er i brug. I så fald genstart og udfør alle trinene inden \texttt{cgdisk} igen.

Når du er klar skal vi så lave nogle filsystemer.
For at få overblik over partitionerne:
\begin{verbatim}
lsblk
\end{verbatim}

\begin{verbatim}
mkfs.ext4 /dev/sda3
mkswap /dev/sda2
swapon /dev/sda2
\end{verbatim}
(Hvis flere partioner blev oprettet til home og root, så gentag øverste linje
for disse partioner)

Ignorer \texttt{sda1} indtil videre.

\begin{verbatim}
mount /dev/sda3 /mnt
\end{verbatim}

Hvis du skulle have lavet en seperat partition til home så lav en mappe
\code{mkdir /mnt/home} og mount home partitionen der \code{mount /dev/sdaX
  /mnt/home}.

\newpage
\subsubsection{(GPT-UEFI)}
\label{gpt_uefi}
Dette er en lidt simplere gennemgang af at sætte et GPT-UEFI system op.
Dette er den nye måde at gøre tingene på som understøttes af moderne hardware.

\code{lsblk} viser alle partitioner på alle diske, alternativt brug \code{fdisk -l}

\note{Følgende sletter alt indhold på harddisken og alle eksisterende partitioner}

\begin{mdframed}[style=boxy]
  Man kan slette alt indholdet på en disk med
\begin{verbatim}
sgdisk --zap-all /dev/sdX
\end{verbatim}
\end{mdframed}

\begin{verbatim}
cgdisk /dev/sdX
\end{verbatim}

Formententligt \texttt{sd\textbf{a}}

Tryk enter. (til alt ``brok'')

\texttt{Delete} alle partitoner.

\textbf{Lav plads til GPT partition}
\textit{Dette giver plads til en EFI partitionstabel}\
\begin{enumerate}
\item \texttt{New}
\item Tryk enter, first sector skal bare  være default.
\item \texttt{512M} enter
\item \texttt{ef00} enter
\item Tryk enter, der behøver ikke at være et navn
\end{enumerate}

\textbf{Lav swap}
\textit{Dette er næppe nødvendigt, men jeg laver den af gammel vane. Alternativt
  kunne man lave en swap fil, men en partion er simplest}
\begin{enumerate}
\item Tryk ned. (til det store område med free space.)
\item \texttt{New}
\item Tryk enter, default er fint, formententligt 2048.
\item \texttt{3G} mindre swap kan også vælges (eller udelades).
\item \texttt{8200} swap Hex koden.
\item tryk enter.
\end{enumerate}

\textbf{Lav en partition}
Hvis flere seperate partitioner ønskes, f.eks. opdelt \texttt{/} og
\texttt{/home} oprettes de her.
\begin{enumerate}
\item Tryk ned.
\item \texttt{New}
\item enter
\item enter
\item enter
\item enter
\end{enumerate}

Vælg \texttt{Write}, skriv \texttt{yes}, og afslut.

Nu kan der stå at den gamle partitionstabel stadig er i brug. I så fald genstart og udfør alle trinene inden \texttt{cgdisk} igen.

Når du er klar skal vi så lave nogle filsystemer.
For at få overblik over partitionerne:
\begin{verbatim}
lsblk
\end{verbatim}

\begin{verbatim}
mkfs.fat -F32 /dev/sda1
mkfs.ext4 /dev/sda3
mkswap /dev/sda2
swapon /dev/sda2
\end{verbatim}
(Hvis flere partioner blev oprettet til home og root, så gentag øverste linje
for disse partioner)

\begin{verbatim}
mount /dev/sda3 /mnt
mkdir /mnt/boot
mount /dev/sda1 /mnt/boot
\end{verbatim}

Hvis du skulle have lavet en seperat partition til home så lav en mappe
\code{mkdir /mnt/home} og mount home partitionen der \code{mount /dev/sdaX
  /mnt/home}.

Du kan bruge \code{lsblk} til at få overblik over partitionerne, og hvor
de er mounted.
\newpage
\subsubsection{Installation}
Sørg for at være på nettet da vi nu skal hente pakker ned til styresystemet.
\begin{verbatim}
pacstrap -i /mnt base base-devel
\end{verbatim}
Tryk enter til spørgsmål.

\begin{verbatim}
genfstab -U -p /mnt >> /mnt/etc/fstab
arch-chroot /mnt /bin/bash
nano /etc/locale.gen
\end{verbatim}

Fjern kommenteringen til linjerne \textit{\#da\_DK.UTF-8 UTF-8} og\textit{\#en\_US.UTF-8 UTF-8}
\note{Henvis til brug af nano}

\begin{verbatim}
locale-gen
echo LANG=en_US.UTF-8 > /etc/locale.conf
export LANG=en_US.UTF-8
ln -s /usr/share/zoneinfo/Europe/Copenhagen /etc/localtime
hwclock --systohc --utc
echo XXX > /etc/hostname
\end{verbatim}
hvor XXX er navnet til maskinen

\note{Hvis du hellere vil arbejde med at netværksinterfacene hedder
  \texttt{eth0} osv. istedet for enpXsX:
  \code{touch /etc/udev/rules.d/80-net-setup-link.rules}}

\note{?}
\begin{mdframed}[style=note]\textbf{kopieret fra noter:}
Brug netctl til automatisk at gå på netværk
(Ret i hovrdan man bruger deet i resten af dokumentationen)
\begin{verbatim}
curl "https://raw/githubusercontent.com/Pilen/ubertex/master/linux/revynet" >
 /etc/netctl/revynet"
nano /etc/netctl/revynet
\end{verbatim}
Og så skal det gerne matche følgende:
\begin{mdframed}[style=code]
  \verbatiminput{linux/revynet}
\end{mdframed}
Hvor XXX erstates med den lokale IP, og YYY med gateway'ens.

\begin{verbatim}
netctl enable revynet
\end{verbatim}
Hvis man ændrer i configurationen, f.eks. IP'en:
\begin{verbatim}
netctl reanable revynet
netctl restart revynet
\end{verbatim}
\end{mdframed}

Hvis maskinen skal kunne gå på trådløst netværk efterfølgende;
\begin{verbatim}
pacman -S iw wpa_supplicant dialog
\end{verbatim}

\begin{verbatim}
mkinitcpio -p linux
\end{verbatim}
\begin{verbatim}
passwd
\end{verbatim}
Skriv løsn til root (f.eks. \texttt{hamsterroot}).

\note{Du kan med fordel oprette din bruger her}
\note{Du kan med fordel sætte sudo op her}
\note{Du kan med fordel installere sshd her}

For et GPT-BIOS system:
\begin{verbatim}
pacman -S grub
grub-install --target=i386-pc --recheck --debug /dev/sda
grub-mkconfig -o /boot/grub/grub.cfg
\end{verbatim}
For et GPT-UEFI system:
\begin{verbatim}
mkdir /boot/efi
pacman -S grub efibootmgr
grub-install --target=x86_64-efi --efi-directory=esp --bootloader-id=grub
grub-mkconfig -o /boot/grub/grub.cfg
\end{verbatim}
\note{
Den vil klage og sige: \texttt{efibootmgr: EFI variables are not supported on
  this system}, ignorer dette?}

Vi er nu klar til at genstarte op i det nyinstallerede system:
\begin{verbatim}
exit
umount -R /mnt
shutdown -h now
\end{verbatim}
\note{?}
\begin{mdframed}[style=note]\textbf{Note:}
  Hvis grub-mkconfig fejler:
\begin{verbatim}
nano /etc/default/grub
...
...
#fix broken grub.cfg gen
GRUB_DISABLE_SUBMENU = y
\end{verbatim}
\end{mdframed}

Hiv usbstikket ud.

Tænd datamaten igen.
Hvis du som mig til jubilæumsrevyen installerede systemet på en harddisk i en
anden datamat en dens egen, kan det være at ramdisken fejler, i grub vælges da
fallback løsningen og du kalder \code{mkinitcpio -p linux} for at skabe et nyt
korrekt image.

\subsection{Opsætning}
\subsubsection{Bruger}
\note{Hvis en bruger og sudo er sat op kan man logge ind som dem, ellers så log
  ind som root}

Log ind som root

\note{Kom på nettet igen}

\begin{verbatim}
ip link show
\end{verbatim}
For at se netværks interfacet

\begin{verbatim}
useradd -m -s /bin/bash revy
\end{verbatim}
\begin{verbatim}
passwd revy
\end{verbatim}
Giv revy et løsn.

Sudo er allerede installeret via base-devel, du kan læse mere om sudo i Appendix \ref{linux_guide}.
\begin{verbatim}
visudo
\end{verbatim}
Tryk pil ned til du finder linjen
\begin{verbatim}
root ALL=(ALL) ALL
\end{verbatim}
placer cursoren under denne og tryk \code{i}
tast \texttt{revy ALL=(ALL) ALL} tryk enter, tryk escape
tryk \text{:wq} enter.

revy kan nu sudo'e.

\begin{verbatim}
pacman -S openssh
\end{verbatim}
\note{Det er nu default at \texttt{PermitRootLogin} er sat til \texttt{no}, så
  det er ikke længere nødvendigt at ændre det i \texttt{/etc/ssh/sshd\_config}}
\begin{verbatim}
systemctl enable sshd.service
systemctl start sshd
\end{verbatim}
Nu kan vi ssh'e ind fra vores lokale maskine \code{ssh revy@192.168.0.XXX}.
\textit{Læs evt. afsnittet om ssh-nøgler.}

Nu kan vi vælge at køre kommandoerne via en ssh forbindelse.
Du kan også genstarte og logge ind som almindelig bruger istedet, flere af
kommandoerne kræver da sudo.

\subsubsection{Grafisk system}
Nu skal vi installere et grafisk interface
\begin{verbatim}
pacman -S xorg-server xorg-server-utils xorg-xinit
\end{verbatim}
Den spørger nu hvilken udbyder af \texttt{libgl} der ønskes, default (1
mesa-libgl) er formententligt et fint valg (medmindre andet ønskes). (tryk 1
efterfulgt af enter)

Den spørger så om hilken udbyder af \texttt{xf86-input-driver} der ønskes,
1: \texttt{evdev} eller 2: \texttt{libinput}. Libinput er et nyere wayland projekt der
også virker til xorg. Begge er tilsyneladende fine, jeg valgte evdev da det er
det ``klassiske'' valg.

Der skal formententligt installeres nogle grafik drivere, læs mere her \url{https://wiki.archlinux.org/index.php/Xorg#Driver_installation}
\texttt{fbdev} og \texttt{vesa} er nogle udemærkede open source fallback drivere hvis
ikke der findes et grafikkort (bemærk at det fører til software rendering).
Hvis det er et intel kort, er intel driveren et godt bud, for nvidia er nouveau
driverne et godt open source bud

\note{mesa og mesa-libgl er installeret af xorg-server}
\begin{verbatim}
pacman -S xf86-video-fbdev xf86-video-vesa xf86-video-intel
\end{verbatim}
Derudover er det en god idé med Hardware video acceleration
\begin{verbatim}
pacman -S libav-intel-driver libav-mesa-driver
\end{verbatim}

% Der kan kun køre én instans af pacman af gangen.
% Så sæt ham til at arbejde mens vi i en anden terminal begynder at konfigurere.
Som windowmanager bruger vi xmonad, en simpel ``tiling'' windowmanager med
minimale vinduesdekorationer, kan styres med tastaturet og fordi jeg kender den.
Se Appendix \ref{linux_guide} for mere om brugen.

\begin{verbatim}
pacman -S xmonad xmonad-cotrib
\end{verbatim}

Vi skal oprette en \texttt{.xinitrc}, den kan hentes sådan
\begin{verbatim}
curl "https://raw/githubusercontent.com/Pilen/ubertex/master/linux/.xinitrc" > .xinitrc
chown revy:revy .xinitrc
nano .xinitrc
\end{verbatim}
Indholdet skal da være:
\begin{mdframed}[style=code]
  \verbatiminput{linux/.xinitrc}
\end{mdframed}
Den usynlige cursor hentes med:
\begin{verbatim}
curl "https://raw/githubusercontent.com/Pilen/ubertex/master/linux/.emptycursor.xbm" > .emptycursor.xbm
\end{verbatim}

\begin{verbatim}
mkdir .xmonad
curl "https://raw/githubusercontent.com/Pilen/ubertex/master/linux/xmonad.hs" > .xmonad/xmonad.hs
chown -R revy:revy .xmonad
\end{verbatim}
\begin{mdframed}[style=code]
  \verbatiminput{linux/xmonad.hs}
\end{mdframed}
\note{Skal borderWidth være 0?}
\begin{verbatim}
xmonad --recompile
\end{verbatim}

Åben \texttt{.bashrc} og tilføj i bunden:
\begin{verbatim}
if [[ -z $DISPLAY && $(tty) = /dev/tty1 ]]; then
    exec startx
fi
\end{verbatim}
Du kan evt. også tilføje følgende alias over linjerne:
\begin{verbatim}
alias d='export DISPLAY=:0'
\end{verbatim}

For at systemet automatisk kan logge ind ved opstart kan man gøre følgende
\begin{verbatim}
curl "https://raw/githubusercontent.com/Pilen/ubertex/master/linux/autologin.conf" > /etc/systemd/system/getty@tty1.service.d/autologin.conf
nano /etc/systemd/system/getty@tty1.service.d/autologin.conf
\end{verbatim}
indhold af \texttt{autologin.conf}:
\begin{mdframed}[style=code]
  \verbatiminput{linux/autologin.conf}
\end{mdframed}
\note{I noterne står det som \code{curl
    "https://raw/githubusercontent.com/Pilen/ubertex/master/linux/autologin.conf"
    | sudo tee /etc/systemd/system/getty@tty1.service.d/autologin.conf}
  Outputet fra curl kan skjule forespørgslen efter løsen, så hvis den hænger er
  det nok derfor}
\subsubsection{Pakker}
\begin{verbatim}
pacman -S alsa-utils rxvt-unicode dmenu xdotool feh mplayer ttf-dejavu screen htop rsync mupdf
\end{verbatim}

\newpage
\section{Worker-protokol}
\note{Her burde følge en introduction af protokollen mellem master og worker.}


% \section{Elisp kode}
% \includepdf[pages=-]{elisp-source.pdf}
% \section{C kode}
% \includepdf[pages=-]{c-source.pdf}

\newpage
\thispagestyle{empty}
{\huge Husk at slukke projektorerne når du går!}

Det forværer deres holdbarhed at lade dem stå tændt en hel revy-uge.

\end{document}

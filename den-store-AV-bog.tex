\documentclass[10pt,a4paper,danish]{article}
\usepackage[danish]{babel}
\usepackage[utf8]{inputenc}
\usepackage{amsmath}
\usepackage{amssymb}
\usepackage{listings}
\usepackage{fancyhdr}
\usepackage[hidelinks]{hyperref}
\usepackage{booktabs}
\usepackage{graphicx}
\usepackage{xfrac}
\usepackage[dot, autosize, outputdir="dotgraphs/"]{dot2texi}
\usepackage{tikz}
\usepackage{ulem}
\usetikzlibrary{shapes}

\pagestyle{fancy}
\fancyhead{}
\fancyfoot{}
\rhead{\today}
\rfoot{\thepage}
\setlength\parskip{1em}
\setlength\parindent{0em}

%% Titel og forfatter
\title{Den Store AV-bog}
\author{Søren Pilgård, 190689, vpb984}

%% Start dokumentet
\begin{document}

%% Vis titel
\maketitle
\newpage

%% Vis indholdsfortegnelse
\tableofcontents
\newpage

\section{Sådan er man en god AV mand}
\section{Teorien bag}
\section{Opsætning af tekniken}
\subsection{Projektorer}
Til DIKUrevyen bruges der typisk i hvertfald 2 projektorer.
Der bruges 1 der peger op på lærredet over scenen, her vises der overtekster,
almindelige av-ting til sketches samt små film.

Kantinen ejer en stor Epson som bruges til introen/av på scenen


\subsection{Netværk}
\subsubsection{Statisk IP}
Hvis dit netværks interface hedder enp0s25
\begin{verbatim}
sudo ip link set enp0s25 up
sudo ip addr add 192.168.0.XXX/24 dev enp0s25
\end{verbatim}

alternativt
\begin{verbatim}
sudo ifconfig eth0 192.168.0.XXX
\end{verbatim}

\subsubsection{Hosts}
Da det kan være svært at huske alle ip-addreserne kan man i stedet navngive
maskinerne.
Dette gøres lokalt i \texttt{/etc/hosts}
et eksempel:

\begin{verbatim}
192.168.0.20    intro
192.168.0.30    brok
192.168.0.40    left
192.168.0.50    mid
192.168.0.60    right
\end{verbatim}

Nu kan man ssh'e ind ved blot at skrive:

\begin{verbatim}
ssh brok
\end{verbatim}

Det virker også med scp ol.

\subsubsection{Forbindelse uden login}
For at logge ind på systemerne over netværket bruges ssh.
Da det bliver jævnt irriterende hele tiden at skulle taste løsener kan man lægge
sin offentlige ssh nøgle ind på de forskellige maskiner
\begin{verbatim}
cat .ssh/authorized_keys | ssh revy@192.168.0.30 "cat >> ~/.ssh/authorized_keys"
\end{verbatim}
det kan være du først skal oprette mappen \texttt{.ssh}.

alternativt kan du bruge
\begin{verbatim}
ssh-copy-id revy@192.168.0.30
\end{verbatim}
Udskift \texttt{pilen} og ip'en med den relevante bruger og ip.
Husk at du skal have en ssh nøgle først
dannes med:

\texttt{ssh-keygen}

\subsubsection{ssh mount}
I stedet for at scp'e alt muligt crap frem og tibage kan man benytte sig af et
ssh mount, som er et filsystem over ssh.
Jeg anbefaler at man har filerne liggende lokalt og fra hver maskine der skal
kommunikeres med køres \texttt{sshfs} som får filerne frem.

\textbf{Installer sshfs}
F.eks.:
\begin{verbatim}
sudo pacman -S sshfs}
\end{verbatim}

\textbf{fuse}
\texttt{sshfs} benytter fuse
\begin{verbatim}
sudo modprobe fuse
\end{verbatim}
\subsubsection{Gateway}
Hvis du er intereseret i at få internet på det lokale netværk kan man lave en
gateway.
Dette består i at en af datamaterne på det lokale netværk også er forbundet til
et netværk med internet, f.eks. eduroam.

Der er en guide på \url{sigkill.dk/writings/guides/gateway.html}

På gateway datamaten sættes først den statiske ip (allerede gjordt i forrige
trin)
Her efter køres gateway.sh som root (hentes på siden)

På clienterne der vil udnytte gatewayen sørger man først for at der er en
statisk ip.

herefter skrives
\begin{verbatim}
route add default gw 192.168.0.XXX
\end{verbatim}

eller
\begin{verbatim}
ip route add default via 192.168.0.XXX
\end{verbatim}
Hvor ``192.168.0.XXX'' er ip'en til gatewaymaskinen

herefter sættes en dns server, f.eks. google
\begin{verbatim}
echo nameserver 8.8.8.8 > /etc/resolv.conf
\end{verbatim}

Hvis dette ikke virker kan man evt. kigge på
\url{https://wiki.archlinux.org/index.php/Internet_Share} for en ip baseret løsning

\subsection{Brok}
Sæt ham til
Projektor, tastatur, netværk og strøm.

Hvis der skal spilles lyd fra brok bør der være jord på da der ellers kan komme
en brummen.

Brok har på skrivende tidspunkt ubunut 11.04 natty.
Det betyder også at der er gnome, unity og ting og ækle sager på.
Disse har en tendens til at gøre livet surt, da de overskriver xorg
konfigurationer og gør mærkelige ting man ikke helt kan gennemskue.

\textit{Dette er \uline{IKKE} optimalt!
  Support til 11.04 udløb oktober 2012. Der bør installeres et ordentligt
  styresystem. UDEN gui (installeres seperat).}

\textbf{1: Tænd brok}\\
Brok logger automatisk ind med brugeren \textbf{\textit{revy}} med løsnet \textbf{\textit{hamster}}
Herefter starter unity der har en `kør' dialog ting.

\textbf{2: Start en terminal}\\
Skriv \texttt{terminal} og tryk enter.
Der vil nu komme en terminal op i hjørnet.

\textbf{3: Kom på netværket}\\
\begin{verbatim}
sudo ifconfig eth0 192.168.0.30
\end{verbatim}
Giv ham en passende ip du kan huske.

\textbf{4: Opret SSH forbindelse}
Der kører allerede en ssh daemon.
forbind til \textbf{\textit{revy}} med løsnet \textbf{\textit{hamster}}.
Du kan evt. uploade din offentlige ssh-nøgle så du slipper for at logge ind.
\begin{verbatim}
ssh revy@192.168.0.30
\end{verbatim}

\textbf{5: Stop gdm}\\
Gnome Desktop Manager skal lukkes.
\begin{verbatim}
sudo service gdm stop
\end{verbatim}
Dette lukker hele den grafiske grænseflade, inklusiv X.

\textbf{6: Start screen}\\
\begin{verbatim}
screen
\end{verbatim}
screen køres så vi kan starte X i baggrunden.
Når programmet startes vises der en menutekst, bare tryk enter.
\texttt{screen} kører nu.

\textbf{7: startx}\\
Start X manuelt.
\begin{verbatim}
startx
\end{verbatim}
Dette starter X op hvorefter .xinitrc eksekveres.

Som det er lige nu startes \texttt{xmonad} som windowmanager.
Det betyder at skærmen pr. default er sort når x er startet


\textbf{8: detach fra screen}\\
Tryk \texttt{Ctrl-a d}. Dette lader alt der kører i screen fortsætte upåvirket
mens du kommer tilbage til det forrige miljø.
for at reattache skriv da \texttt{screen -r}.

\textbf{9: test xmonad}\\
Test om xmonad virker. På broks tastatur trykkes \texttt{Alt-Shift-Enter}.
En terminal burde åbne sig.
Luk igen med \texttt{Alt-Shift-c}.

\textit{Læs evt. afsnitet om xmonad.}

Da terminalen er en default \texttt{xterm} med hvid baggrund er dette et snedigt trick
til at se hele området projektoren kan lyse op.
Jeg bruger dette ofte når jeg tweaker projektoren.



\textbf{10: prøv ting af}
Med en ssh forbindelse åben, lad os prøve om vi kan få noget til at virke.
Start med at vælge det `display' der skal vises grafiske ting på
\begin{verbatim}
export DISPLAY=:0
\end{verbatim}
Dette skal køres for hver gang du ssh'er ind.

start nu xpdf med en af de gamle overtekst filer der ligger et eller andet sted.

Se sektionen om kommandoer for nærmere detaljer

Hvis man gerne vil lave flere ting simultant på brok kan man sagtens åbne flere
lokale terminaler og ssh'e ind parallelt.




\subsubsection{Ting der køres på brok}
For at hacke uden om alt muligt gøjl køres et par scripts gennem
\texttt{.xinitrc}

\textbf{swarp}\\
\texttt{swarp} er et program der flytter musse-markøren (cursoren) til et givent koordinat.
Swarp findes på \url{tools.suckless.org/swarp}.
\texttt{swarp} findes desuden i arch's repository, det kan være den også findes til
debian baserede styresystemer, potientielt i en suckless pakke.

\texttt{swarp} køres på brok med argumenterne 20000 20000, hvilket flytter
markøren ad helvede til.

istedet for \texttt{swarp} kan man bruge \texttt{unclutter}
med kommandoen \texttt{unclutter -idle 0 -root \&}\\
Som i fjern cursoren efter 0 sekunders delay efter bevægelse inklusiv når
cursoren er over rod baggrunden (altså ikke kun over vinduer).

\textbf{fixsleep}
\texttt{fixsleep.sh} er et script der forsøger at forhindre X i at slukke skærm outputet.
X bruger et system der hedder DPMS (Display Power Management Signaling) til
automatisk at slukke skærm outputet efter en periode uden tastaturaktivitet.

fixsleep benytter følgende to kommandoer
\begin{verbatim}
xset dpms 0 30000 40000
xset s 30000
\end{verbatim}
den første linje dækker over \texttt{xset dpms [standby [suspend [off]]]}
den anden linje dækker over \texttt{xset s [timeout [cycle]]}

\textbf{keepon}
\begin{verbatim}
xset dpms force on
\end{verbatim}
Denne kommando forcer dpms fra, (det svarer til at at trykke på en tast)

\texttt{keepon.sh} er et script der ligger i baggrunden og kører denne kommando
whert 30 sekund.


fixsleep og keepon forsøger begge at holde dpms stangen ved at lave aktivitet.
Man kan måske istedet bruge
\begin{verbatim}
xset -dpms; xset s off
\end{verbatim}
Til at slå dpms fra.
De to systemer kan dog \uline{ikke} blandes.

Man kan se om dpms er slået til ved at kalde \texttt{xset -q}
\subsection{Xmonad}
Windos/super eller alt som modifier knap.

\section{Komandoer}

\subsection{xpdf}
Overtekster køres i xpdf der åbnes manuelt med:
\begin{verbatim}
xpdf -remote ubertex -fullscreen -mattecolor black -fg black
    -bg black -papercolor black filnavn
\end{verbatim}
Da dette er meget langt kan man istedet bruge aliaset
\begin{verbatim}
p filnavn
\end{verbatim}


\subsection{mplayer}
Til at vise videoer manuelt bruges mplayer:
\begin{verbatim}
mplayer -nolirc -msglevel all=-1 -msglevel statusline=5
    -vo gl2 -autosync 30 -cache 1048576
    -cache-min 99:100 -xy 500 -geometry 49%:40% filnavn
\end{verbatim}
eller aliaset
\begin{verbatim}
m filnavn
\end{verbatim}

\subsubsection{Positionering af mplayer}
Noget om positionering af mplayer her

\section{Installation af Arch}
\begin{verbatim}
loadkeys colemak
ip link show
ip link set enpXsX up
ip route add default via 192.168.0.XXX
echo nameserver 8.8.8.8 > /etc/resolv.conf
\end{verbatim}

\begin{verbatim}
ping google.com
\end{verbatim}
\textit{tryk Ctrl-c for at stoppe.}

\begin{verbatim}
cgdisk /dev/sdX
\end{verbatim}

Formententligt \texttt{sd\textbf{a}}

Tryk enter.

\texttt{Delete} alle partitoner.

\textbf{Lav plads til GPT partition}
\textit{Dette giver plads ti en partions tabel til grub}\
\begin{enumerate}
\item \texttt{New}
\item Tryk enter, first sector skal bare  være default.
\item \texttt{1007KiB} enter
\item \texttt{ef02} enter
\item Tryk enter, der behøver ikke at være et navn
\end{enumerate}

\textbf{Lav swap}
\textit{Dette er næppe nødvendigt, men jeg laver den af gammel vane}
\begin{enumerate}
\item Tryk ned.
\item \texttt{New}
\item Tryk enter, default er fint, formententligt 2048.
\item \texttt{3G} mindre swap kan også vælges (eller udelades).
\item \texttt{8200} swap Hex koden.
\item tryk enter.
\end{enumerate}

\textbf{Lav en partition}
\begin{enumerate}
\item Tryk ned.
\item \texttt{New}
\item enter
\item enter
\item enter
\item enter
\end{enumerate}

Vælg \texttt{Write}, skriv \texttt{yes}.
Nu kan der stå at den gamle partitionstabel stadig er i brug.
Afslut, genstart og udfør alle trinene inden \texttt{cgdisk} igen.

Når du er klar igen skal vi lave nogle filsystemer.
\begin{verbatim}
lsblk
\end{verbatim}
For at få overblik

\begin{verbatim}
mkfs.ext4 /dev/sda3
mkswap /dev/sda2
swapon /dev/sda2
\end{verbatim}

Ignorer sda1 indtil videre.

\begin{verbatim}
mount /dev/sda3 /mnt
\end{verbatim}

Hvis du skulle have lavet en seperat partition til home så lav en mappe
\texttt{mkdir /mnt/home} og mount home partitionen der \texttt{mount /dev/sdaX
  /mnt/home}.

Sørg for at være på nettet da vi nu skal hente pakker ned til styresystemet.
\begin{verbatim}
pacstrap -i /mnt base
\end{verbatim}
Tryk enter til spørgsmål.

\begin{verbatim}
genfstab -U -p /mnt >> /mnt/etc/fstab
arch-chroot /mnt /bin/bash
nano /etc/locale.gen
\end{verbatim}

Fjern kommenteringen til linjerne \textit{\#da\_DK.UTF-8 UTF-8} og\textit{\#en\_US.UTF-8 UTF-8}

\begin{verbatim}
locale-gen
echo LANG=en_US.UTF-8 > /etc/locale.conf
export LANG=en_US.UTF-8
ln -s /usr/share/zoneinfo/Europe/Copenhagen /etc/localtime
hwclock --systohc --utc
echo XXX > /etc/hostname
\end{verbatim}
hvor XXX er navnet til maskinen

\begin{verbatim}
mkinitcpio -p linux
\end{verbatim}
\begin{verbatim}
passwd
\end{verbatim}
Skriv løsn til root.

\begin{verbatim}
pacman -S grub
grub-install --target=i386-pc --recheck --debug /dev/sda
grub-mkconfig -o /boot/grub/grub.cfg
exit
umount -R /mnt
shutdown -h now
\end{verbatim}
Hiv usbstikket ud.

Tænd datamaten igen.
Hvis du som mig til jubilæumsrevyen installerede systemet på en harddisk i en
anden datamat en dens egen, kan det være at ramdisken fejler, i grub vælges da
fallback løsningen og du kalder \texttt{mkinitcpio -p linux} for at skabe et nyt
korrekt image.

\begin{verbatim}
ip link show
\end{verbatim}
For at se netværks interfacet

\begin{verbatim}
useradd -m -s /bin/bash revy
\end{verbatim}
\begin{verbatim}
passwd revy
\end{verbatim}
Giv revy et løsn.


\begin{verbatim}
su revy
nano /home/revy/lan.sh
\end{verbatim}
indtast
\begin{verbatim}
ip link set enpXsX up;
ip addr add 192.168.0.XXX/24 dev enpXsX;

ip route add default via 192.168.0.YYY;
echo nameserver 8.8.8.8 > /etc/resolv.conf;
\end{verbatim}
Gem, luk nano og skriv \texttt{exit}.

\begin{verbatim}
pacman -S sudo
visudo
\end{verbatim}
Tryk pil ned til du finder linjen
\begin{verbatim}
root ALL=(ALL) ALL
\end{verbatim}
placer cursoren under denne og tryk \texttt{i}
tast \texttt{revy ALL=(ALL) ALL} tryk enter, tryk escape
tryk \text{:wq} enter.

revy kan nu sudo'e.

\begin{verbatim}
pacman -S openssh
\end{verbatim}
\begin{verbatim}
systemctl enable sshd.service
systemctl start sshd
\end{verbatim}
Nu kan vi ssh'e ind fra vores lokale maskine \texttt{ssh revy@192.168.0.XXX}.
\textit{Læs evt. afsnittet om ssh-nøgler.}

Nu kan vi vælge at køre ting via ssh forbindelse.


Nu skal vi installere pakker
\begin{verbatim}
pacman -S xorg-server xorg-server-utils xorg-xinit
\end{verbatim}

Der skal formententligt installeres nogle grafik drivere, følgende er nogle
udemærkede open source standard drivere

\begin{verbatim}
pacman -S mesa xf86-video-vesa
\end{verbatim}

Der kan kun køre én instans af pacman af gangen.
Så sæt ham til at arbejde mens vi i en anden terminal begynder at konfigurere.
\begin{verbatim}
pacman -S xmonad xmonad-cotrib alsa-utils rxvt-unicode feh mplayer ttf-dejavu
\end{verbatim}

I en terminal anden åbner du \texttt{.xinitrc} og skriver aller nederst:
\begin{verbatim}
exec xmonad
\end{verbatim}

Åben så \texttt{lan.sh} og i bunden tilføjer du
\begin{verbatim}
xset -dpms; xset s off
\end{verbatim}

Åben \texttt{.bashrc} og tilføj i bunden:
\begin{verbatim}
if [[ -z $DISPLAY && $(tty) = /dev/tty1 ]]; then
    exec startx
fi
\end{verbatim}
Du kan evt. også tilføje følgende alias:
\begin{verbatim}
alias d='export DISPLAY=:0'
\end{verbatim}


\section{Om at lave gode overtekster}
% 6 ord
% Læser i høj grad på ordenes form.
%     Drfr kn mn stdg lse dnne stnng
% Sætningen læses relativt hurtigt, men var den længere ville man skulle læse hele
% teksten.
% Man kan ca overskue

% <billede af en 4 lego brik>
% kan genkendes med det samme

% <billede af en 9'er lego klods>
% man skal tælle
\section{Konvertering fra manuskrift til overtekster}
\section{ubertex}
\section{ubersicht}
\section{Hvordan virker koden}

Der sørges automatisk for at det korrekte major mode køres på .tex og .el filer
\section{FAQ}
ubertex tager ikke højde for pauser i comments, latex gør.
\section{APPENDIX A: En guide til linux}
% Sådan bruger du en terminal
\subsection{cd}
\subsection{ls}
\subsection{cp}
\subsection{mv}

\subsection{ip}
\subsection{ssh}
\subsection{nano}
\section{APPENDIX B: Stripped Emacs}
\section{APPENDIX C: Emacs}
\section{APPENDIX D: mangler}
revy-shell skal escape kommandoer
Youtube


\section{Andet}

husk blanke slides
\end{document}
